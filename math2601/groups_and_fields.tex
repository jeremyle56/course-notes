\section{Group and Fields}
\subsection{Groups}

\paragraph{Definition}
A group \(G\) is a non-empty set with a binary operation defined on it. That is
\begin{enumerate}
    \item \textbf{Closure:} for all \(a,b\) in \(G\) a composition \(a * b\) is defined and in \(G\),
    \item \textbf{Associativity:} \((a * b) * c = a * (b * c)\) for all \(a,b,c \in G\),
    \item \textbf{Identity:} there is an element \(e\in G\) such that \(a * e = e * a\) for all \(a \in G\),
    \item \textbf{Inverse:} for each \(a \in G\) there is an \(a'\) in \(G\) such that \(a * a' = a' * a = e\),
\end{enumerate}
If \(G\) is a finite set then the order of \(G\) is \(|G|\), the number of elements in G.

Groups are defined as \((G, *)\). We say this as ''the group \(G\) under the operation \(*\)''.

\paragraph{Abelian Groups}
A group \(G\) is abelian if the operation satisfies the commutative law
\[a * b = b * a \qquad \text{for all} \quad a,b \in G\]

\paragraph{Notation}
\begin{itemize}
    \item We use power notation for repeated applications: \(a * a \cdots * a = a^n\) and \(a^{-n} = (a^{-1})^n\).
    \item For group operation, \(\times\) we use 1 for the identity and \(a^{-1}\) for inverse of \(a\).
    \item For group operation, \(+\) we use 0 for the identity and \(-a\) for the inverse of \(a\).
    \item We would then write \(na\) for \(a+a+\cdots a\) (repeated addition, not multiplying by \(n\)).
\end{itemize}

\paragraph{Trivial Groups}
The trivial group is the group consisting of exactly one element, \(\{e\}\). It is the smallest possible group, since there has to be at least one element in a group.

\paragraph{More Properties of Groups}
\begin{itemize}
    \item There is only one identity element in \(G\).
    \item Each element of \(G\) only has one inverse.
    \item For each \(a \in G, (a^{-1})^{-1} = a\)
    \item For every, \(a,b \in G, (a *  b)^{-1} = b^{-1} * a^{-1}\).
    \item Let \(a,b,c \in G\). Then if \(a * b = a * c, b = c\).
\end{itemize}

\subsubsection{Permutation Groups}
Let \(\Omega_n = \{1,2,\dots,n\}\). As an ordered set \(\Omega_n = (1,2,\dots,n)\) has \(n!\) rearrangements. We may think of these permutations as being functions \(f:\Omega_n \to \Omega_n\). These are bijections.

Observe that the set \(\mathcal{S}_n\) of all permutations of \(n\) objects forms a group under composition of order \(n!\).

\paragraph{Small Finite Groups}
Small groups can be pictured using a multiplication table, where the row element is multiplied on the left of the column element.

In a multiplication table of finite group each row must be a permutation of the elements of the group, because:
\begin{itemize}
    \item If we had repetition in a row (or column), so that \(xa = xb\), then the cancellation rule will give \(a=b\). Hence each element occurs no more than once in a row (or column).
    \item If \(a^2 = a\) then multiplying by \(a^{-1}\) gives \(a=e\), so the identity is the only element that can be fixed.
\end{itemize}

\subsection{Fields}
A field \((\bF, +, \times)\) is a set \(\bF\) with two binary operations on it, addition (+) and multiplcation (\(\times\)), where
\begin{enumerate}
    \item \((\bF, +)\) is an abelian group,
    \item \(\bF^* = \bF\setminus\{0\}\) is an abelian group under multiplication,
    \item The distributive laws \(a \times (b + c) = a \times b + a \times c\) and \((a+b) \times c = a \times c + b \times c\) hold.
\end{enumerate}

\paragraph{Additional Notes}
\begin{itemize}
    \item Our definition is equivalent to saying \(\bF\) satisfies the \(12 = 5 + 5 + 2\) number laws.
    \item We use juxtaposition for the multiplication in fields and 1 for the identity under multiplication.
    \item The smallest possible field has two elements, and is written \(\{0,1\}\) with \(1 + 1 = 0\).
\end{itemize}

\paragraph{Finite Fields}
The only finite fields are those of size \(p^k\) for some prime \(p\) (referred to as the characteristic of the field) and positive integer \(k\). These fields are called Galois fields of size \(p^k\), \(\operatorname{GF}(p^k)\). Note that \(\operatorname{GF}(p^k) \neq \mbb{Z}_{p^k}\) unless \(k=1\).

\paragraph{Properties of Fields}
Let \(\bF\) be a field and \(a,b,c \in \bF\). Then
\begin{itemize}
    \item \(a0=0\)
    \item \(a(-b) = -(ab)\)
    \item \(a(b - c) = ab - ac\)
    \item if \(ab = 0\) then either \(a = 0\) or \(b = 0\).
\end{itemize}

\subsection{Subgroups and Subfields}
\paragraph{Subgroups}
Let \((G,*)\) be a group and \(H\) a non-empty subset of \(G\). If \(H\) is a group under the restriction of \(*\) to \(H\), we call it a subgroup of \(G\). We write this as \(H \leq G\) and say \(H\) inherits the group structure from \(G\).

\paragraph{The Subgroup Lemma}
Let \((G,*)\) be a group and \(H\) a non-empty subset of \(G\). Then \(H\) is a subgroup of \(G\) if and only if
\begin{enumerate}
    \item for all \(a,b \in H, a * b \in H\)
    \item for all \(a \in H, a^{-1} \in H\).
\end{enumerate}
i.e. \(H\) is closed under \(*\) and \(^{-1}\).

Note that every non-trivial group \(G\) has at least two subgroups: \(\{e\}\) and \(G\).

\paragraph{General Linear Groups}
Let \(n \geq 1\) be an integer. The set of invertible \(n \times n\) matrices over field \(\bF\) is a group under matrix multiplication. This is a special case of a bijection function \(f: S \to S\) with \(S = \bF^n\) and is non-abelian if \(n > 1\).

It is called the general linear group, \(\GL(n,f)\).

The groups \(\GL(n,\bR)\) and \(\GL(n,\bC)\) are especially important in this course.
They have many important subgroups, such as
\begin{itemize}
    \item the special linear groups \(\SL(n,\bR)\) and \(\SL(n,\bC)\) of matrices with determinant 1.
    \item \(O(n) \leq \GL(n,\bR)\) the group of orthogonal matrices.
    \item \(\operatorname{SO}(n) = O(n) \cap \SL(n,\bR)\) of special orthogonal matrices.
\end{itemize}

\paragraph{Subfields}
If \((\bF, +, \times)\) is a field and \(\bE \subseteq \bF\) is also a field under the same operations (restricted to \(\bE\)), then \((\bE, +, \times)\) is a subfield of \((\bF, +, \times)\), usually written \(\bE \leq \bF\).

\paragraph{The Subfield Lemma}
Let \(\bE \neq \{0\}\) be a non-empty subset of field \(\bF\). Then \(\bE\) is a subfield of \(\bF\) if and only iff for all \(a,b \in \bE\):
\[a + b \in \bE, \qquad -b \in \bE, \qquad a \times b \in \bE, \qquad b^{-1} \in \bE \quad \text{\textbf{if}} \quad  b \neq 0.\]

\paragraph{Rational + Irrational Field}
Let \(\alpha\) be any (non-rational) real or complex number. We defined \(\mbb{Q}(a)\) to be the smallest field containing both \(\mbb{Q}\) and \(\alpha\). Such fields are important in number theory and can clearly be generalised to e.g. \(\mbb{Q}(\alpha, \beta)\). For example, it can be shown
\[\mbb{Q}(\sqrt{2}) = \{a + b\sqrt{2}: a,b \in \mbb{Q}\}\]

\subsection{Morphisms}
A morphism is a category of ''nice'' maps between the members.

\paragraph{Homomorphism}
Let \((G, *)\) and \((H, \circ)\) be two groups. A (group) homomorphism from \(G\) to \(H\) is a map \(\phi: G \to H\) that respects the two operations, that is where
\[\phi(a*b) = \phi(a) \circ \phi(b) \quad \text{for all } \, a,b \in G.\]

\paragraph{Isomorphism}
A bijective homomorphism \(\phi: G \to H\) is called an isomorphism: the groups are then said to be isomorphic. That is, \(G \cong H\).

\paragraph{Isomorphism Lemmas}
Let \((G, *)\) and \((H, \circ)\) be two groups and \(\phi\) a homomorphism between them. Then
\begin{itemize}
    \item \(\phi\) maps the identity of \(G\) to the identity of \(H\).
    \item \(\phi\) maps inverses to inverse, i.e. \(\phi(a^{-1}) = (\phi(a))^{-1}\) for all \(a \in G\).
    \item if \(\phi\) is an isomorphism from \(G\) to \(H\) then \(\phi^{-1}\) is an isomorphism from \(H\) to \(G\).
\end{itemize}

\paragraph{Images and Kernel}
Let \(\phi: G \to H\) be a group homorphism, with \(e'\) the identity of \(H\).
The kernel of \(\phi\) is the set
\[\ker(\phi) = \{g \in G: \phi(g) = e'\}\]
The image of \(\phi\) is the set
\[\operatorname{im}(\phi) = \{h \in H: h = \phi(g), \text{ some } g \in G\}.\]
Note that \(\ker\phi \leq G\) and \(\operatorname{im}\phi \leq H\).

\paragraph{One-to-One Homomorphism}
A homomorphism \(\phi\) is one-one if and only if \(\ker\phi = \{e\}\), with \(e\) the identity of \(G\).
If \(\phi\) is one-one then \(\im(\phi)\) is isomorphic to \(G\).

\paragraph{Linear Groups}
A common use of group homomorphisms is to look for a homomorphism \(\phi: G \to \GL(n,\bF)\) for some \(n\) and some field \(\bF\). The group \(\im(\phi)\) is called a (linear) representation of \(G\) on \(\bF^n\). If \(\phi\) is one-one (so every element maps to a distinct matrix), we call the representation faithful.
\section{Vector Spaces}
\subsection{Vector Spaces}

\paragraph{Motivation for Vector Spaces}
The concept of a vector space is a natural and important generalisation of \(\bR^n\). It is natural to consider them whenever possible to add objects and multiply them by scalars.

It may be convenient to consider a field \(\mathbb{F}\) as a vector
space over one of its subfields.

\paragraph{Vector Spaces}
Let \(\bF\) be a field. A vector space over the field \(\bF\) consists of an abelian group \((V, +)\) plus a function from \(\bF \times V\) to \(V\) called scalar multiplication and written \(\alpha\vv\) where
\begin{enumerate}
    \item \(\alpha(\beta\vv) = (\alpha\beta)\vv\) for all \(\alpha, \beta \in \bF\) for all \(\vv \in V\).
    \item \(1\vv = \vv\) for all \(\vv \in V\).
    \item \(\alpha(\vu + \vv) = \alpha\vu + \alpha\vu\) for all \(\alpha \in \bF\) for all \(\vu, \vv \in V\).
    \item \((\alpha + \beta)\vu + \alpha\vu + \beta\vu\) for all \(\alpha, \beta \in \bF\) for all \(\vu \in V\).
\end{enumerate}

\paragraph{Properties and Notation for Vector Spaces}
\(\vec{dsdfa}\)
\begin{enumerate}
    \item There are ten axioms here: 5 from the abelian group, closure of scalar multiplication and the four explicit ones.
    \item Addition in \(V\) is called vector addition to distinguish it from the addition in \(\bF\).
    \item Being a group, \(V\) cannot be empty.
    \item Bold face letters are used to distinguish elements of \(V\) from elements of \(\bF\).
\end{enumerate}

\paragraph{Vector Space Lemma}
Let \(V\) be a vector space over a field \(\bF\). For all \(\vv, \vw\) in \(V\) and \(\lambda \in \bF\):
\begin{enumerate}
    \item \(0\vv = \zero\) and \(\lambda\zero = \zero\).
    \item \((-1)\vv = -\vv\).
    \item \(\lambda\vv = \zero\) implies either \(\lambda = 0\) or \(\vv = \zero\).
    \item if \(\lambda\vv = \lambda\vw\) and \(\lambda \neq 0\) then \(\vv = \vw\).
\end{enumerate}

\subsection{Standard Examples of Vector Spaces}
\paragraph{The Space \(\bF^n\) over \(\bF\)}
The set \(\bF^n\) consists of all \(n\)-tuples of elements of \(\bF\):
\[\bF^n = \left\{ \begin{pmatrix}
    \alpha_1 \\ \vdots \\ \alpha_n
\end{pmatrix}: \alpha_i \in \bF \right\}.\]
If \(\vx = (\alpha_i)_{1 \leq i \leq n}, \vy = (\beta_i)_{1 \leq i \leq n}\) are elements of \(\bF^n\), then vector addition on \(\bF^n\) is defined as 
\[\vx + \vy = (\alpha_i + \beta_i)_{1 \leq i \leq n}.\]
Scalar multiplication on \(\bF^n\) is \(\lambda\vx = (\lambda\alpha_i)_{1 \leq i \leq n}\).

With these operations, \(\bF^n\) is a vector space over \(\bF\).

\paragraph{Geometric Vectors}
Geometric vectors are ordered pairs of points in \(\bR^n\), joined by labelled arrows. We add these objects by placing them head to tail and scalar multiplying is just stretching the vector's length while preserving the direction.

The set of all geometric vectors does not form a vector space. However, if you define 2 geometric vectors to be equivalent if one is a translation of the other then the set of equivalence classes of geometric vectors is a vector space.

\paragraph{Matrices}
For any positive integers \(p\) and \(q\) the set \(M_{p,q}(\bF)\) is the set of \(p \times q\) matrices with element from \(\bF\). Then \(M_{p,q}(\bF)\) is a vector space over \(\bF\) with vector addition the usual addition of matrices and scalar multiplication multiplying each element of the matrix.

\paragraph{Polynomials}
The set of all polynomials with coefficients in \(\bF, \mathcal{P}(\bF)\), is a vector space over \(\bF\) with 
\begin{align*}
(f+g)(x) & = f(x) + g(x) \quad \text{for all } x \in \bF \\
(\lambda f)(x) & = \lambda f(x) \quad \text{for all } \lambda, x \in \bF
\end{align*}
Similarly, \(\mathcal{P}_n(F)\) (polynomials of degree \(n\) or less) is a vector space over \(\bF\).

\paragraph{Function Spaces}
Let \(X\) be a non-empty set and \(\bF\) be a field. Then define
\[\mathcal{F}[X] = \{ f: X \to \bF \}.\]
The set \(\mathcal{F}[X]\) is a vector space over \(\bF\) if we define
\begin{itemize}
    \item the zero in \(\mathcal{F}[X]\) to be the zero function: \(x \to 0\) for all \(x \in X\)
    \item \((f + g)(x) = f(x) + g(x)\) for all \(x \in X\)
    \item \((\lambda f)(x) = \lambda(f(x))\) for all \(x \in X\)
\end{itemize}

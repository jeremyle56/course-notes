\section{Linear Transformations}

\subsection{Linear Transformations}
\paragraph{Linear Transformation}
Suppose \(V\) and \(W\) are vector spaces over the field \(\bF\). A function \(T: V \to W\) is a linear transformation or a linear map (or simply linear) if
\begin{itemize}
    \item \(T(\vu + \vv) = T(\vu) + T(\vv)\), and
    \item \(T(\lambda\vv) = \lambda T(\vv)\),
\end{itemize}
for all \(\vu, \vv \in V\) and for all \(\lambda \in \bF\).

\paragraph{Identity Map, Zero Vector and Negatives}
Let \(V\) and \(W\) be vector spaces over the field \(\bF\).
\begin{itemize}
    \item The identity map, \(\id:V \to V\) defined by \(\id(\vv) = \vv\) is linear.
    \item If \(T: V \to W\) is linear then \(T(\vzero) = \vzero\) and \(T(-\vv) = -T(\vv)\).
\end{itemize}

\paragraph{Linearity Test Lemma}
A function \(T: V \to W\) between vector spaces over the same field \(\bF\) is linear if and only if 
\[T(\lambda\vu + \vv) = \lambda T(\vu) + T(\vv)\]
for all \(\lambda \in \bF\), and \(\vu, \vv \in V\).

\paragraph{Linear Transformations are Vector Spaces}
Let \(V\) and \(W\) be two vector spaces over field \(\bF\). The set \(L(V,W)\) of all linear transformations from \(V\) to \(W\) is a vector space under the operations
\[(S+T)(\vv) = S(\vv) + T(\vv), \qquad (\lambda S)(\vv) = \lambda S(\vv).\]

\paragraph{Composition of Linear Maps}
Let \(T: V \to W\) and \(S:W \to X\) be linear maps between vector spaces. Then \(S \circ T: V \to X\) is also linear.

\paragraph{Linearity of Inverse}
Let \(T: V \to W\) be an invertible linear map between two vector spaces over field \(\bF\). Then \(T^{-1}: W \to V\) is linear.

\paragraph{Invertible Linear Maps are Groups}
The invertible linear maps in \(L(V,V)\) form a group under compositions. Note that composition of maps is always associative so and the inverse exists by definition of \(L(V, V)\), only closure and the identity need to be proved.

Closure exists since composition of linear transformations are vector spaces. The identity map is linear and clearly invertible and so, also exists in the group.

\paragraph{Taking Coordinates is Linear}
Let \(V\) be a (finite-dimensional) vector space over \(\bF\) with a basis \(\mathcal{B} = \{\vv_1, \dots, \vv_p\}\). Then the function \(S: V \to \bF^p\) defined by \(S(\vx) = [\vx]_\mathcal{B}\) is linear.

\subsection{Kernel and Image}
\paragraph{Kernel} 
Let \(T: V \to W\) be a linear transformation. The kernel (or nullspace) of \(T\) is the set
\[\ker T = \{\vv \in V: T(\vv) = \vzero\}.\]

\paragraph{Image}
If \(U \leq V\) then the image of \(U\) is the set
\[T(U) = \{T(\vu) : \vu\in U\}.\]
We also define the image of \(T\) (or range of \(T\)), \(\im(T)\) as the image of all of \(V:\im(T) = T(V)\).

\paragraph{Kernal and Image of Linear Transformations}
Let \(T: V \to W\) be a linear transformation between vector spaces over \(\bF\) and \(U \leq V\). Then
\begin{enumerate}
    \item \(\ker T\) is a subspace of \(V\).
    \item \(T(U)\) is a subspace of \(W\), and so \(\im(T) \leq W\).
    \item If \(U\) is finite-dimensional, so is \(T(U)\), so if \(V\) is finite dimensional, so is \(\im(T)\).
\end{enumerate}

\paragraph{Rank and Nullity}
If \(T\) is a linear transformation, then the dimension of the kernel of \(T\) is called the nullity of \(T\), and the dimension of its image is called the rank of \(T\).

\paragraph{Nullity - One to One}
A linear map \(T: V \to W\) is one-to-one if and only if \(\nullity(T) = 0\).

\paragraph{Rank-Nullity Theorem}
If \(V\) is a finite dimensional vector space over \(\bF\) and \(T: V \to W\) is linear then
\[\rank(T) + \nullity(T) = \dim(V).\]

\paragraph{Bijective, Injective, Surjective}
Let \(V, W\) be vector spaces over \(\bF\) with \(\dim(V) = \dim(W)\) finite and \(T: V \to W\) be linear. The following are equivalent:
\begin{itemize}
    \item \(T\) is invertible (bijective).
    \item \(T\) is one-ton-one (injective) i.e. \(\nullity(T) = 0\).
    \item \(T\) is onto (subjective) i.e. \(\rank(T) = \dim(V)\).
\end{itemize}

\paragraph{Isomorphism}
An invertible linear map \(T: V \to W\) is called an isomorphism of the vector spaces \(V\) and \(W\). 

\paragraph{Isomorphism + Dimensions}
Finite dimension vector spaces \(V\) and \(W\) over \(\bF\) are isomorphic if and only if they have the same dimension.

\subsection{Spaces Associated with Matrices}
\paragraph{Kernel, Image, Nullity and Rank}
Let \(A\) be a \(p \times q\) matrix over field \(\bF\), and define a map \(T:\bF^q \to \bF^p\) by \(T(\vx) = A\vx\). The kernel, image, nullity and rank of \(A\) are by definition the same as those of this map \(T\).

\paragraph{Column Space}
Suppose \(A\) has columns \(\vc_1, \dots, \vc_q\) (all in \(\bF^p\)). Then
\begin{align*}
    \im(A) & = \{A\vx : \vx \in \bF^q\} \\
    & = \{ x_1\vc_1 + \cdots x_q\vc_q : x_i \in \bF \} \\
    & = \span(\{\vc_1, \dots, \vc_q\})
\end{align*}
That is, \(\im(A)\) is the space spanned by the columns of \(A\): the column space of \(A, \col(A)\), a subspace of \(\bF^p\). The rank of \(A\) is thus the dimension of the column space of \(A\).

\paragraph{Rank-Nullity Theorem for Matrices}
For \(A \in M_{p,q}(\bF), \rank(A) + \nullity(A) = q,\) the number of columns of \(A\).

\paragraph{Row Space}
The row space of \(A, \row(A)\), is defined similarly as the space spanned by the rows: it is a subspace of \(\bF^q\). Note that \(\row(A) = \col(A^T) = \im(A^T)\).

\paragraph{Row and Col Spaces}
Let \(A \in M_{p,q}(\bF)\). The spaces \(\row(A)\) and \(\col(A)\) have the same dimension.

\subsection{The Matrix of a Linear Map}
\paragraph{Matrices of Linear Maps}
Let \(V,W\) be two finite dimensional vector spaces over \(\bF\). Suppose \(\dim(V) = q\) and \(V\) has basis \(\mathcal{B}\) and also \(\dim(W) = p\) and \(W\) has basis \(\mathcal{C}\). If \(T: V \to W\) is linear then there is a unique \(A \in M_{p,q}(\bF)\) with
\[[T(\vv)]_\mathcal{C} = A[\vv]_\mathcal{B}.\]
Conversely, for any \(A \in M_{p,q}(\bF)\), the equation defines a unique linear map from \(V\) to \(W\).

\paragraph{Notation}
We call \(A\) in the above theorem the matrix of \(T\) with respect to \(\mathcal{B}\) and \(\mathcal{C}\). A useful notation is to denote this matrix by \([T]_\mathcal{C}^\mathcal{B}\) and then the equation takes the form
\[[T(\vv)]_\mathcal{C} = [T]_\mathcal{C}^\mathcal{B}[\vv]_\mathcal{B}.\]

\paragraph{Dimension of Linear Map}
If \(\dim(V) = q\) and \(\dim(W) = p\) then \(\dim(L(V,W)) = pq\).

\paragraph{Composition of Linear Maps as Matrices}
Let \(T: V \to W\) and \(S: W \to X\) be linear maps between vector spaces and suppose \(V, W\) and \(X\) have bases \(\mathcal{A}, \mathcal{B}, \mathcal{C}\) respectively. Then, the matrix \(S \circ T : V \to X\) is the product of the matrices of \(T\) and \(S\), all taken with respect to the appropriate bases:
\[ [S \circ T]_\mathcal{C}^\mathcal{A} = [S]_\mathcal{C}^\mathcal{B} \cdot [T]_\mathcal{B}^\mathcal{A}.\]

\paragraph{Inverting Matricies as Transformations}
If \(T: V \to W\) is linear and invertible, the matrix of \(T^{-1}\) is the inverse of the matrix of \(T\). Thus the group of invertible linear maps on an \(n\)-dimensional vector space over \(\bF\) is isomorphic to \(\GL(n,\bF)\). Formally,
\[[T^{-1}]_\mathcal{B}^\mathcal{C} = ([T]_\mathcal{C}^\mathcal{B})^{-1}.\]

\paragraph{Change of Basis Matrix}
If vector space \(V\) has two bases \(\mathcal{B}\) and \(\mathcal{C}\), the matrix \([\id]_\mathcal{C}^\mathcal{B}\) of the identity map is called the change of basis matrix (from \(\mathcal{B}\) to \(\mathcal{C}\)). This can be used to change coordinates:
\[[\vv]_\mathcal{C} = [\id]_\mathcal{C}^\mathcal{B}[\vv]_\mathcal{B}.\]

\paragraph{Rank and Nullity of Matrices}
Let \(T: V \to W\) be a linear map between finite dimensional vector spaces over \(\bF\) and \(A\) its matrix with respect to any two bases in \(V\) and \(W\). Then
\[\nullity(A) = \nullity(T) \quad \text{ and } \quad \rank(A) = \rank(T).\]

\paragraph{Invariant Subspace}
Let \(V\) be a vector space over \(\bF\) and \(T: V \to V\) a linear map. If \(X \leq V\) is such that \(T(X) \leq X\), we call \(X\) and invariant subspace of \(T\).

\paragraph{Linear Maps of Invariant Subspaces}
Let \(T: V \to V\) be a linear map on a finite dimensional vector space. Suppose \(V = X \oplus Y\) with both \(X\) and \(Y\) invariant subspaces of \(T\) with dimensions \(p\) and \(q\) respectively. Then there is a basis \(\mathcal{B}\) for \(V\) in which the matrix \([T]_\mathcal{B}^\mathcal{B}\) of \(T\) is of the form
\[[T]_\mathcal{B}^\mathcal{B} = \begin{pmatrix}
    A & \vzero \\
    \vzero & B
\end{pmatrix}
\]
with \(A\) a \(p \times p\) and \(B\) a \(q \times q\) matrix.

\subsection{Isomorphisms Again}
\paragraph{Invertiable Maps and Matrices}
Let \(T: V \to W\) be a linear map between finite-dimensional spaces, and let \(A\) be a matrix of \(T\) (with respect to any bases). Then \(T\) is invertible if and only if \(A\) is invertible. If this is the case, then the matrix of \(T^{-1}\) (with respect to the same bases) is \(A^{-1}\).
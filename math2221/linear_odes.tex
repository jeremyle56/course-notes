\chapter{Linear ODEs}

\section{Introduction}
Recall that a first-order ordinary differential equation (ODE) has, in its most general realisation, the form
\[y'(t) = f(t, y(t)).\]
A special case is the equation
\[a(t)y'(t) + b(t)y(t) = f(t),\]
with \(a(t) \neq 0\) on some interval \(I \in \bR\). This special first-order ODE is called a \textbf{linear first-order ODE}. Another special case is
\[y'(t) = f(t)g(y),\]
which is known as a \textbf{separable first-order ODE}. \\

For a separable equation the solution is found (at least, implicitly by) writing:
\[\int \frac{1}{g(y)} \, dy = \int f(t) \, dt.\]

\exmp{Solving Seperable ODEs}{
    Consider \(y' = t^2 y, y(0) = 3.\) This is seperable with \(f(t) = t^2\) and \(g(y) = y\). Then
    \[\int \frac{1}{y} \, dy = \int t^2 \, dt\]
    so that
    \[\ln \abs*{y(t)} = \frac{1}{3}t^3 + C.\]
    Now apply \(e^t\) to both sides to obtain
    \[\abs*{y(t)} = e^{\frac{1}{3}t^3 + C} = e^C e^{\frac{1}{3} t^3}.\]
    Thus, a general solution of the equation is
    \[y(t) = Ae^{\frac{1}{3}t^3}.\]
    Since \(y(0) = 3\), we see that the unique solution is \(y(t) = 3e^{\frac{1}{3}t^3}\).
}

In the case of a linear first-order equation, i.e. \(y' + a(t)y = f(t)\), a useful solution method is the integrating factor technique. The idea is to find a function \(\mu\) so that when we multiply both sides of the equation with \(\mu\) we find that
\[[\mu y](t)' = \mu(t)(y' + a(t)y) = \mu(t)f(t),\]
for if this happens, then the general solution of the ODE should be
\[y(t) = \frac{1}{\mu(t)}\int \mu(t)f(t) \, dt + \frac{C}{\mu(t)}.\]

\exmp{Solving Linear First-Order ODE}{
    Solve \(y' - 2ty = 3t\). We pick
    \[\mu(t) = e^{\int - 2t \, dt} = e^{-t^2}.\]
    Then
    \begin{align*}
        (e^{-t^2}y)' & = 3te^{-t^2}                                        \\
        e^{-t^2}y    & = \int 3te^{-t^2} \, dt  = -\frac{3}{2}e^{-t^2} + C \\
        y(t)         & = -\frac{3}{2} + Ce^{t^2}.
    \end{align*}
}

\section{Linear Differential Operators}
In linear algebra, you have seen the compact notation \(A\bf{x} = \bf{b}\) for system of linear equations. A similar notation when dealing with a linear ordinary differential equations is
\[Lu = f.\]
Here, \(L\) is an operator (or transformation) that acts on a function \(u\) to create a new function \(Lu\).

Given coefficients \(a_0(x), a_1(x), \dots, a_m(x)\) we define the \textbf{linear differential operator} \(L\) of \textbf{order} \(m\),
\begin{align*}
    Lu(x) & = \sum_{j=0}^{m} a_j(x) D^ju(x)               \\
          & = a_mD^mu + a_{m-1}D^{m-1}u + \cdots + a_o u,
\end{align*}
where \(D^j u = d^j u / dx^j\) (with \(D^0 u = u\)). \\

We refer to \(a_m\) as the \textbf{leading coefficient} of \(L\) and assume that each \(a_j(x)\) is a smooth function of \(x\). \\

The ODE \(Lu = f\) is said to be \textbf{singular} with respect to an interval \([a, b]\) if the leading coefficient \(a_m(x)\) vanishes for any \(x \in [a, b]\). \\
\exmp{Example}{
    \(Lu = (x - 3)u''' - (1 + \cos x)u' + 6u\) is a linear differential of order 3, with leading coefficient \(x - 3\). Thus, \(L\) is singular on \([1, 4]\), but not singular on \([0 ,2]\).
}
\exmp{Example}{
    \(N(u) = u'' + u^2u' - u\) is a nonlinear differential operator of order 2.
}

\paragraph{Linearity} For any constants \(c_1\) and \(c_2\) and any \(m\)-times differentiable functions \(u_1\) and \(u_2\),
\[L(c_1u_1 + c_2u_2) = c_1 Lu_1 + c_2 Lu_2.\]

Ordinary differential equations of the form \(Lu = 0\) are known as \textbf{homogenous}. Those of the form \(Lu = f\) are known as \textbf{inhomogeneous}. \\

When the solution to a differential equation is prescribed at a particular point \(x = x_0\), that is
\[u(x_0) = v_0, \quad u'(x_0) = v_1, \quad \dots, \quad u^{(m - 1)}(x_0) = v_{m - 1},\]
we call it an \textbf{initial value problem}. Where a differential equation is order 2 or greater, solutions at 2 or more locations can be prescribed. Such problems are called \textbf{boundary value problems}. \\

\thrm{Unique Solution to Linear Initial Problem}{
    For an ODE \(Lu = f\) which is not singular with repsect to \(a, b\), with \(f\) continuous on \([a, b]\), the IVP for an \(m\)th-order linear differential operator with \(m\) inital values has a unique solution.
}

\thrm{Solution to mth Order Problem has Dimension m}{
    Assume that the linear, \(m\)th-order differential operator \(L\) is not singular on \([a, b]\). Then the set of all solutions to the homogenous equation \(Lu = 0\) on \([a, b]\) is a vector space of dimension \(m\).
}

\bigskip
If \(\{u_1, u_2, \dots, u_m\}\) is \textbf{any} basis for the solution space of \(Lu = 0\), then every solution can be written in a unique way as
\[u(x) = c_1u_1(x) + c_2u_2(x) + \cdots + c_mu_m(x) \quad \text{for } a\leq x \leq 4.\]
We refer to this as the \textbf{general solution} of the homogenous equation \(Lu = 0\) on \([a, b]\). \\

\textbf{Linear superposition} refers to this technique of constructing a new solution out of a linear combination of old ones. \\

\exmp{Example}{
    The general solution to \(u'' - u' - 2u = 0\) is \(u(x) = c_1 e^{-x} + c_2 e^{2x}.\)
}

\bigskip
Consider the inhomogeneous equation \(Lu = f\) on \([a, b]\), and fix a particular solution \(u_P\).

For \textit{any} solution \(u\), the difference \(u - u_P\) is a solution of the homogenous equation because
\[L(u - u_P) = Lu - Lu_P = f - f = 0 \text{ on } [a, b].\]
Hence, \(u(x) - u_P(x) = c_1u_1(x) + \dots _cmu_m(x)\) for some constants \(c_1, \dots, c_m\) and so
\[u(x) = u_P(x) + \underbrace{c_1u_1(x) + \cdots + c_mu_m(x)}_{u_H(x)}, \quad a \leq x \leq b,\]
is the \textbf{general solution} of the inhomogeneous equation \(Lu = f\).

\exmp{Example}{
The inhomogenous ODE \(u'' - u' - 2u = -2e^x\) has a particular solution \(u_P(x) = e^x\).

The general solution for its homogenous counterpart is \(u_H(x) = c_1e^{-x} + c_2e^{2x}\).

So the general solution of the inhomogeneous ODE is
\[u(x) = u_P(x) + u_H(x) = e^x + c_1 e^{-x} + c_2e^{2x}.\]
}

\thrm{Reduction of Order}{
    For \(u = u_1(x) \neq 0\), a solution to the ODE
    \[u'' + p(x)u' + q(x) u = 0,\]
    on some interval \(I\), then a second solution is
    \[u = u_1(x) \int \frac{1}{u_1^2 \exp(\int p \, dx)} \, dx.\]
}

\exmp{Example}{
    For the ODE \(u'' -6u' + 9u = 0\), take \(u_1 = e^{3x}\) and find \(v\). \textbf{Answer} \(xe^{3x}\).
}

\section{Differential Operators with Constant Coefficients}
If \(L\) has constant coefficients, then the problem of solving \(Lu = 0\) reduces to that of factorising the polynomial having the same coefficients. \\

Suppose that \(a_j\) is constant for \(0 \leq j \leq m\), with \(a_m \neq 0\). We define the associated polynomial of degree \(m\),
\[p(z) = \sum_{j=0}^{m} a_jz^j = a_mz^m + a_{m - 1} z^{m-1} + \cdots + a_1z + a_0,\]
so that if
\[Lu = a_mu^{(m)} + a_{m-1}u^{(m -1) + \cdots + a_1 u + a_0},\]
then formally, \(L = p(D)\). \\

By the fundamental theorem of algebra,
\[p(z) = a_m(z - \lambda_1)^{k_1}(z - \lambda_2)^{k_2} \cdots (z - \lambda)^{k_r}\]
where \(\lambda_1, \lambda_@, \dots, \lambda_r\) satisfying
\[k_1 + k_2 + \dots + k_r = m.\]

\prop{Lemma}{
\((D - \lambda)x^je^{\lambda x} = jx^{j-1}e^{\lambda x}\) for \(j \geq 0.\)
}

\prop{Lemma}{
    \((D - \lambda)^k x^j e^{\lambda x} = 0 \) for \(j = 0, 1, \dots, k - 1\).
}

\prop{Basic Solutions}{
If \((z - \lambda)^k\) is a factor of \(p(z)\) then the function \(u(x) = x^je^{\lambda x}\) is a solution of \(Lu = 0\) for \(0 \leq j \leq k - 1\).
}

\thrm{General Solution}{
    For the constant-coefficient case, the general solution of the homogenous equation \(Lu = 0\) is
    \[u(x) = \sum_{q=1}^{r} \sum_{l=0}^{k_q - 1} c_{ql}x^l e^{\lambda_q x},\]
    where the \(c_{ql}\) are arbitrary constants.
}

\exmp{Repeated Real Root}{
From the factorisation
\[D^4 + 6D^3 + 9D^2 - 4D - 12= (D - 1)(D + 2)^2(D + 3)\]
we see that the general solution of
\[u'''' + 6u''' + 9'' - 4u' - 12u = 0\]
is
\[u = c_1e^x + c_2e^{-2x} + c_3xe^{-2x} + c_4e^{-3x}.\]
}

\exmp{Complex Root}{
    From the factorisation
    \begin{align*}
        D^3 - 7D^2 + 17D - 15 & = (D^2 - 4D + 5)(D - 3)         \\
                              & = (D - 2 - i)(D - 2 + i)(D - 3)
    \end{align*}

    we see that the general solution of
    \[u''' - 7u'' + 17u' - 15u = 0\]
    is
    \begin{align*}
        u(x) & = c_1e^{(2 + i)x} + c_2e^{(2 -i)x} + c_3e^{3x}       \\
             & = c_4 e^{2x}\cos x + c_5 e^{2x} \sin x + c_3 e^{3x}.
    \end{align*}
}

\bigskip
Second-order ODEs arise naturally in classical mechanics for example a harmonic simple oscillator.

\section{Wronskians and Linear Independence}
We introduce a function, called the Wronskain that provides us with a way of testing whether a family of solutions to \(Lu = 0\) is linearly independent. \\

Let \(u_1(x), u_2(x), \dots, u_m(x)\) be functions defined on an interval \(I \in \bR\). The functions \(u_1, \dots, u_m\) are called \textbf{linearly dependent} if there exist constant \(a_1, a_2, \dots, a_m\) \textbf{not all zero} such that
\[a_1u_1(x) + a_2u_2(x) + \cdots + amu_m(x) = 0 \quad \forall x \in I.\]
If the above equation only holds for
\[a_i = 0, \quad i = 1, 2, \dots, m\]
then the functions are \textbf{linearly independent}.

\exmp{Example}{
    \(u_1 = \sin 2x\) and \(u_2 = \sin x \cos x\) are linearly dependent. \\
    \(u_1 = \sin x\) and \(u_2 = \cos x\) are linearly indepdent.
}

\bigskip The \textbf{Wronskian} of the functions \(u_1, u_2, \dots, u_m\) is the \(m \times m\) determinant
\[W(x) = W(x; u_1, u_2, \dots, u_m) = \det[D^{i-1}u_j].\]

\exmp{Example}{
    The Wronskian of the functions \(u_1 = e^{2x}, u_2 = xe^{2x}\) and \(u_3 = e^{-x}\) is
    \[
        W = \begin{vmatrix}
            e^{2x}  & xe^{2x}            & e^{-x}  \\
            2e^{2x} & e^{2x} + 2xe^{2x}  & -e^{-x} \\
            4e^{2x} & 4e^{2x} + 4xe^{2x} & e^{-x}
        \end{vmatrix}
        = 9e^{3x}.
    \]
}

\prop{Lemma}{
    If \(u_1, \dots, u_m\) are linearly dependent over an interval \([a, b]\) then \(W(x; u_1, \dots, u_m) = 0\) for \(a \leq x \leq b\).
}

\prop{Lemma}{
    If \(u_1, u_2, \dots, u_m\) are solutions of \(Lu = 0\) on the interval \([a, b]\) then their Wronskain satisfies
    \[a_m(x) W'(x) + a_{m-1}(x) W(x) = 0, \quad a \leq x \leq b.\]
}

\thrm{Linear Independence of Solutions}{
    Let \(u_1, u_2, \dots, u_m\) be solutions of a non-singular, linear, homogenous, \(m\)-th order ODE \(Lu = 0\) on the interval \([a, b]\). \\
    Either \\
    \indent \(W(x) = 0\) for \(a \leq x \leq b\) and the \(m\) solutions are linearly \textbf{dependent}, \\
    or else \\
    \indent \(W(x) \neq 0\) for \(a \leq x \leq b\) and the \(m\) solutions are linearly \textbf{independent}.
}

\section{Methods for Inhomogeneous Equations}
\subsection{Judicious Guessing Method}
You would have learned the mthod of undetermined coefficients for constructing a particular solution \(u_P\) to an inhomogeneous second-order linear ODE \(Lu = f\) in some simple cases. We will study this method systematically for higher-order linear ODEs with constant coefficients.

\exmp{Superposition of Solutions}{
    Suppose that \(u_1\) solves \(Lu = e^{3x}\), and \(u_2\) solves \(Lu = \sin x\), where \(L\) is a linear differential operator. Then the solution of
    \[Lu = e^{3x} + \sin x\]
    is
    \[u(x) = u_1(x) + u_2(x).\]
    And a solution of
    \[Lu = \frac{1}{2} e^{3x} - 5\sin x\]
    is
    \[u(x) = \frac{1}{2} u_1(x) - 5u_2(x).\]
}

\bigskip
Now we want to investigate some methods for finding particular solutions - i.e., finding a solution of \(Lu = f\). One such method is the method of judicious guessing. For example:

\begin{enumerate}
    \item If \(f\) is a polynomial, then guess that \(u_p\) is a polynomial.
    \item If \(f\) is a exponential, then guess that \(u_p\) is exponential.
    \item If \(f\) is a sine or cosine, then guess that \(u_p\) is a combination of such functions.
\end{enumerate}

One problem with this method: it will only work for the types of functions identified above.

\exmp{Example}{
    Suppose that \(u'' - u' = t^2 + 2t\). Note as before that,
    \[u_h(t) = c_1 + c_2e^t.\]
    So guess,
    \[u_p(t) = At^3 + Bt^2 + Ct + D.\]
    Then
    \[t^2 + 2t = u_p''-u' = -3At^2 + (6A - 2B)t + (2B - C).\]
    So, equating coefficients of like power terms, we see that
    \[A = -\frac{1}{3}, B = -2, C = -4, \text{ and } D \text{ is unrestricted.}\]
    Therefore, reabsorbing \(D\) into \(c_1\), we see that
    \[u(t) = u_h(t) + u_p(t) = c_1 + c_2e^t - \frac{1}{3}t^3 - 2t^2 - 4t.\]
}

\bigskip
Now we look at this idea of judicious guessing in a more systematic way. Let \(L = p(D)\) be a linear differential operator of order \(m\) with constant coefficients.

\thrm{Polynomial Solutions}{
    Assume that \(a_0 = p(0) \neq 0\). For any integer \(r \geq 0\), there exists a unique polynomial \(u_P\) of degree \(r\) such that \(Lu_P = x^r\).
}

\thrm{Exponential Solutions}{
    Let \(L = p(D), M \in \bR\) and \(\mu \in \bC\). If \(p(\mu) \neq 0\), then the function
    \[u_P(x) = \frac{Me^{\mu x}}{p(\mu)}\]
    satisfies \(Lu_P = Me^{\mu x}\).
}

\exmp{Example}{
    A particular solution of \(u'' + 4u' - 3i = 3e^{2x}\) is \(u_P = e^{2x} / 3\).
}

\thrm{Product of Polynomial and Exponential}{
    Let \(L = p(D)\) and assume that \(p(\mu) \neq 0\). For any integer \(r \geq 0\), there exists a unique polynomial \(v\) of degree \(r\) such that \(u_P = v(x)e^{\mu x}\) satisfies \(Lu_P = x^r e^{\mu x}\).
}

\subsection{Annihilator Method}
In the previous cases we proposed a solution \(u = u_P\) and showed that it satisfied \(Lu = f\). The following is a method to derive a particular solution given \(Lu = f\). If \(f(x)\) is differentiable at least \(n\) times and
\[[a_nD^n + a_{n-1}D^{n-1} + \cdots a_1 D^1 + a_0]f(x) = 0\]
then \([a_n D^n + a_{n-1}D^{n-1} + \cdots + a_1D^1 + a_0]\) \textbf{annihilates} \(f\).

\exmp{Example}{
\(D^n\) annhihilates \(x^{m - 1}\) for \(m \leq n\). \\
\((D - \alpha)^n\) annihilates \(x^{m-1}e^{\alpha x}\) for \(m \leq n\).
}

\exmp{Annhilator Method: Simple Example}{
    Given \(Lu = f\) we can apply the appropriate annhiliator to both sides and solving the resutling homogenous DE.

    Let \(Lu = u'' - u'\) and suppose we want a solution such that \(Lu = x^2\). Annihilating both sides we have
    \[D^3(u'' - u') = u^{(5)} - u^{(4)} = 0.\]
    Setting \(w = u^{(4)}\), clearly \(w = Ce^x\) is the general solution. Integrating four times yields
    \[u = Ce^x + Ex^3 + Fx^2 + Gx + H.\]
    Clearly \(u_h = Ae^x + H\) and the form of the particular solution is \(u_P = x(Ex^2 + Fx + G\). Substituting find \(E = -1/3, F= -1\) and \(G = -2\).
}

\subsection{Judicious Guessing Method Continued}
\thrm{Polynomial Solutions: The Reamining Case}{
    Let \(L = p(D)\) and assume \(p(0) = p'(0) = \cdots = p^{(k - 1)}(0) = 0\) but \(p^{(k)}(0) \neq 0\) where \(1 \leq k \leq m - 1\). For any integer \(r \geq 0\), there exists a unique polynomial \(v\) of degree \(r\)such that \(u_P(x) = x^Kv(x)\) satisfies \(Lu_P = x^r\).
}

\thrm{Exponential Times Polynomial: Remaining Case}{
Let \(L = p(D)\) and assume \(p(\mu) = p'(\mu) = \cdots = p^{(k - 1)}(\mu) = 0\). But \(p^{(k)}(\mu) \neq 0\), where \(1 \leq k \leq m - 1.\) For any integer \(r \geq 0\), there exists a unique polynomial \(v\) of degree \(r\) such that \(u_P(x) = x^kv(x)e^{\mu x}\) satisfies \(Lu_P = x^re^{\mu x}\).
}

\subsection{Variation of Parameters}
\exmp{Example}{
Find the general solution to \(u'' - 4u' + 4u = (x + 1)\exp 2x\).

Note first that the general solution, \(u_h\), to \(u'' - 4u' + 4u = 0\) is
\[u(x) = c_1 e^{2x} + c_2xe^{2x}\]
since the characteristic equation is \(0 = r^2 - 4r + 4 = (r - 2)^2\). Then
\[W(x) = \begin{vmatrix}
        e^{2x}  & xe^{2x}           \\
        2e^{2x} & e^{2x} + 2xe^{2x} \\
    \end{vmatrix}
    = e^{4x} + 2xe^{4x} - 2xe^{4x} = e^{4x}.
\]

So by the method of variation of parameters:
\[v_1'(x) = e^{-4x} \cdot -xe^{2x}(x + 1)e^{2x} \text{ and } v_2'(x) = e^{-4x} \cdot e^{2x}(x+1)e^{2x}.\]
In other words,
\[v_1'(x) = -x^2 - x \text{ and } v_2'(x) = x + 1.\]
Therefore \(u(x) = c_1e^{2x} + c_2xe^{2x} - (\frac{1}{3}x^3 + \frac{1}{2}x^2)e^{2x} + (\frac{1}{2}x^2 + x) xe^{2x}\).
}

\section{Solution via Power Series}

\paragraph{General Case}
Consider a general second-order, linear, homogenous ODE
\[Lu = a_2(x)u'' + a_1(x)u' + a_0(x)u = 0.\]
Equivalently,
\[u'' + p(x)u' + q(x)u = 0,\]
where
\[p(x) = \frac{a_1(x)}{a_2(x)} \text{ and } q(x) = \frac{a_0(x)}{a_2(x)}.\]
Assume that \(a_j\) is \textbf{analytic} at 0 for \(0 \leq j \leq 2\). Then \(p\) and \(q\) are analytic at 0, that is, they admit power series expansions
\[p(z) = \sum_{k=0}^{\infty} p_kz^k \text{ and } q(z) = \sum_{k=0}^{\infty}q_kz^k \text{ for } |z| < \rho,\]
for some \(\rho > 0\).

\thrm{Convergence Theorem}{
    If the coefficients \(p(z)\) and \(q(z)\) are analytic for \(|z| < \rho\), then the formal power series for the solution \(u(z)\), constructed above, is also analytic for \(|z| < \rho\).
}

\exmp{Power Series at Zero}{
Consider
\[Lu = (1-x^2)u'' - 5xu' - 4u = 0, \quad u(0) = 1, \quad u'(0) = 2.\]
In this case,
\[p(z) = \frac{-5z}{1-z^2} = -5\sum_{k=0}^{\infty}z^{2k + 1} \text{ and } q(z) = \frac{-4}{1 - z^2} = -4 \sum_{k=0}^{\infty}z^{2k}\]
are analytic for \(|z| < 1\), so the theorem guarantees that \(u(z)\), given by the formal power series, is also analytic for \(|z| < 1\).
}

\paragraph{Expansion about a Point other than Zero}
Suppose we want a power series expansion about a point \(c \neq 0\), for instance because the initial conditions are given at \(x = c\).

A simple change of the independent variable allows us to write
\[u = \sum_{k=0}^{\infty} A_k(z- c)^k = \sum_{k=0}^{\infty} A_kZ^k \text{ where } Z=z-c.\]

Since \(du / dx = du / dZ\) and \(d^2u/dz^2 = d^2u/dZ^2\), we obtain the translated equation
\[\frac{d^2u}{dZ^2} + p(Z + c) \frac{du}{dZ} + q(Z + c)u = 0.\]
Now compute that \(A_k\) using the series expansions of \(p(Z + c)\) and \(q(Z + c)\) in powers of \(Z\).

\section{Singular ODEs}
In general, we do not want \(L\) to be singular on an interval for which we wish to solve \(Lu = f\). However, some important applications lead to singular ODEs so we now address this case.

A second-order \textbf{Euler-Cauchy ODE} has the form
\[Lu = ax^2u'' + bxu' + cu = f(x),\]
where \(a, b\) and \(c\) are constants with \(a \neq 0\). This ODE is singular at \(x = 0\).

Noticing that
\[Lx^r = [ar(r - 1) + br + c]x^r,\]
we see that \(u = x^r\) is a solution of the homogenous equation \((f = 0)\) iff
\[ar(r - 1) + br + c = 0.\]

\paragraph{Factorisation}
Suppose \(ar(r - 1) + br + c = a(r - r_1)(r - r_2)\). If \(r_1 \neq r_2\) then the general solution of the homogenous equation \(Lu = 0\) is
\[u(x) = C_1x^{r_1} + C_2x^{r_2}, \quad x > 0.\]

\prop{Lemma}{
If \(r_1 = r_2\) then the general solution of the homogenous Euler-Cauchy equation \(Lu = 0\) is
\[u(x) = C_1x^{r_1} + C_2x^{r_1}\ln x, \quad x > 0.\]
}

\paragraph{Euler-Cauchy Equations with Nonreal Indicial Roots}
Suppose that \(r_{1,2} = \alpha \pm \beta i\) are the roots of the indicial equation
\[ar(r - 1) + br + c = 0\]
associated to the Euler-Cauchy equation
\[at^2u'' + btu' + cu = 0.\]
Then the real-valued solutions can be derived as follows. First note that
\[t^{\alpha + \beta i} = t^\alpha t^{\beta i}\]
is a solution. Then notice that
\[t^{\beta i} = e^{\ln t^{\beta i}} = e^{i \ln t^\beta} = \cos(\ln(t^\beta)) + i\sin(\ln(t^\beta)).\]
So,
\[t^\alpha t^{\beta i} = t^{\alpha} e^{\ln t^{\beta i}} = t^\alpha e^{i \ln t^\beta} = t^\alpha \left(\cos(\ln(t^\beta)) + i\sin(\ln(t^\beta))\right)\]
is a solution. Finally, since each of the real part an the imaginary part is .separately a (linear independent) solution, we see that the general solution in this case is (for \(t > 0\))
\[u(t) = t^{\alpha}\left(c_1 \cos(\ln(t^\beta)) + i\sin(\ln(t^\beta))\right).\]

\exmp{Example}{
    Consider \(t^2 u'' - tu' + 5u = 0\). Then the indicial equation is
    \[r(r - 1) - r + 5 = 0 \implies r = 1 \pm 2i.\]
    So the general solution is,
    \[u(t) = t(c_1 \cos \ln t^2 + c_2 \sin\ln t^2).\]
}

\bigskip
A number of important applications lead to ODEs that can be written in the \textbf{Frobenious normal form}
\[z^2u'' + zP(z)u' + Q(z)u = 0,\]
where \(P(z)\) and \(Q(z)\) are analytic at \(z = 0\):
\[P(z) = \sum_{k=0}^{\infty}P_k z^k \text{ and } Q(z) = \sum_{k=0}^{\infty}Q_k z^k, \quad |z| < \rho.\]

Now consider \(z^2u'' + zP(z) u' + Q(z)u = 0\). FOrmal manipulations show that \(Lu(z)\) equals
\[I(r)A_0 z^r + \sum_{k=1}^{\infty}\left(I(k + r) A_k + \sum_{j=0}^{k-1} \left[(j + r)P_{k - j} + Q_{k - j} \right]A_j \right)z^{k + r},\]
where \(I(r)\) is the indicial polynomial \(I(r) := r(r - 1)P_0 r + Q_0\), so we define \(A_0(r) = 1\) and
\[A_k(r) = \frac{-1}{I(k + r)} \sum_{j=0}^{k-1}[(j + r)P_{k-j} + Q_{k - j}]A_j(r), \quad k \geq 1,\]
provided \(I(k + r) \neq 0\) for all \(k \geq 1\).

\section{Bessel and Legendre Equations}
The \textbf{Beseel equation with parameter} \(\nu\) is
\[z^2u'' + zu' + (z^2 - \nu^2)u = 0.\]
This ODE is in Frobenius normal form, with indicial polynomial \(I(r) = (r + \nu)(r - \nu)\),
and we seek a series solution
\[u(z) = \sum_{k=0}^{\infty} A_k z^{k + r}.\]
We assume \(\Re \nu \geq 0\), so \(r_1 = \nu\) and \(r_2 = -\nu\).
\chapter{Initial-Boundary Value Problems in 1D}

We have seen that an initial-value problem for a (nonsingular) linear ODE \(Lu = f\) always has a unique solution. However, matters are not so simple for a \textbf{boundary-value problem}: a solution might not exist, or if one exists it might not be unique.

\section{Two-Point Boundary Value Problems}
In an \(m\)th order \textbf{initial-value problem} we specify \(m\) initial conditions at the left end of the interval. In an \(m\)th order \textbf{boundary-value problem}, we again specify \(m\) conditions involving the solution and its derivatives, but some apply at the left end and some at the right end.

\exmp{Boundary Conditions}{
    Consider the second-order ODE
    \[u'' + u' = 0 \quad \text{ for } 0 < x < \pi\]
    whose general solution is
    \[u(x) = A\cos x + B\sin x.\]

    A \textbf{unique solution} \(u(x) = \sin x\) exists satisfying
    \[u'(0) = 1 \text{ and } u(\pi) = 0.\]
    \textbf{No solution} exists satisfying
    \[u'(0) = 0 \text{ and } u(\pi) = 1.\]
    \textbf{Infinitely many solutions} \(u(x) = C\sin x\) exists satisfying
    \[u'(0) = 0 \text{ and } u(\pi) = 0.\]
}

\bigskip
We want to solve (Inhomogeneous BVP):
\[Lu = f \quad \text{ for } a < x < b, \quad \text{ with } B_1u = \alpha_1 \text{ and } B_2u = \alpha_2,\]
where
\[Lu = a_2u'' + a_1u' + a_0u\]
is a 2nd-order linear differential operator, and the \textbf{boundary operators} have the form
\begin{align*}
    B_1u & = b_{11}u'(a) + b_{10}u(a), \\
    B_2u & = b_{21}u'(b) + b_{20}u(b).
\end{align*}

\exmp{Linear Two-Point Bounary Value}{
    \begin{alignat*}{3}
        u'' -u  & = x - 1       & \quad & \text{ for } 0 < x < \log 2, \\
        u       & = 2           & \quad & \text{ at } x = 0,           \\
        u' - 2u & = 2\log 2 - 4 & \quad & \text{ at } x = \log 2.      \\
    \end{alignat*}
}

\section{Existence and Uniqueness}
Since \(L, B_1\) and \(B_2\) are all linear, the solutions of the \textbf{homogenous BVP}
\[Lu = 0 \quad \text{ for } a < x < b, \quad \text{ with } B_1u = 0 \text{ and } B_2u = 0,\]
form a vector space: if \(u_1\) and \(u_2\) are solutions of the inhomogeneous BVP then so is \(u = c_1u_! + c_2u_2\) for any constants \(c_1\) and \(c_2\).

\thrm{Uniqueness}{
    The inhomogenous BVP has \textbf{at most} one solution iff the homogenous BVP has only the \textbf{trivial solution} \(u \equiv 0\).
}

\thrm{Exactly One Solution}{
    If the homogenous problem has only the trivial solution, then for every choice of \(f, \alpha_1\) and \(\alpha_2\) the inhomogenous problem
    \[Lu = f \quad \text{ for } a < x < b, \quad \text{ with } B_1u = \alpha_1 \text{ and } B_2u = \alpha_2,\]
    has a unique solution.
}

\section{Inner Products and Norms of Functions}
If a homogenous initial boundary value problem admits non-trivial solutions, then the inhomogeneous problem might or might not have any solutions, depending on the forcing term and boundary values. \\

To formulate a condition that guarantees existence we require a short digression that introduces some ideas from functional analysis. \\

The \textbf{inner product} \(\langle f, g \rangle\) of a pair of continuous functions \(f, g : [a, b] \to \bR\) is defined by
\[\langle f, g \rangle = \int_a^b f(x)g(x) \, dx.\]

The corresponding \textbf{norm} of \(f\) is defined by
\[\norm*{f} = \sqrt{\langle f, f \rangle} = \left(\int_{a}^{b}[f(x)]^2\right)^{1/2}.\]

We say that \(f\) and \(g\) are \textbf{orthogonal} if \(\langle f, g \rangle = 0\).

\exmp{Inner Product and Norms}{
If
\[[a, b] = [-1, 1], \quad f(x) = x, \quad g(x) = \cos\pi x,\]
then
\[\inp{f}{g} = \int_{-1}^{1} x\cos\pi x \, dx = 0, \quad \norm*{f} = \sqrt{\frac{2}{3}}, \quad \norm{g} = 1.\]
Thus, \(f\)and \(g\) are orthogonal over the interval \([-1, 1]\).
}

\thrm{Cauchy-Schwarz Inequality}{
    \(|\langle f, g \rangle \leq \norm*{f}\norm*{g}\).
}

\prop{Triangle Inequality}{
    \(\norm*{f + g} \leq \norm*{f} + \norm*{g}\).
}

\section{Self-Adjoint Differential Operators}
Define the \textbf{formal adjoint} as
\begin{align*}
    L^*v & = (a_2v)'' - (a_1v)' + a_0v                        \\
         & = a_2v'' + (2a_2' - a_1)v' + (a_2'' - a_1' + a_0)v
\end{align*}

and the \textbf{bilinear concomitant}
\[P(u, v) = u'(a_2 v) - u(a_2v)' + u(a_1v),\]
we have the \textbf{Lagrange identity}
\[\inp{Lu}{v} = \inp{u}{L^*v} + [P(u, v)]_a^b.\]

\exmp{Adjoint Operators and Lagrange identity}{
    If
    \[Lu = 3xu'' - (\cos x)u' + e^xu\]
    then
    \begin{align*}
        L^*v & = (3xv)'' + [(\cos x)v]' + e^xv            \\
             & = 3xv'' + (6 + \cos x)v' + (e^x - \sin x)v
    \end{align*}
    and
    \begin{align*}
        P(u, v) & = u'(3xv) - u(3xv)' - uv\cos x     \\
                & = 3x(u'v - uv') - (3 + \cos x) uv.
    \end{align*}
    Then \((Lu)v = uL^*v + \frac{d}{dx}P(u,v)\).
}

\bigskip
The operator \(L\) is \textbf{formally self-adjoint} if \(L^* = L\).

\thrm{Formally Self-Adjoint Condition}{
    A second-order, linear differential operator L is formally self-adjoint iff it can be written in the form
    \[Lu = -(pu')' + qu = -pu'' - p'u' + qu,\]
    in which case the Lagrange identity takes the form
    \[(Lu)v - u(Lv) = -(p(x)(u'v - uv'))',\]
    or in other words, the bilinear concomitant it
    \[P(u,v) = -p(x)(u'v - uv').\]
}

\exmp{Bessel and Legendre are Self-Adjoint}{
Conside the Bessel equation
\[x^2u'' +xu' + (x^2 - \nu^2)u = f(x).\]
Dividing both sides by \(x\) gives \(Lu = -x^{-1}f(x)\) where
\[Lu = -(xu')' + (\nu^2x^{-1}-x)u.\]

The Legendre equation
\[(1-x^2)u'' - 2xu' + \nu(\nu + 1)u = f(x)\]
has the form \(Lu = -f(x)\) with \(Lu = -[(1-x^2)u']u' - \nu(\nu + 1)u\).
}

\paragraph{Transforming to Formally Self-Adjoint Form}
If we can evaluate the integrating factor
\[p(x) = \exp\left(\int \frac{a_1(x)}{a_2(x)} dx\right),\]
then we can transform an ODE of the form \(a_2u'' + a_1u' + a_0u = f(x)\) to formally self-adjoint form:
\begin{align*}
    -pu'' - \frac{pa_1}{a_2}u' - \frac{pa_0}{a_2}u & = \frac{-pf(x)}{a_2}, \\
    -(pu'' + p'u') - \frac{pa_0}{a_2}u             & = \frac{-pf(x)}{a_2}, \\
    -(pu')' + qu                                   & = \tilde{f}(x)
\end{align*}
where \(q = -pa_0/a_2\) and \(\tilde{f} = -pf/a_2\).

\exmp{Euler-Cauchy ODE}{
Write the Euler-Cauchy ODE \(ax^2u'' + bxu' + cu = f(x)\) in formally self-adjoint form. Note that here \(a_2(x) = ax^2, a_1(x) = bx\) and \(a_0(x) \equiv c\).

Define \(p\) by
\[p(x) = \exp\left(\int \frac{bx}{ax^2} dx\right) = \exp\left(\frac{b}{a} \int \frac{1}{x} dx\right) = e^{\frac{b}{a}\ln x} = x^{\frac{b}{a}}.\]

Then recalling that
\[q = -\frac{pa_0}{a_2} \text{ and } \tilde{f} = \frac{pf}{a_2},\]
the formally self-adjoint form is
\[-(x^{\frac{b}{a}}u')' - \frac{c}{ax^2}x^{\frac{b}{a}}u = -\frac{1}{ax^2}x^{\frac{b}{a}}f(x).\]
}

\prop{Self-Adjointness and Boundary Operators}{
    Any formally self-adjoint operator \(L = -(pu')' + qu\) satisfies the identity
    \[\inp{Lu}{v} - \inp{u}{Lv} = \sum_{i=1}^{2}(B_iuR_iv - R_iuB_iv),\]
    for all \(v\) and \(v\) where
    \[R_1u = \frac{p(a)u(a)}{b_{11}} \text{ or } R_1u = -\frac{p(a)u'(a)}{b_{10}}\]
    and
    \[R_2u = -\frac{p(b)u(b)}{b_{21}} \text{ or } R2_u = \frac{p(b)u'(b)}{b_{20}}.\]
}

\paragraph{Necessary Condition for Existence}
If \(u\) is a solution of the inhomogeneous BVP, and if \(v\) is a solution of the homogenous problem
\begin{alignat*}{3}
    Lv   & = 0 & \quad & \text{ for } a < x < b, \\
    B_1v & = 0 & \quad & \text{ at } x = a,      \\
    B_2v & = 0 & \quad & \text{ at } x = b.      \\
\end{alignat*}

then on the one hand
\[\inp{Lv}{v} - \inp{u}{Lv} = \inp{f}{v} - \inp{u}{0} = \inp{f}{v}\]

and on the other hand,
\[\inp{Lv}{v} - \inp{u}{Lv} = \underbrace{\alpha_1}_{=B_1u} R_1 v - R_1u \times \underbrace{0}_{=B_1v} + \underbrace{\alpha_2}_{=B_2u} R_2v - R_2u \times \underbrace{0}_{=B_2v}.\]

then the data \(f, \alpha_1\) and \(\alpha_2\) must satisfy
\[\inp{f}{v} = \alpha_1 R_1v + \alpha_2R_2v.\]

\thrm{Fredholm Alternative}{
    Either the homogenous BVP has only the trivial solution \(v \equiv 0\), in which case
    \begin{quote}
        the inhomogeneous BVP has a unique solution \(u\) for every choice of \(f, \alpha_1\) and \(\alpha_2\),
    \end{quote}
    OR else the homogenous BVP admits non-trivial solutions, in which case
    \begin{quote}
        the inhomogeneous BVP has a solution \(u\) iff \(f, \alpha_1\) and \(\alpha_2\) satify \(\inp{f}{v} = \alpha_1 R_1v + \alpha_2R_2v\) for every solution \(v\) of the homogenous BVP.
    \end{quote}
    In the latter case, \(u + Cv\) is also a solution of the inhomogeneous BVP for any constant \(C\).
}
\documentclass{article}
\usepackage[utf8]{inputenc}
\usepackage{geometry}
 \geometry{
 a4paper,
 total={170mm,257mm},
 left=20mm,
 top=20mm,
 }
\usepackage{enumerate}
\usepackage{enumitem}

\begin{document}

%%%%%%%%%%%%%%%%%%%%%%%%%%%%%%%%%%%%%%%%%%%%%%%%%%%%%%%%%%%%%%%%%%%%%%%%%%%%%%%%%%%%%%%%%%%%%%%%%%

\section{Correctness}
\textbf{Verification}
\begin{itemize}
    \renewcommand\labelitemi{--}
    \item Verification in a system life cycle context is a set of activities that compares a product of the system life cycle against the required characteristics for that product.
    \item Checking if the system has been built correctly.
    \item \textbf{Static} $=>$ Type Safety, linting.
    \item \textbf{Dynamic} $=>$ Preformed during the execution of software, testing if your program works as intended.
    \begin{itemize}
        \item \textbf{Small tests} $->$ Unit tests, testing of individual software components.
        \item \textbf{Larger tests} $->$ Performed to expose defects in the interfaces and in the interactions between integrated components or systems.
    \end{itemize}
\end{itemize}
\textbf{Exceptions}
    \begin{itemize}
        \renewcommand\labelitemi{--}
        \item An exception is an action that disrupts the normal flow of a program.
        \item This action is often representative of an error being thrown.
        \item  Exceptions are ways that we can elegantly recover from errors.
    \end{itemize}
\textbf{Coverage}
    \begin{itemize}
        \renewcommand\labelitemi{--}
        \item \textbf{Test Coverage}: a measure of how much of the feature set is covered with tests. 
        \item \textbf{Code Coverage}: a measure of how much code is executed during testing.
    \end{itemize}

\section{Agile}
    \begin{itemize}
        \renewcommand\labelitemi{--}
        \item Philosophy and culture that are used to inform a range of different processes. 
        \item \textbf{Waterfall method} - Requirements, Design, Implementation, Testing, Deployment \& Maintenance    
    \end{itemize}

\section{Full Stack}
    \begin{itemize}
        \renewcommand\labelitemi{--}
        \item \textbf{Standard Interfaces} - A universal method of connecting different systems together. 
        \item \textbf{JSON} - A format made up of braces for objects, square brackets for arrays, where all non-numeric items must be wrapped in quotations. Similar to JS structures.
        \item \textbf{YAML} - Ease of editing and concise, indentation matters, dash used to begin a list item.
        \item \textbf{XML} - More verbose, demanding to process, more bytes to store.
        \item \textbf{Authentication}: Process of verifying the identity of a user.
        \item \textbf{Authorisation}: Process of determining an authenticated user's access privileges.
    \end{itemize}
    
\section{Continuous $\int$}
    \begin{itemize}
        \renewcommand\labelitemi{--}
        \item Practice of automating the integration of code changes from multiple contributors into a single software project.
        \item Planning $->$ Analysis $->$ Design $->$ Implementation $->$ Testing and $\int$ $->$ Maintenance
        \item Coding $->$ Build $->$ Test $->$ Report $->$ Merge $->$ Release
        \item \textbf{Continuous Delivery} - Allows accepted code changes to be deployed to customers quickly and sustainably. (By pressing button)
        \item \textbf{Continuous Deployment} - Allows changes to be deployed automatically as long as all tests pass.
    \end{itemize}

\section{Design}
\textbf{Maintainability}
    \begin{itemize}
        \renewcommand\labelitemi{--}
        \item Software in the real world changes over time. The less easily maintainable software is,
the harder it is to adapt to these changes.
        \item Maintainable software resists the tendency to break as software changes or grows.
        \item To improve software maintainability, testing, system design and code design.
        \item One source of truth, DRY, KISS, Refactoring code etc...
    \end{itemize}
\textbf{Modelling}
    \begin{itemize}
        \renewcommand\labelitemi{--}
        \item Simplified representation to assist in understanding something more complex.
        \item Conceptual model captures a system
        \item State Diagrams
    \end{itemize}
\textbf{Complexity}
    \begin{itemize}
        \renewcommand\labelitemi{--}
        \item Essential / Accidental
        \item \textbf{Coupling} - Measure of how closely connected different software components are. Loose coupling is good.
        \item \textbf{Cohesion} - The degree to which elements of a module belong together. High cohesion is good.
        \item \textbf{Cyclomatic Complexity} Convert functions into a control flow graph, use formula $V(G) = e - n + 2$
    \end{itemize}

\section{Requirements}
    \begin{itemize}
        \renewcommand\labelitemi{--}
        \item A condition or capability needed by a user to solve a problem or achieve an objective
        \item Good requirements are the following: clear, concise, atomic, verifiable, attainable, abstract 
        \item \textbf{Functional} specify a specific capability/service that the system should provide. It's what the system does.
        \item \textbf{Non-functional} place a constraint on how the system can achieve that. Performance characteristic.
        \item Elicitation, Analysis, Specification, Validation
    \end{itemize}
\textbf{User stuff}
    \begin{itemize}
        \renewcommand\labelitemi{--}
        \item User Stories
        \item User Acceptance Criteria
        \item Use Cases (list or diagram)
    \end{itemize}
\textbf{Validation}
    \begin{itemize}
        \renewcommand\labelitemi{--}
        \item Validation in a system life cycle context is a set of activities ensuring and gaining confidence that a system is able to accomplish its intended use, goals and objectives.
        \item Right system has been built.
        \item Done by User Acceptance Testing, formal testing with respect to user needs. 
    \end{itemize}

\end{document}

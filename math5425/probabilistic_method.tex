\chapter{The Probabilistic Method}

This chapter assumes knowledge of elementary probability knowledge. Content from first year is sufficient.

\begin{example}
    Let \(\Omega\) be the set of all graphs on the vertex set \(\{1, 2, \dots, n\}\). Then \(|\Omega| = 2^{\binom{n}{2}}\). Define \(\pi(G) = 2^{\binom{n}{2}}\) for all \(G \in \Omega\). This is the \textit{uniform model of random graphs}.
\end{example}

\begin{lemma}
    The expected number of edges in a uniformly chosen graph on the vertex set \(\{1, 2, \dots n\}\) is \(\frac{1}{2}\binom{n}{2}\).
    \proof{
        (From Definition) For \(0 \leq m \leq \binom{n}{2} = N\), there \(\binom{N}{m}\) are exactly of graphs on vertex set \(\{1, \dots, n\}\) with \(m\) edges. Let \(X\) be the number of edges in the random graph. Then
        \begin{align*}
            EX & = \sum_{m=0}^{N} \Pr(X = m) \cdot m                              \\
               & = \sum_{m=0}^{N} \frac{\binom{N}{m}}{2^N} \cdot m                \\
               & = \frac{N}{2^N} \sum_{m=1}^{N} \frac{(N-1)!}{(m-1)!(N-m)!}       \\
               & = \frac{N}{2^N} \sum_{j=0}^{N-1} \binom{N - 1}{j} \tag{j = m -1} \\
               & = \frac{N}{2^N} 2^{N-1} \tag{by the binomal theorem}             \\
               & = \frac{N}{2} = \frac{1}{2}\binom{n}{2}.
        \end{align*}
    }
\end{lemma}

Let \(A \subseteq \Omega\) be an event. The indicator variable \(I_A\) for \(A \subseteq \Omega\) is
\[I_A(z) = \begin{cases}
        1 & \text{ if } z \in A \\
        0 & \text{ otherwise.}
    \end{cases}\]

\begin{definition}[Linearity of Expectation]
    Let \(X_1, \dots, X_k\) be random variables on \(\Omega\) and let \(c_1, \dots, c_k \in \bR\). Define the random variable \(X = c_1X_1 + \cdots + c_k X_k\). Then
    \[\bE[X] = c_1\bE[X_1] + c_2\bE[X_2] + \cdot + c_k\bE[X_k].\]
\end{definition}

\begin{definition}[Markov's Inequality]
    Suppose that \(X: \Omega \to [0, \infty)\) is a nonnegative random variable on \(\Omega\) and let \(k > 0\). Then
    \[\Pr(X \geq k) \leq \frac{\bE[X]}{k}.\]
    In particular, if \(X\) is a nonnegative integer-valued random variable then
    \[\Pr(X \neq 0) \leq \bE[X].\]
\end{definition}

Let \(k \geq 2\) be an integer. Events \(A_1, \dots, A_k\) in \(\Omega\) are \textbf{mutually independent} if for all \(j, \ell_1, \dots, \ell_j\) with \(2 \leq j \leq k\) and \(1 \leq \ell_1 < \ell_2 < \cdots < \ell_j \leq k\),
\[\Pr\left(\bigcap_{i=1}^j A_{\ell_i}\right) = \prod_{i = 1}^j \Pr(A_{\ell_i}).\]

\begin{lemma}
    Let \(\Omega\) be the set of all subsets of some given set \(S\), where \(|S| = n\). Define a random set \(X \subseteq S\) by setting \(\Pr(x \in X) = \frac{1}{2}\), independently for each \(x \in S\). Then \(\Pr(X = A) = 2^{-n}\) for all \(A \subseteq S\), so this gives the uniform probability space on \(\Omega\).
    \proof{
        Fix \(A \subseteq \Omega\). Then
        \begin{align*}
            \Pr(X = A) & = \prod_{x \in A} \Pr(\text{heads}) \cdot \prod_{x \notin A}\Pr(\text{tails}) \tag{using independence} \\
                       & = \left(\frac{1}{2}\right)^{|A|} \cdot \left(\frac{1}{2}\right)^{n - |A|}                              \\
                       & = \left(\frac{1}{2}\right)^n = 2^{-n}
        \end{align*}
        as claimed.
    }
\end{lemma}

\begin{theorem}[Alon \& Spencer, Theorem 2.2.1]
    Let \(G\) be a graph with \(n\) vertices and \(m\) edges. Then \(G\) contains a bipartite subgraph with at least \(m / 2\) edges.
    \proof{
        Let \(\Omega\) be the set of all subsets of \(V(G)\). Then \(|\Omega| = 2^n\). Consider the uniform probability space on \(\Omega\). Let \(A \subseteq V\) be a randomly chosen element of \(\Omega\) and define \(B = V - A\). Call \(xy \in E(G)\) a crossing edge if exactly one of \(x, y\) belongs to \(A\). Let \(X\) be the number of crossing edges. Finally, for each edge \(e \in E(G)\) define the indicator variable
        \[X_e = \begin{cases}
                1 & \text{ if \(e\) is a crossing edge}, \\
                0 & \text{ otherwise}.
            \end{cases}\]
        Then \(X = \sum_{e \in E(G)} X_e\). For any \(e = xy \in E(G)\), we have,
        \begin{align*}
            \Pr(x \in A \text{ and } y \notin A) & = \Pr(x \in A)\Pr(y \in A) \tag{using independence} \\
                                                 & = \frac{1}{2} \times \frac{1}{2} = \frac{1}{4}.
        \end{align*}
        Therefore
        \begin{align*}
            \bE X_e & = \Pr((x \in A \tand y \notin A) \text{ or } (x \notin A \tand y \in A))                  \\
                    & = \Pr(x \in A \tand y \notin A) + \Pr(x \notin A \tand y \in A) \tag{events are disjoint} \\
                    & = \frac{1}{4} + \frac{1}{4} = \frac{1}{2}.
        \end{align*}
        Hence, by linearity of expectation,
        \[\bE X = \sum_{e\in E(G)} \bE X_e = \frac{m}{2}.\]
        Thus there exists a fixed set \(A_0 \subseteq V(G)\) which has at least \(\frac{m}{2}\) crossing edges. The corresponding bipartition \((A_0, V(G) - A_0)\) defines a bipartite subgraph consisting of the \(\geq \frac{m}{2}\) crossing edges.
    }
\end{theorem}

An \textbf{independent set} in a graph \(G\) is a subset \(U \subseteq V\) such that if \(v, w \in U\) then \(vw \in E(G)\). Let \(\alpha(G)\) be the size of a maximum independent set in \(G\), called the \textbf{independence number}.

\begin{theorem}
    Let \(G\) have \(n\) vertices and \(nd / 2\) edges, where \(d \geq 1\). Then \(\alpha(G) \geq \frac{n}{2d}\). Note \(d\), is the average degree of \(G\).
    \proof{
        Define the random subset \(S \subseteq V(G)\) by \(\Pr(v \in S) = p\), independently for all \(v \in V\). Here \(p \in [0, 1]\) which we will fix later. \\

        Let \(X = |S|\) and let \(Y\) be the number of edges of \(G\) with both endvertices in \(S\). Then \(\bE X = pn\). For \(e \in E(G)\) let \(Y_e\) be the indicator variable for the event \(e \subseteq S\). Then for every \(e = xy \in E(G)\),
        \begin{align*}
            \bE Y_e & = \Pr(x \in S \tand y \in S)                            \\
                    & = \Pr(x \in S) \cdot \Pr(y \in S) \tag{by independence} \\
                    & = p^2.
        \end{align*}
        Therefore, by linearity of expectation and the fact that \(Y = \sum_{e \in E(G)}\) we have
        \[\bE Y = \sum_{e \in E(G)} \bE Y_e = \frac{nd}{2} p^2.\]
        By linearity of expectation,
        \[\bE(X - Y) = \bE X - \bE Y = pn - p^2 \frac{nd}{2}.\]
        Want to choose \(p\) to maximise this, so \(p = \frac{1}{d}\) and \(p \in [0, 1]\). Substituting gives \(\bE (X - Y) = \frac{n}{2d}\). Hence there exists a fixed set \(S_0 \subseteq V(G)\) with \(|S_0| - (\text{\# edges in} S_0) \geq \frac{n}{2d}\). Delete one vertex from each edge within \(S_0\) to give a set \(S^*\) of at least \(\frac{n}{2d}\) vertices which is an independent set.
    }
\end{theorem}
\chapter{Graph Colourings}

A \textbf{vertex colouring} of a graph \(G = (V, E)\) is a function \(c: V \to S\) such that \(c(u) \neq c(v)\) whenever \(uv \in E\). Here \(S\) is the set of available colours, usually \(S = \{1, 2, \dots, k\}\) for some positive integer \(k\). \\

A \textbf{\(k\)-colouring} of \(G\) is a colouring \(c: V \to \{1, 2, \dots, k\}\). Often we want the smallest value of \(k\) for which a \(k\)-colouring of \(G\) exists. This smallest value of \(k\) is called the \textbf{chromatic number} of \(G\), denoted \(\chi(G)\). \\

If \(\chi(G) = k\) then \(G\) is said to be \(k\)-chromatic.

If \(\chi(G) \leq k\) then \(G\) is said to be \(k\)-colourable. \\

The set of all vertices in \(G\) with a given colour under \(c\) is called a \textbf{colour class}. Each colour class is an independent set. \(k\)-colouring is a partition of \(V(G)\) into \(k\) independent sets. \\

A \textbf{clique} in a graph \(G\) is a complete subgraph of \(G\). The order of the largest clique in \(G\) is called the \textbf{clique number} of \(G\), denoted \(\omega(G)\). \\

Fact: \(\chi(G) \geq \omega(G)\) and \(\chi(G) \geq n/\alpha(G)\). \\

An \textbf{edge colouring} of \(G\) is a map \(c : E \to S\) such that \(c(e) \neq c(f)\) whenever \(e\) and \(f\) share an endvertex. If \(S = \{1, 2, \dots, k\}\) then \(c\) is a \textbf{\(k\)-edge-colouring} and \(G\) is \(k\)-edge-colourable. \\

Let \(\chi'(G)\) be the smallest positive integer \(k\) for which \(G\) is \(k\)-edge-colourable. We call \(\chi'(G)\) the \textbf{chromatic index} of \(G\). \\

A \textbf{colour class} in an edge colouring is a matching of \(G\). Hence an edge colouring displays \(E(G)\) as a union of disjoint matchings. \\

The \textbf{line graph}, denoted \(L(G)\), has vertex set \(E(G)\) and \(e, f \in E(G)\) form an edge of \(L(G)\) if and only if \(e, f\) share an endvertex in \(G\). Every edge-colouring of \(G\) is a vertex colour of \(L(G)\) and vice-versa. So \(\chi'(G) = \chi(L(G))\).

\section{Vertex Colourings}

\begin{proposition}
    If graph \(G\) has \(m\) edges then \(\chi(G) \leq \frac{1}{2} + \sqrt{2m + \frac{1}{4}}\).
    \proof{
        Fix a \(k\)-colouring of \(G\) with \(k = \chi(G)\) colours. Then \(G\)has at least one edge between any two distinct colour classes, or we could merge them to give a colouring of \(G\) with \(\leq k - 1\) colours. Hence \(m \geq \binom{k}{2} = \frac{1}{2}(k)(k - 1)\) then solve for \(k\) to complete the proof.
    }
\end{proposition}

\paragraph{Greedy Algorithm} Given a graph \(G\), fix an ordering \(v_1, v_2, \dots, v_n\) on the vertices of \(G\) and colour them one by one in this order using the first available colour (least positive integer) as you go along. Since \(v_i\) has at most \(\Delta(G)\) neighbours, this produces a \(k\)-colouring of \(G\) with \(k \leq \Delta(G) + 1 \implies \chi(G) \leq \Delta(G) + 1\). \\

Fact: \(\chi(G) = \Delta(G) + 1\) if \(G\) is a complete graph or an odd cycle.

\begin{theorem}[Brooks, 1941] \label{brooks-thrm}
    Let \(G\) be a connected graph. If \(G\) is neither complete nor a \(n\) odd cycle then \(\chi(G) \leq \Delta(G)\).
    \proof{
    We give a proof by Mariusz Zajac (2018). First an observation. Let \(G\) be a graph with maximum degree \(\Delta(G) \leq k\), where \(\{1, \dots, k\}\) will be our set of colours. Suppose that \(G\) is partially coloured. Let \(P = v_1v_2 \dots v_j\) be a path in \(G\) such that all vertices of \(P\) are uncoloured. Then we can colour vertices \(v_1, v_2, \dots, v_{j - 1}\) in this order, since at the moment that we colour \(v_i (1 \leq i \leq j - 1)\), we know that \(v_i\) has an uncoloured neighbour \(v_{i + 1}\), and hence at most \(\Delta - 1\) neighbours. Call this procedure \texttt{PATHCOLOUR}\((v_1, v_2, \dots, v_{j-1};v_j)\). Note that this procedure colours \(v_1, \dots, v_{j - 1}\) but it leaves \(v_j\) uncoloured. In particular, if \(j = 1\) then \texttt{PATHCOLOUR}\((v_1)\) leaves the graph unchanged. \\

    {\bf Theorem. } (Restatement of Brooks Theorem) Let \(k \geq 3\) be an integer and let \(G\) be a graph with \(\Delta(G) \leq k\). If \(G\) does not contain \(K_{k + 1}\) as a subgraph then \(G\) is \(k\)-colourable. \\

    Before proving this new version, we show that it implies Theorem \ref{brooks-thrm}. Assume that ``new version'' is true. Suppose that \(G\) is a connected graph and take \(\Delta(G) = k \geq 3\) colours. Hence \(G\) is not an odd cycle as \(\Delta(G) \geq 3\). Also suppose that \(G\) is not complete, so \(G \neq K_{\Delta(G) + 1}\). But since \(G\) has maximum degree \(\Delta(G)\), \(G\) contains \(K_{\Delta(G) + 1}\) as a subgraph if and only if \(G = K_{\Delta(G) + 1}\). Therefore \(G\) satisfies the conditions of ``new version'' with \(k = \Delta(G)\). Applying this result, we conclude that \(G\) is \(\Delta(G)\)-colourable. Therefore \(\chi(G) \leq \Delta(G)\). Hence we obtain Brooks Theorem as a corollary to ``new version''.
    }
\end{theorem}

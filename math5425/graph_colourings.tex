\chapter{Graph Colourings}

A \textbf{vertex colouring} of a graph \(G = (V, E)\) is a function \(c: V \to S\) such that \(c(u) \neq c(v)\) whenever \(uv \in E\). Here \(S\) is the set of available colours, usually \(S = \{1, 2, \dots, k\}\) for some positive integer \(k\). \\

A \textbf{\(k\)-colouring} of \(G\) is a colouring \(c: V \to \{1, 2, \dots, k\}\). Often we want the smallest value of \(k\) for which a \(k\)-colouring of \(G\) exists. This smallest value of \(k\) is called the \textbf{chromatic number} of \(G\), denoted \(\chi(G)\). \\

If \(\chi(G) = k\) then \(G\) is said to be \(k\)-chromatic.

If \(\chi(G) \leq k\) then \(G\) is said to be \(k\)-colourable. \\

The set of all vertices in \(G\) with a given colour under \(c\) is called a \textbf{colour class}. Each colour class is an independent set. \(k\)-colouring is a partition of \(V(G)\) into \(k\) independent sets. \\

A \textbf{clique} in a graph \(G\) is a complete subgraph of \(G\). The order of the largest clique in \(G\) is called the \textbf{clique number} of \(G\), denoted \(\omega(G)\). \\

Fact: \(\chi(G) \geq \omega(G)\) and \(\chi(G) \geq n/\alpha(G)\). \\

An \textbf{edge colouring} of \(G\) is a map \(c : E \to S\) such that \(c(e) \neq c(f)\) whenever \(e\) and \(f\) share an endvertex. If \(S = \{1, 2, \dots, k\}\) then \(c\) is a \textbf{\(k\)-edge-colouring} and \(G\) is \(k\)-edge-colourable. \\

Let \(\chi'(G)\) be the smallest positive integer \(k\) for which \(G\) is \(k\)-edge-colourable. We call \(\chi'(G)\) the \textbf{chromatic index} of \(G\). \\

A \textbf{colour class} in an edge colouring is a matching of \(G\). Hence an edge colouring displays \(E(G)\) as a union of disjoint matchings. \\

The \textbf{line graph}, denoted \(L(G)\), has vertex set \(E(G)\) and \(e, f \in E(G)\) form an edge of \(L(G)\) if and only if \(e, f\) share an endvertex in \(G\). Every edge-colouring of \(G\) is a vertex colour of \(L(G)\) and vice-versa. So \(\chi'(G) = \chi(L(G))\).

\section{Vertex Colourings}

\begin{proposition}
    If graph \(G\) has \(m\) edges then \(\chi(G) \leq \frac{1}{2} + \sqrt{2m + \frac{1}{4}}\).
    \proof{
        Fix a \(k\)-colouring of \(G\) with \(k = \chi(G)\) colours. Then \(G\)has at least one edge between any two distinct colour classes, or we could merge them to give a colouring of \(G\) with \(\leq k - 1\) colours. Hence \(m \geq \binom{k}{2} = \frac{1}{2}(k)(k - 1)\) then solve for \(k\) to complete the proof.
    }
\end{proposition}

\paragraph{Greedy Algorithm} Given a graph \(G\), fix an ordering \(v_1, v_2, \dots, v_n\) on the vertices of \(G\) and colour them one by one in this order using the first available colour (least positive integer) as you go along. Since \(v_i\) has at most \(\Delta(G)\) neighbours, this produces a \(k\)-colouring of \(G\) with \(k \leq \Delta(G) + 1 \implies \chi(G) \leq \Delta(G) + 1\). \\

Fact: \(\chi(G) = \Delta(G) + 1\) if \(G\) is a complete graph or an odd cycle.

\begin{theorem}[Brooks, 1941] \label{brooks-thrm}
    Let \(G\) be a connected graph. If \(G\) is neither complete nor an odd cycle then \(\chi(G) \leq \Delta(G)\). In fact we will prove the following restatement of Brooks Theorem, due to Zajac (2018):
    \begin{mdframed}
        \begin{center}
            Let \(k \geq 3\) be an integer and let \(G\) be a graph with \(\Delta(G) \leq k\). If \(G\) does not contain \(K_{k + 1}\) as a subgraph then \(G\) is \(k\)-colourable.
        \end{center}
    \end{mdframed}

    We call this the ``new'' version of Brooks Theorem and prove that this implies Brooks Theorem.
    \proof{
        Suppose that \(G\) is a graph which satisfies the assumptions of Brooks Theorem. That is, \(G\) be a connected graph which is not an odd cycle and which is not complete. Let \(\Delta = \Delta(G)\) be the maximum degree of \(G\). We want to show that \(\chi(G) \leq \Delta\), as this is the conclusion required for Brooks Theorem. \\

        First suppose that \(\Delta \leq 2\). Then \(G\) is either a path or an even cycle, as \(G\) is connected. Hence \(G\) is bipartite and so \(\chi(G) \leq 2 = \Delta\), as required. \\

        Now suppose that \(\Delta \geq 3\). We wish to apply the new version of Brooks Theorem with \(k = \Delta\), so we must check that \(G\) does not contain \(K_{\Delta + 1}\) as a subgraph. For a contradiction, suppose that \(G\) does have a subgraph \(H\) which is isomorphic to \(K_{\Delta + 1}\). Then \(H\) is \(\Delta\)-regular, and \(G\) has maximum degree \(\Delta\), so there is no edge from a vertex of \(H\) to a vertex of \(G - V(H)\). It follows that \(H\) is a component of \(G\). But \(G\) is connected, so the only possibility is that \(G = H\). BUt this contradicts our assumption that \(G\) is not complete. \\

        Therefore, \(G\) satisfies the assumptions of the new version of Brooks Theorem, and by applying this result we find that \(G\) is \(\Delta\)-colourable. From this we conclude that \(\chi(G) \leq \Delta\), as required. \\

        In both cases, the conclusion of Brooks Theorem holds, completing the proof.
    }
    We now prove that this ``new'' version is true.
    \proof{
        First an obversation, let \(G\) be a graph with maximum degree \(\Delta(G) \leq k\), where \(\{1, \dots, k\}\) will be our set of colours. Suppose that \(G\) is partially coloured. Let \(P = v_1 v_2\dots v_j\) be a path in \(G\) such that all vertices of \(P\) are uncoloured. Then we can colour vertices \(v_1, v_2, \dots, v_{j - 1}\) in this order, since at the moment that we colour \(v_i (1 \leq i \leq j - 1)\), we know that \(v_i\)has an uncoloured neighbour \(v_{i+1}\) and hence aat most \(\Delta - 1\) neighbours. Call this procedure \texttt{PATHCOLOUR}\((v_1, \dots, v_{j-1};v_j)\). Note that this procedure colours \(v_1, \dots, v_{j-1}\) but it leaves \(v_j\) uncoloured. In particular if \(j = 1\) then \texttt{PATHCOLOUR}\((v_1)\) leaves the graph unchanged. \\

        Proof by induction on \(n = |G|\), where \(G\) is a graph with \(\Delta(G) \leq k\) and \(k \geq 3\). If \(n \leq k\) then we can \(k\)-colour \(G\) by giving each vertex a distinct colour.
        \begin{quote}
            {\bf Claim.} If \(G\) has a vertex of degree \(< k\) then \(G\) is \(k\)-colourable.

            {\bf Proof.} Let \(v\) be a vertex of degree \(< k\) and let \(G' = G - v\). By the inductive hypothesis we can \(k\)-colour \(G'\). Fix one such colouring \(C\). Then at most \(k - 1\) colours are used by \(C\) on neighbours of \(v\), so we have an available colour which we can use to colour \(v\).
        \end{quote}
        Now we assume that \(G\) is \(k\)-regular. Let \(v\) be a vertex of \(G\) and consider \(G[\{v\} \cup N(v)]\). Since \(G\) has no subgraph isomorphic to \(K_{k + 1}\), we know that \(v\) has two neighbours \(x, y\) which are not adjacent. Let \(v_1 = x, v_2 = v, v_3 = y\), and extend the path \(v_1 v_2 v_3\) to a maximal length path in \(G, P = v_1v_2v_3 \dots v_r\) which starts with \(v_1v_2v_3\).
        \begin{quote}
            {\bf Case 1.} Suppose that \(r = n\). This means that all vertices of \(G\) lie on \(P\) (Hamilton Path). Let \(v_j\) be any neighbour of \(v_2\) other than \(v_2\) and \(v_3\). Since \(G\) is \(k\)-regular and \(k \geq 3\) we can choose such a vertex \(v_j\). We now describe how to \(k\)-colour \(G\).
            \begin{itemize}
                \item First colour \(v_1\) and \(v_3\) the same colour.
                \item Next apply \(\texttt{PATHCOLOUR}(v_4, v_5, \dots, v_{j - 1};v_j)\) which colours \(v_4, \dots, v_{j-1}\) and leaves \(v_j\) uncoloured.
                \item Next apply \(\texttt{PATHCOLOUR}(v_n, v_{n - 1}, \dots, v_j ; v_2)\) which will colour all remaining vertices of \(G\) except \(v_2\).
                \item Finally we have an available colour for \(v_2\) since two of its neighbour (\(v_1 \tand v_3\)) have the same colour. Colour \(v_2\) with an available colour.
            \end{itemize}

            {\bf Case 2.} Now suppose that \(r < n\). Recall that all neighbours of \(v_r\) lie on \(P\). Let \(v_j\) be the neighbour of \(v_r\) with the smallest index. Then \(C = v_jv_{j + 1} \dots v_rv_j\) is a cycle in \(G\). Let \(G' = G - V(C)\). We can \(k\)-colour \(G'\) by induction. If there is no edge between \(G'\) and \(C\) then we can also \(k\)-colour \(G[V(C)]\), by induction and we are done. Otherwise (\(G[V(C)]\) is not a component of \(G\)): let \(v_\ell\) be the vertex on \(C\) with largest index which has a neighbour in \(G'\), and let \(u\) be a neighbour of \(v_\ell\) in \(G'\). Note, \(v_\ell\) is well defined as \(v_j\) has a neighbour in \(G'\) if \(j \geq 2\). Note \(\ell \leq r - 1\) since all neighbours of \(v_r\)belong to \(V(C)\). Also \(v_{\ell + 1}\) has no neighbours outside \(C\), by choice of \(v_\ell\). We now describe how to \(k\)-colour vertices of \(C\), giving a \(k\)-colouring of \(G\).
            \begin{itemize}
                \item First, colour \(v_{\ell + 1}\) with the colour assigned to \(u\).
                \item Next, apply \(\texttt{PATHCOLOUR}(v_{\ell + 2}, \dots, v_r, v_j, v_{j + 1}, \dots, v_{\ell - 1}; v_\ell)\) which colours all remaining vertices of \(G\) except \(v_\ell\).
                \item Finally, colour \(v_\ell\) with an available colour which exists because \(v_\ell\) has two neighbours with the same colour.
            \end{itemize}
        \end{quote}
        This completes thr proof in Case 2, by mathematical induction.
    }
\end{theorem}

\section{Edge Colourings}
By considering a vertex of maximum degree, we see that the chromatic index \(\chi'(G)\) satisfies \(\chi'(G) \geq \Delta(G)\)for all graphs \(G\).

\begin{proposition}[K\"onig, 1916]
    If \(G\) is bipartite then \(\chi'(G) = \Delta(G)\).
    \proof{
        We prove t his by induction on \(m = |E(G)|\). If \(m = 0\) then the result is trivially true. So, assume that \(m \geq 1\) and that the result holds for all bipartite graphs with at most \(m - 1\) edges. \\

        Let \(\Delta = \Delta(G)\), choose \(xy \in E\) and let \(G' = G - xy\). By induction, we can fix a \(\Delta\)-edge-colouring of \(G'\). We call edges coloured \(\alpha\), ``\(\alpha\)-edges'', etc. In \(G'\), vertices \(x, y\) both have degree \(\Delta - 1\). So there are colours \(\alpha, \beta \in \{1, 2, \dots, \Delta\}\) such that \(x\) is not incident with an \(\alpha\)-edge, and \(y\) is not incident with a \(\beta\)-edge. \\

        If \(\alpha = \beta\) then we can colour the edge \(xy\) with colour \(\alpha\) to give a \(\Delta\)-edge-colouring of \(G\), and we are done. Now assume that \(\alpha \neq \beta\). Without loss of generality, we can assume that \(x\) is incident with a \(\beta\)-edge \(xu\). Extend the \(\beta\)-edge \(xu\) to a maximal walk \(W\) whose edges are coloured \(\alpha, \beta\) alternately. Since no such walk can contain a vertex colour twice, \(W\) is a path.
        \begin{quote}
            {\bf Claim.} \(W\) does not contain \(y\).

                {\bf Proof.} For a contradiction, suppose that \(y\) lies on \(W\). Then \(y\) must be an endvertex of \(W\), and the edge of \(W\) incident with \(y\) must be an \(\alpha\)-edge. Hence \(W\) has even length, and so \(W + xy\) is an odd cycle in the bipartite graph \(G\). This is a contradiction.
        \end{quote}
        By maximality of \(W\), we can swap the colours \(\alpha\) and \(\beta\) on all edges of \(W\). This gives a new \(\Delta\)-edge-colouring of \(G'\) such that \(\beta\) does not appear on any edge incident with \(x\). Since y does not lie on \(W\), there is still no \(\beta\)-edge incident with \(y\). FInally we can colour edge \(xy\) with colour \(\beta\) in \(G\), giving a \(\Delta\)-edge-colouring of \(G\). This completes the proof, by induction.
    }
\end{proposition}

\begin{theorem}[Vizing, 1964]
    Every graph \(G\) satisfies
    \[\Delta(G) \leq \chi'(G) \leq \Delta(G) + 1.\]
\end{theorem}

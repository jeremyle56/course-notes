\chapter{Connectivity}

\section{2-Connected Graphs}
Let \(G\) be a graph. A maximal connected subgraph of \(G\) with no cut vertex is called a \textbf{block}. Every block of \(G\) is either a maximal 2-connected subgraph of \(G\) or a bridge or an isolated vertex. \\

By maximality, different blocks of \(G\) overlap in at most one vertex, which must be a \textbf{cut vertex} in \(G\). Hence every edge of \(G\) lies in a unique block, and \(G\) is the union of its blocks. \\

Let \(A\) be the set of cut vertices in \(G\) and let \(\mathcal{B}\) be the set of blocks in \(G\). Form the bipartite graph on \(A \cup \mathcal{B}\) with edge set
\[\{ aB : a \in A, B \in \mathcal{B} \tand a \in B\}.\]

\begin{lemma}
    The block graph of a connected graph is a tree.
\end{lemma}

Let \(H\) be a subgraph of a graph \(G\). An \textbf{\(H\)-path} is a path in \(G\) which intersects \(H\) only in its endvertices.

\begin{proposition} \label{prop-5.1.2}
    A graph is 2-connected if and only if it can be constructed from a cycle by successively adding \(H\)-paths to graphs \(H\) already constructed.
    \proof{
        Every graph constructs in this way is \(2\)-connected. Conversely, let \(G\) be 2-connected. Then \(|G| \geq 3 \tand G\) contains a cycle. Hence \(G\) has a maximal subgraph \(H\) which is constructible using the method described in the proposition stated. \\

        If \(H = G\), then we are done. For a contradiction, suppose that \(H \neq G\). Since any edge \(xy \in E(G) - E(H)\) with \(x, y \in H\) is an \(H\)-path, by maximality we see that every \(xy \in E(G)\) with \(x,y \in H\) must belong to \(E(H)\). Hence, \(H\) is an induced subgraph of \(G\). \\

        By the fact that \(G\) is connected, there is an edge \(vw\) with \(v \in G - H, w \in H\). Since \(G\) is 2-connected we know that \(G - W\) is connected. Let \(P\) be the shortest path from \(v\) to \(H\) in \(G - w\). Then \(wvP\) is a \(H\)-path in \(G\), and \(H \cup wvP\) is a larger constructible subgraph than \(H\), contradicting the maximality of \(H\).
    }
\end{proposition}

\section{3-Connected Graphs}
Let \(e = xy \in E(G)\). Define the graph \(G / e = (V', E')\) where \(V' = (V - \{x, y\}) \cup \{v_e\}\),
\[E' = \{uw \in E(G): \{u, w\} \cup \{x, y\} = \emptyset\} \cup \{v_ew: xw \in E(G) \text{ or } yw \in E(G)\}.\]
We say that \(G / e\) is formed by \textbf{contradicting} the edge \(e\) in \(G\). This creates a new vertex \(v_e\) which replaces the endvertices of \(e\).

\begin{lemma}
    Let \(G\) be a 3-connected graph with \(|G| \geq 5\). Then \(G\) has an edge \(e\) such that \(G / e\) is 3-connected.
    \proof{
        For a contradiction, suppose that no such edge exists. For any edge \(xy \in E(G)\), the graph \(G / xy\) is not 3-connected, but \(|G / xy| = |G| - 4 \geq 4\) by assumption that \(|G| \geq 5\). Hence \(G / xy\) has a separating set \(S\) with \(|S| \leq 2\). Since \(G\) is 3-connected, the contracted vertex \(v_{xy}\) must belong to \(S\), and \(|S| = 2\), or we would have a separating set in \(G\) with \(\leq 2\) vertices. So there is some \(z \in V(G), z \notin \{x, y\}\) such that \(S = \{v_{xy}, z\}\). Any two vertices separated in \(G /xy\) by \(S\) are also separated in \(G\) by the set \(T = \{x, y, z\}\). \\

        \texttt{FACT}: Since no proper subset of \(T\) separates \(G\), by the 3-connectivity of \(G\), every vertex in \(T\) has a neighbour in every component \(C\) of \(G - T\). \\

        Choose the edge \(xy\), vertex \(z\), and component \(C\) of \(G - \{x, y, z\}\) such that \(|C|\) is as small as possible. Let \(v\) be a neighbour of \(z\) in \(C\), which we know must exists by our \texttt{FACT}. By assumption, \(G / zv\) is not 3-connected, and \(|G / zv| = |G| - 1 \geq 4\). Hence (by our earlier argument) there is a vertex \(w \notin \{v, z\}\) such that \(\{v, w, z\}\) separates \(G\). Also by our \texttt{FACT}, every vertex in \(\{v, w, z\}\) has a neighbour in every component of \(G - \{v, w, z\}\). \\

        Since \(x\) and \(y\) are adjacent, \(G - \{z, v, w\}\) has a component \(D\) such that \(D \cap \{x, y\} = \emptyset\). By our \texttt{FACT} we know that \(v\) has a neighbour in \(D\). Recall that \(v \in C\) in \(G - \{x, y, z\}\). Since \(D\) is connected and \((\{v\} \cup V(D)) \cap \{x, y, z\}\), it follows that \(\{v\} \cup V(D) \subseteq V(C)\). Hence \(D\) is a proper subgraph of \(C\), as \(v \notin V(D)\). Therefore \(|D| < |C|\), contradicting the minimality of \(C\). \\

        Hence \(G / e\) is 3-connected for some \(e \in E(G)\).
    }
\end{lemma}

Reversing this, we can construct all 3-connected graphs starting with \(K_4\) and ``uncontracting'' edges.

\begin{theorem}
    A graph \(G\) is 3-connected if and only if there exists a sequence \(G_0, G_1, \dots, G_r\) of graphs such that
    \begin{enumerate}[label=(\roman*)]
        \item \(G_0 = K_4\) and \(G_r = G\),
        \item \(G_{i + 1}\) has an edge \(xy\) with degrees \(d(x), d(y) \geq 3\) such that \(G_i = G_{i + 1}/xy\), for \(i = 0, \dots, r - 1\).
    \end{enumerate}
\end{theorem}

\section{Menger's Theorem}
A set \(S \subset V\) separating \(A\) from \(B\) in \(G\) is called an \textbf{\((A, B)\)-separator}. This means that every \((A, B)\)-path intersects \(S\), and in particular \(A \cap B \subseteq S\). \\

Let \(\mathcal{P}, \mathcal{Q}\) be sets of \textbf{disjoint \((A, B)\)-paths} in \(G\). Say that \(\mathcal{Q}\) \textbf{exceeds} \(\mathcal{P}\) if the set of vertices in \(A\) which belong to paths in \(\mathcal{P}\) is a \textit{proper subset} of the set of vertices in \(A\) which belong to paths in \(\mathcal{Q}\) and similarly for \(B\). \\

If \(P = x_0 x_1 \cdots x_k\) then we write \(P_{x_i}\) for the subpath \(x_0 \cdots x_i\) and we write \(x_i P\) for the subpath \(x_i x_{i + 1} \cdots x_k\).

\begin{theorem}[Menger's Theorem, 1927]
    Let \(G = (V, E)\) be a graph and \(A, B \subseteq V\). Then the minimum number of vertices separating \(A\) from \(B\) in \(G\) equals the maximum number of disjoint \((A, B)\)-paths in \(G\).
    \proof{
    Let \(k = k(G, A, B)\) be the minimum number of vertices separating \(A\) and \(B\) in \(G\). (That is, \(k = |S|\) where \(S \subseteq V\) is a smallest \((A, B)\)-separating set.) Then \(k\) is an upper bound on the maximum number of disjoint \((A, B)\)-paths or else we could not separate \(A\) and \(B\) by deleting any set of \(k\) vertices. So it suffices to prove that a set of \(k\) disjoint \((A, B)\)-paths exists. In fact, we will prove a stronger statement:
    \begin{quote}
        If \(\mathcal{P}\) is any set of \(< k\) disjoint \((A, B)\)-paths, then there is a set \(\mathcal{Q}\) of \(|\mathcal{P}| + 1\) disjoint \((A, B)\)-paths in \(G\) which exceeds \(\mathcal{P}\).
    \end{quote}
    We will keep \(G\) and \(A\) fixed and let \(B\) vary, applying induction on the number of vertices in \(\bigcup_{P \in \mathcal{P}} P\). \\

    {\it Base Case:} If \(\mathcal{P} = \emptyset\) then \(|\bigcup_{P \in \mathcal{P}} P| = 0\). We can let \(\mathcal{Q} = \{\mathcal{P}\}\) for any \((A, B)\)-path \(P\). Then \(\mathcal{Q}\) exceeds \(\mathcal{P}\). \\

    {\it Inductive Step:} Let \(\mathcal{P}\) be a set of \(< k\) disjoint \((A, B)\)-paths, and \(B_0 \subseteq B\) be the set of end vertices of paths in \(\mathcal{P}\) (``start vertices'' are in \(A\), ``endvertices'' are in \(B\)). Since \(|B_0| \leq k - 1, B_0\) is not an \((A, B)\)-separating set and hence there is an \((A, B)\)-path in \(G - B_0\). Call this \((A, B)\)-path \(R\). So \(R\) is disjoint from \(B_0\). If \(R\) avoids all vertices in \(\bigcup_{P \in \mathcal{P}}\) then \(\mathcal{Q} = \mathcal{P} \cup \{R\}\) exceeds \(\mathcal{P}\), as required. Otherwise, let \(x\) be the last vertex of \(R\) (traversing \(R\) from \(A\) to \(B\)) that lies on some path \(P \in \mathcal{P}\). Note that \(x \notin B\), by choice of \(R\), so \(Px\) is shorter than \(P\). \\

    Let \(B' = B \cup V(xP \cup xR)\) and let \(\mathcal{P}' = (\mathcal{P} - \{P\}) \cup \{Px\}\). Then \(\mathcal{P}'\) is a set of disjoint \((A, B')\)-paths. Also \(|\mathcal{P}'| = |\mathcal{P}|\), but the union of paths in \(\mathcal{P'}\) is strictly smaller than \(|\bigcup_{\hat{P} \in \mathcal{P}} \hat{P}|\). Also, \(B \subseteq B'\), so an \((A, B')\)-separating set is also an \((A, B)\)-separating set. Hence \(k(G, A, B') \geq k(G, A, B)\). So \(|\mathcal{P'} < k(G, A, B) \leq k(G, A, B')\). Applying the inductive hypothesis to \(G, A, B', \mathcal{P'}\), we conclude that there is a set \(\mathcal{Q'}\) of \(|\mathcal{P}| + 1\) disjoint \((A, B')\)-paths in \(G\) which exceeds \(\mathcal{P}\)! Now \(\mathcal{Q'}\) contains a path \(Q\) which ends in \(x\), and a unique path \(Q'\) whose last vertex \(y\) is not among the last vertices of the paths in \(\mathcal{P}\)! In particular, \(y \neq x\).
    \begin{quote}
        {\it Case 1:} \(y \in B\).  If \(y \in B\), then define \(\mathcal{Q} = (\mathcal{Q'} - \{Q\}) \cup \{QxP\}\)

        {\it Case 2:} \(y \notin B \tand y \in xR\). If \(y \in xR \tand y \notin B\), then \(y \notin xP\), and we define \(\mathcal{Q} = (\mathcal{Q'} - \{Q, Q'\}) \cup \{QxP, Q'yR\}\).

            {\it Case 3:} \(y \notin B \tand y \in xP\). If \(y \in xP \tand y \notin\) then \(y \notin xR\), and we define \(\mathcal{Q} = (\mathcal{Q'} - \{Q, Q'\} \cup \{QxR, Q'yP\})\).
    \end{quote}
    In all cases, \(Q\) is a set of \(|\mathcal{P}| + 1\) disjoint \((A, B)\)-paths which exceeds \(\mathcal{P}\), proving the inductive step. Hence there is a set of \(k\) disjoint \((A, B)\)-paths in \(G\), as required.
    }
\end{theorem}

\begin{corollary}
    Let \(a, b\) be distinct vertices of \(G\).
    \begin{enumerate}[label=(\roman*)]
        \item If \(ab \notin E\) then the minimum number of vertices (distinct from \(a\) and \(b\)) separating \(a\) from \(b\) is equal to the maximum number independent \((a, b)\)-paths in \(G\).
        \item The minimum number of edges separating \(a\) from \(b\) in \(G\) equals the maximum number of edge-disjoint \((a, b)\)-paths in \(G\).
    \end{enumerate}
    \proof{
        \begin{enumerate}[label=(\roman*)]
            \item Apply Menger's Theorem with \(A = N(A), B = N(b)\). Note that a set of \(k\) disjoint \((A, B)\)-path corresponds to a set of independent \((a, b)\)-paths by adding vertex \(a\) at the start and vertex \(b\) to the end.
            \item Apply Menger's Theorem to the line graph \(L(G)\) of \(G\) with \(A = E(a)\), the set of edges of \(G\) incident with \(a\), \(B = E(b)\), the set of edges of \(G\) incident with \(b\).
        \end{enumerate}
    }
\end{corollary}

\begin{theorem}[Global version of Menger's Theorem]
    \phantom{}
    \begin{enumerate}[label=(\roman*)]
        \item A graph is \(k\)-connected if and only if it has order at least 2 and there are \(k\) independent paths between any two distinct vertices.
        \item A graph is \(k\)-edge-connected if and only if it has at least two vertices and \(k\) edge-disjoint paths between any two distinct vertices.
    \end{enumerate}
    \proof{
        \begin{enumerate}[label=(\roman*)]
            \item Suppose that \(G\) is a graph and \(|G| \geq 2\). Now suppose that \(G\) has \(k\) independent paths between any two distinct vertices \(a, b \in V\). Then \(|G| \geq k\), as there are at least \(k - 1\) paths of length at least two between \(a\) and \(b\). Also, \(G\) cannot be disconnected by deleting a set of \(\leq k - 1\) vertices. Hence \(G\) is \(k\)-connected.

                  For the converse, suppose that \(G\) is \(k\)-connected and assume for a contradiction that there are distinct vertices \(a, b\) with at most \(k - 1\) independent \((a, b)\)-paths. Since \(G\) is \(k\)-connected we have \(|G| \geq k + 1\). By Corollary 5.3.2, we must have \(ab \in E\). Let \(G' = G - ab\). Then \(G'\) has at most \(k - 2\) independent \((a, b)\)-paths. Hence by Corollary 5.3.2, there is an \((a, b)\)-separating set \(X \subseteq V\) with \(|X| \leq k - 2\). Since \(|G| \geq k + 1\), there is at least one more vertex \(v \notin X \cup \{a, b\}\) in \(G\). Now \(X\) separates \(v\) from at least one of \(a\) or \(b\), say from \(a\) (since \(a, b\) lie in distinct components of \(G' - X\)). But then \(X \cup \{b\}\) is a set of at most \(k - 1\) vertices which separates \(v\) from \(a\) in \(G\). This contradicts the fact that \(G\) is \(k\)-connected.

                  Hence \(G\) has at least \(k\) independent \((a, b)\)-paths in \(G\), completing the proof.
            \item Follows immediately from Corollary 5.3.2.
        \end{enumerate}
    }
\end{theorem}
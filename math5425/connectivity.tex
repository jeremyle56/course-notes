\chapter{Connectivity}

\section{2-Connected Graphs}
Let \(G\) be a graph. A maximal connected subgraph of \(G\) with no cut vertex is called a \textbf{block}. Every block of \(G\) is either a maximal 2-connected subgraph of \(G\) or a bridge or an isolated vertex. \\

By maximality, different blocks of \(G\) overlap in at most one vertex, which must be a \textbf{cut vertex} in \(G\). Hence every edge of \(G\) lies in a unique block, and \(G\) is the union of its blocks. \\

Let \(A\) be the set of cut vertices in \(G\) and let \(\mathcal{B}\) be the set of blocks in \(G\). Form the bipartite graph on \(A \cup \mathcal{B}\) with edge set
\[\{ aB : a \in A, B \in \mathcal{B} \tand a \in B\}.\]

\begin{lemma}
    The block graph of a connected graph is a tree.
\end{lemma}

Let \(H\) be a subgraph of a graph \(G\). An \textbf{\(H\)-path} is a path in \(G\) which intersects \(H\) only in its endvertices.

\begin{proposition}
    A graph is 2-connected if and only if it can be constructed from a cycle by successively adding \(H\)-paths to graphs \(H\) already constructed.
    \proof{
        Every graph constructs in this way is \(2\)-connected. Conversely, let \(G\) be 2-connected. Then \(|G| \geq 3 \tand G\) contains a cycle. Hence \(G\) has a maximal subgraph \(H\) which is constructible using the method described in the proposition stated. \\

        If \(H = G\), then we are done. For a contradiction, suppose that \(H \neq G\). Since any edge \(xy \in E(G) - E(H)\) with \(x, y \in H\) is an \(H\)-path, by maximality we see that every \(xy \in E(G)\) with \(x,y \in H\) must belong to \(E(H)\). Hence, \(H\) is an induced subgraph of \(G\). \\

        By the fact that \(G\) is connected, there is an edge \(vw\) with \(v \in G - H, w \in H\). Since \(G\) is 2-connected we know that \(G - W\) is connected. Let \(P\) be the shortest path from \(v\) to \(H\) in \(G - w\). Then \(wvP\) is a \(H\)-path in \(G\), and \(H \cup wvP\) is a larger constructible subgraph than \(H\), contradicting the maximality of \(H\).
    }
\end{proposition}
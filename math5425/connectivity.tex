\chapter{Connectivity}

\section{2-Connected Graphs}
Let \(G\) be a graph. A maximal connected subgraph of \(G\) with no cut vertex is called a \textbf{block}. Every block of \(G\) is either a maximal 2-connected subgraph of \(G\) or a bridge or an isolated vertex. \\

By maximality, different blocks of \(G\) overlap in at most one vertex, which must be a \textbf{cut vertex} in \(G\). Hence every edge of \(G\) lies in a unique block, and \(G\) is the union of its blocks. \\

Let \(A\) be the set of cut vertices in \(G\) and let \(\mathcal{B}\) be the set of blocks in \(G\). Form the bipartite graph on \(A \cup \mathcal{B}\) with edge set
\[\{ aB : a \in A, B \in \mathcal{B} \tand a \in B\}.\]

\begin{lemma}
    The block graph of a connected graph is a tree.
\end{lemma}

Let \(H\) be a subgraph of a graph \(G\). An \textbf{\(H\)-path} is a path in \(G\) which intersects \(H\) only in its endvertices.

\begin{proposition}
    A graph is 2-connected if and only if it can be constructed from a cycle by successively adding \(H\)-paths to graphs \(H\) already constructed.
    \proof{
        Every graph constructs in this way is \(2\)-connected. Conversely, let \(G\) be 2-connected. Then \(|G| \geq 3 \tand G\) contains a cycle. Hence \(G\) has a maximal subgraph \(H\) which is constructible using the method described in the proposition stated. \\

        If \(H = G\), then we are done. For a contradiction, suppose that \(H \neq G\). Since any edge \(xy \in E(G) - E(H)\) with \(x, y \in H\) is an \(H\)-path, by maximality we see that every \(xy \in E(G)\) with \(x,y \in H\) must belong to \(E(H)\). Hence, \(H\) is an induced subgraph of \(G\). \\

        By the fact that \(G\) is connected, there is an edge \(vw\) with \(v \in G - H, w \in H\). Since \(G\) is 2-connected we know that \(G - W\) is connected. Let \(P\) be the shortest path from \(v\) to \(H\) in \(G - w\). Then \(wvP\) is a \(H\)-path in \(G\), and \(H \cup wvP\) is a larger constructible subgraph than \(H\), contradicting the maximality of \(H\).
    }
\end{proposition}

\section{3-Connected Graphs}
Let \(e = xy \in E(G)\). Define the graph \(G / e = (V', E')\) where \(V' = (V - \{x, y\}) \cup \{v_e\}\),
\[E' = \{uw \in E(G): \{u, w\} \cup \{x, y\} = \emptyset\} \cup \{v_ew: xw \in E(G) \text{ or } yw \in E(G)\}.\]
We say that \(G / e\) is formed by \textbf{contradicting} the edge \(e\) in \(G\). This creates a new vertex \(v_e\) which replaces the endvertices of \(e\).

\begin{lemma}
    Let \(G\) be a 3-connected graph with \(|G| \geq 5\). Then \(G\) has an edge \(e\) such that \(G / e\) is 3-connected.
    \proof{
        For a contradiction, suppose that no such edge exists. For any edge \(xy \in E(G)\), the graph \(G / xy\) is not 3-connected, but \(|G / xy| = |G| - 4 \geq 4\) by assumption that \(|G| \geq 5\). Hence \(G / xy\) has a separating set \(S\) with \(|S| \leq 2\). Since \(G\) is 3-connected, the contracted vertex \(v_{xy}\) must belong to \(S\), and \(|S| = 2\), or we would have a separating set in \(G\) with \(\leq 2\) vertices. So there is some \(z \in V(G), z \notin \{x, y\}\) such that \(S = \{v_{xy}, z\}\). Any two vertices separated in \(G /xy\) by \(S\) are also separated in \(G\) by the set \(T = \{x, y, z\}\). \\

        \texttt{FACT}: Since no proper subset of \(T\) separates \(G\), by the 3-connectivity of \(G\), every vertex in \(T\) has a neighbour in every component \(C\) of \(G - T\). \\

        Choose the edge \(xy\), vertex \(z\), and component \(C\) of \(G - \{x, y, z\}\) such that \(|C|\) is as small as possible. Let \(v\) be a neighbour of \(z\) in \(C\), which we know must exists by our \texttt{FACT}. By assumption, \(G / zv\) is not 3-connected, and \(|G / zv| = |G| - 1 \geq 4\). Hence (by our earlier argument) there is a vertex \(w \notin \{v, z\}\) such that \(\{v, w, z\}\) separates \(G\). Also by our \texttt{FACT}, every vertex in \(\{v, w, z\}\) has a neighbour in every component of \(G - \{v, w, z\}\). \\

        Since \(x\) and \(y\) are adjacent, \(G - \{z, v, w\}\) has a component \(D\) such that \(D \cap \{x, y\} = \emptyset\). By our \texttt{FACT} we know that \(v\) has a neighbour in \(D\). Recall that \(v \in C\) in \(G - \{x, y, z\}\). Since \(D\) is connected and \((\{v\} \cup V(D)) \cap \{x, y, z\}\), it follows that \(\{v\} \cup V(D) \subseteq V(C)\). Hence \(D\) is a proper subgraph of \(C\), as \(v \notin V(D)\). Therefore \(|D| < |C|\), contradicting the minimality of \(C\). \\

        Hence \(G / e\) is 3-connected for some \(e \in E(G)\).
    }
\end{lemma}

Reversing this, we can construct all 3-connected graphs starting with \(K_4\) and ``uncontracting'' edges.

\begin{theorem}
    A graph \(G\) is 3-connected if and only if there exists a sequence \(G_0, G_1, \dots, G_r\) of graphs such that
    \begin{enumerate}[label=(\roman*)]
        \item \(G_0 = K_4\) and \(G_r = G\),
        \item \(G_{i + 1}\) has an edge \(xy\) with degrees \(d(x), d(y) \geq 3\) such that \(G_i = G_{i + 1}/xy\), for \(i = 0, \dots, r - 1\).
    \end{enumerate}
\end{theorem}

\section{Menger's Theorem}
A set \(S \subset V\) separating \(A\) from \(B\) in \(G\) is called an \textbf{\((A, B)\)-separator}. This means that every \((A, B)\)-path intersects \(S\), and in particular \(A \cap B \subseteq S\). \\

Let \(\mathcal{P}, \mathcal{Q}\) be sets of \textbf{disjoint \((A, B)\)-paths} in \(G\). Say that \(\mathcal{Q}\) \textbf{exceeds} \(\mathcal{P}\) if the set of vertices in \(A\) which belong to paths in \(\mathcal{P}\) is a \textit{proper subset} of the set of vertices in \(A\) which belong to paths in \(\mathcal{Q}\) and similarly for \(B\).

\begin{theorem}
    Let \(G = (V, E)\) be a graph and \(A, B \subseteq V\). Then the minimum number of vertices separating \(A\) from \(B\) in \(G\) equals the maximum number of disjoint \((A, B)\)-paths in \(G\).
\end{theorem}
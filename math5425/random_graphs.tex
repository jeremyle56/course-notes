\chapter{Random Graphs}

We define the uniform model of random graphs in a similar manner to what was done in the Probabilistic Method chapter. \\

For some probability \(p \in [0, 1]\), each pair of distinct vertices \(\{i, j \}\) let \(\Pr(ij \in E) = p\) independently for each \(i \neq j\). This gives a random graph model called the binomial model denoted \(G(n, p)\). Note \(G(n, \frac{1}{2})\) is the uniform model. \\

We write \(G \in G(n, p)\) to mean that \(G\) is a random graph chosen from the binomial model. For a fixed \(G_0 \in \Omega_n\), the probability that the random graph \(G\) equals \(G_0\) is
\[\Pr(G = G_0) = p^{|E(G_0)|} (1 - p)^{\binom{n}{2} - |E(G_0)|}\]
which depends only on \(|E(G_0)|\) using independence. \\

For \(G \in G(n, p)\), the expected number of edges of \(G\) is \(p\binom{n}{2}\). \\

For fixed \(p \in [0, 1]\), we have a \textbf{sequence} of probability spaces,
\[(G(n, p))_{n \in \bZ^+}.\]
We can also let \(p\) be a function of \(n\), where \(p(n) \in [0, 1]\) for all \(n \in \bZ^+\). This gives the sequence of probability spaces
\[(G(n, p(n)))_{n \in \bZ^+}.\]

Recall that \(\omega(G)\) is the clique number of \(G\), and \(\alpha(G)\) is the independence number of \(G\).

\begin{lemma} \label{lemma-8.0.1}
    Let \(G \in G(n, p)\). Then for any integer \(k \geq 2\),
    \begin{align*}
        \Pr(\omega(G) \geq k) & \leq \binom{n}{k} p^{\binom{k}{2}}, \\
        \Pr(\alpha(G) \geq k) & \leq \binom{n}{k} (1 - p)^{\binom{k}{2}}.
    \end{align*}
    \proof{
        Let \(G \in G(n, p)\). If \(G\) has a clique of order \(\geq k\) then \(G\) has a clique of order \(k\). For a set \(S\) of \(k\) vertices, let \(A_s\) be the event ``\(G[S]\) is a clique''. Then \(\Pr[A_s] = p^{\binom{k}{2}}\), using independence, since \(\binom{k}{2}\) edges within in \(S\) must be present. Hence
        \[\Pr(\omega(G) \geq k) = \Pr \left(\bigcup_{|S| = k} A_s \right) \leq \sum_{|S| = k} \Pr(A_s) = \binom{n}{k} p^{\binom{k}{2}}\]
        (using the union bound), the result as required.
    }
\end{lemma}

For \(a \in \bR\) and \(r \in \bN\), let
\[(a)_r = a(a - 1) \cdots (a - r + 1)\]
denote the \textbf{falling factorials}.

\begin{lemma} \label{lemma-8.0.2}
    Let \(k \geq 3\) be an integer. The expected number of \(k\)-cycle in \(G \in G(n, p)\) is
    \[\frac{(n)_k}{2k} p^k.\]
    \proof{
        Let \(X\) be the number of \(k\)-cycles in \(G \in G(n, p)\). Given a sequence \((v_1, v_2, \dots, v_k)\) of \(k\) distinct vertices, the probability that this sequence describes a walk around a \(k\)-cycle is
        \[\Pr(v_1v_2, v_2,v_3, \dots, v_{k-1}v_k, v_kv_1 \in E(G)) = p^k, \text{ using independence}\]
        There are \((n)_k\) ways to choose this sequence of \(k\) distinct vertices. Each cycle in \(G\) corresponds to exactly \(2k\) such sequences corresponding to the choice of start vertex and direction. \\

        Hence, by linearity of expectation, \(\bE X = \frac{(n)_k}{2k} p^k, \text{ as claimed.}\)
    }
\end{lemma}

If \(\Pr(G \in \mathcal{P}) \to 1\) as \(n \to \infty\), for some graph property \(\mathcal{P}\), we say that \(G \in P\) holds \textbf{asymptotically almost surely}, abbreviated to ``a.a.s.''.

\begin{proposition}
    For fixed \(p \in (0, 1)\) and every graph \(H\), a.a.s. \(G \in G(n, p)\) has an induced subgraph which is isomorphic to \(H\).
    \proof{
        Let \(k = |V(H)|\). Suppose that \(n \geq k\) and let \(\mathcal{U} \subseteq \{1, 2, \dots, n\}\) be a fixed set of \(k\) vertices. The probability that \(G[\mathcal{U}] \cong H\) is some fixed constant \(r \in (0, 1)\) which depends only on \(H\) and \(P\) but not on \(n\). \\

        Now we can find \(\lfloor \frac{n}{k} \rfloor\) disjoint sets of \(k\) vertices, \(\mathcal{U}_1, \dots, \mathcal{U}_{\lfloor \frac{n}{k} \rfloor}\), within \(V(G) = [n]\). The probability that none of \(\mathcal{U}_1, \dots, \mathcal{U}_{\lfloor \frac{n}{k} \rfloor}\) induces a copy of \(H\) is \((1 - r)^{\lfloor \frac{n}{k} \rfloor}\), since the \(\mathcal{U}_j\) are disjoint and hence the events \(G[\mathcal{U}_j] \ncong H\) are independent of each other (for \(j = 1, \dots, \lfloor \frac{n}{k} \rfloor\)). \\

        But \((1 - r)^{\lfloor \frac{n}{k} \rfloor} \to 0\) as \(n \to \infty\), since \(1 - r \in (0, 1)\) and \(\lfloor \frac{n}{k} \rfloor \to \infty\) as \(n \to \infty\). Hence a.a.s., one of \(\mathcal{U}_1, \dots, \mathcal{U}_{\lfloor \frac{n}{k} \rfloor}\) induces a copy of \(H\).
    }
\end{proposition}

Given \(i, j \in \bN\), let \(\mathcal{P}_{ij}\) be the property that given any disjoint vertex set \(U, W\) with \(|U| \leq i\) and \(|W| \leq j\), the graph contains a vertex \(v \in U \cup W\) that is adjacent to all vertices in \(U\) but none in \(W\).

\begin{lemma} \label{lemma-8.0.4}
    For every constant \(p \in (0, 1)\) and all \(i, j \in \bN\), let \(G \in G(n, p)\). Then a.a.s. \(G \in \mathcal{P}_{ij}\).
    \proof{
        Assume that \(n \geq i + j + 1\). For fixed disjoint set \(U, W \subseteq [n]\) and \(v \in [n] - (U \cup W)\), the probability that \(v\) is adjacent to all vertices of \(U\) and to no vertices of \(W\) is \(p^{|U|}(1 - p)^{|W|} \geq p^i(1 - p)^j\) using independence. To simplify notation we write \(q = 1 - p\). Hence the probability that no such \(v\) exists for the given sets \(U\) and \(W\) is
        \[(1 - p^{|U|}q^{|W|})^{n - |U| - |W|}\]
        since these events are independent for distinct \(v \in U \cup W\) (no edge/non-edge choices are considered in more than one of these events). Now
        \begin{align*}
            (1 - p^{|U|}q^{|W|})^{n - |U| - |W|} &\leq (1 - p^jq^j)^{n - |U| - |W|} \\
            &\leq (1 - p^i q^j)^{n - i - j}.
        \end{align*}
        There are at most \(n^{i + j + 2}\) pairs of disjoint sets \(U, W\) with \(|U| \leq i\) and \(|W| \leq j\), as
        \[\sum_{s = 0}^i \binom{n}{s} \leq \sum_{s = 0}^i n^s = \frac{n^{i + 1} - 1}{n - 1} \leq n^{i + 1},\]
        and similarly for \(W\). Hence the probability that some \(U, W\) has no suitable \(v\) is at most
        \[n^{i + j + 2}(1 - p^i q^j)^{n - i - j} \to 0 \text{ as } n \to \infty\]
        since \(1 - p^i q^j \in (0, 1)\). Hence a.a.s. \(\mathcal{P}_{ij}\) holds, as required.
    }
\end{lemma}

\begin{corollary}
    For every constant \(p \in (0 , 1)\) and all \(k \in \bN\), a.a.s. \(G \in G(n, p)\) is \(k\)-connected.
    \proof{
        By Lemma \ref{lemma-8.0.4}, it is enough to show that every graph in \(\mathcal{P}_{2, k - 1}\) is \(k\)-connected when \(n\) is sufficiently large. Assume that \(n \geq k + 2\) (one more than is needed for \(k\)-connectivity). Let \(W\) be any set of at most \(k - 1\) vertices. We want to prove that \(G - W\) is still connected. So let \(x, y\) be distinct vertices in \([n] - W\) and define \(U = \{x, y\}\). By definition of \(\mathcal{P}_{2, k - 1}\) there is a vertex \(v\) in \([n] - (U \cup W)\) such that \(v\) is adjacent to both \(x\) and \(y\). Hence \(xvy\) is a path between \(x\) and \(y\) in \(G - W\), proving that \(G - W\) is connected.
    }
\end{corollary}

\begin{proposition}
    For every constant \(p \in (0, 1)\) and all \(\epsilon > 0\), a.a.s. \(G \in G(n, p)\) satisfies
    \[\chi(G) \geq \frac{\ln(1 / q)n}{(2 + \epsilon) \ln n}\]
    where \(q = 1 - p\).
    \proof{
        Let \(a\) be any fixed integer, \(2 \leq a \leq n\). Then by Lemma \ref{lemma-8.0.1}
        \begin{align*}
            \Pr (\alpha(G) \geq a) &\leq \binom{n}{a}(1 - p)^{\binom{a}{2}} \\
            &\leq n^a(1 - p)^{\binom{a}{2}} \\
            &= q^{a \frac{\ln n}{\ln q} + \frac{a(a - 1)}{2}} \\
            &= q^{\frac{a}{2}(\frac{2 \ln n}{\ln q} + a - 1)} \\
            &= q^{\frac{a}{2}(a - 1 - \frac{2 \ln n}{\ln(1 / q)})}
        \end{align*}
        Set \(a = \lceil \frac{(2 + \epsilon) \ln n}{\ln(1 / q)} \rceil\). Then
        \[\lim_{n \to \infty} \frac{a}{2}\left(a - 1 - \frac{2 \ln n}{\ln(1/q)}\right) \geq \lim_{n \to \infty} \frac{(2 + \epsilon)\ln n}{2\ln(1/q)}\left(\frac{\epsilon \ln n}{\ln(1 / q)} - 1\right) = \infty\]
        Hence \(\Pr(\alpha(G) \geq a) \to \infty\) as \(n \to \infty\), since \(q \in (0, 1)\). \\

        This shows that a.a.s. \(G \in G(n, p)\) has no independent set of order \(\lceil \frac{(2 + \epsilon)\ln n}{\ln (1 / q)}\rceil\), and hence a.a.s. \(\alpha(G) < \frac{(2 + \epsilon) \ln n}{\ln (1 / q)}\). Therefore a.a.s. for \(G \in G(n, p)\),
        \[\chi(G) \geq \frac{n}{\alpha(G)} > \frac{\ln (1 / q) n}{(2 + \epsilon) \ln n}.\]
    }
\end{proposition}

\begin{lemma} \label{lemma-8.0.7}
    Let \(k\) be a positive integer and let \(p = p(n)\) be a function of \(n\) such that \(p(n) \in (0, 1)\) and
    \[p(n) \geq \frac{6k \ln n}{n}\]
    for sufficiently large \(n\). Then for \(G \in G(n, p)\), a.a.s.
    \[\alpha(G) < \frac{n}{2k}.\]
    \proof{
        Let \(n, r \in \bZ, n \geq r \geq 2\). By Lemma \ref{lemma-8.0.1} for \(G \in G(n, p)\) we have
        \begin{align*}
            \Pr(\alpha(G) \geq r) &\leq \binom{n}{r}(1 - p)^{\binom{r}{2}} \\
            &\leq n^r(1 - p)^{\binom{r}{2}} \\
            &= (n(1 - p)^{\frac{r -1}{2}})^r \\
            &\leq (ne^{\frac{-p(r - 1)}{2}})^r
        \end{align*}
        since \(1 - p \leq e^{-p}\). If \(p \geq \frac{6k \ln n}{n}\) and \(r \geq \frac{n}{2k}\) then
        \begin{align*}
            ne^{-\frac{p(r - 1)}{2}} &= ne^{-\frac{pr}{2} + \frac{p}{2}} \\
            &\leq ne^{-\frac{3}{2}\ln n + \frac{p}{2}} \\
            &\leq ne^{-\frac{3}{2} \ln n + \frac{1}{2}} \\
            &= n \cdot n^{-\frac{3}{2}} \tag{since \(p \leq 1\)} \\
            &= \sqrt{\frac{e}{n}} \to 0 \text{ as } n \to \infty.
        \end{align*}
        Since \(p = p(n) \geq \frac{6k \ln n}{n}\) for sufficiently large \(n\), we take \(r = \lceil \frac{n}{2k} \rceil\) to conclude that
        \[\lim_{n \to \infty} \Pr(\alpha(G) \geq \frac{n}{2k}) = \lim_{n \to \infty} \Pr(\alpha(G) \geq r) = 0.\]
    }
\end{lemma}

Recall that the \textbf{girth} of a graph is the length of its smallest cycle and Markov's inequality: if \(X: \Omega \to \bN\) is a nonnegative integer-valued random variable on a set \(\Omega\), and \(k > 0\), then
\[\Pr(X \geq k) \leq \frac{\bE X}{k}.\]

\begin{theorem}[Erd\H os, 1959]
    For every integer \(k \geq 3\) there exists a graph \(H\) with girth \(g(H) > k\) and chromatic number \(\chi(H) > k\).
    \proof{
        Fix \(\epsilon\) with \(0 < \epsilon < \frac{1}{k}\) and let \(p = p(n) = n^{\epsilon - 1}\). Let \(X(G)\) be the number of cycles in \(G \in G(n, p)\) with length \(\leq k\). By Lemma \ref{lemma-8.0.2} and linearity of expectation,
        \[\bE X = \sum_{i = 3}^{k} \frac{(n)_i}{2i} p^i \leq \frac{1}{2} \sum_{i = 3}^k (np)^i \leq \frac{k - 2}{2}(np)^k,\]
        as \(np = n^\epsilon > 1\). Using Markov's inequality
        \[\Pr(X \geq \frac{n}{2}) \leq \frac{\bE X}{\frac{n}{2}} \leq (k - 2)n^{k - 1}p^k = (k - 2)n^{k - 1}n^{(\epsilon - 1)k} = (k - 2)n^{k \epsilon - 1}.\]
        Note \(k\epsilon < 1\) by choice of \(\epsilon\). Hence
        \[\lim_{n \to \infty} \Pr(X \geq \frac{n}{2}) = 0.\]
        That is, a.a.s. \(X(G) < \frac{n}{2}\). Note also that, \(p = n^{\epsilon} - 1 \geq \frac{6k \ln n}{n}\) for large enough \(n\), as \(k\) is constant. Hence by Lemma \ref{lemma-8.0.7}, we can choose \(n\) large enough so that
        \[\Pr(X \geq \frac{n}{2}) < \frac{1}{2} \tand \Pr(\alpha(G) \geq \frac{n}{2k}) < \frac{1}{2}.\]
        This shows that for some fixed graph \(G_0\) on \(n\) vertices we have \(\alpha(G) < \frac{n}{2k}\) and \(G_0\) has fewer than \(\frac{n}{2}\) cycles of length \(\leq k\). Construct \(H\) from \(G_0\) by deleting one vertex from every cycle in \(G_0\) of length \(\leq k\). Then \(|H| \geq \frac{n}{2}\) and by construction, \(g(H) > k\). \\

        Also \(\alpha(H) \leq \alpha(G_0) < \frac{n}{2k}\) since every independent set in \(H\) is also an independent set in \(G_0\). Therefore
        \[\chi(H) \geq \frac{|H|}{\alpha(H)} \geq \frac{\frac{n}{2}}{\alpha(H)} > \frac{\frac{n}{2}}{\frac{n}{(2k)}} = k,\]
        completing the proof.
    }
\end{theorem}
\chapter{Planar Graphs}

A graph which is drawn in the plane so that no edges meet except at common endvertices is called a \textbf{plane graph}. An abstract graph which can be drawn in this way is called \textbf{planar}. \\

A graph is drawn in the Euclidean plane \(\bR^2\) by representing each vertex by a point and each edge by a curve between two distinct points.

\section{Plane Graphs}

An \textbf{arc} (or \textbf{polygonal arc}) is a subset of \(\bR^2\) composed of the union of finitely many straight line segments, which is homeomorphic to \([0, 1]\). \\

A \textbf{plane graph} is a pair \((V, E)\) of finite sets (with elements of \(V\) called vertices and elements of \(E\) called edges) such that
\begin{enumerate}[label=(\roman*)]
    \item \(V \subseteq \bR^2\);
    \item Every edge is an arc between two distinct vertices (no loops);
    \item Different edges have different sets of endvertices (no repeated edges);
    \item The interior of an edge contains no vertex and no point of any other edge.
\end{enumerate}

Here the \textbf{interior} of an edge/arc \(e\), denoted \(\mathring{e}\), is the arc minus its endpoints: if \(e\) is the arc from \(x\) to \(y\) then \(\mathring{e} = e - \{x, y\}\). \\

A \textbf{plane graph} defines a graph \(G\) in a natural way. We use the name \(G\) for abstract graph, the plane graph and the \textbf{point set}
\[V \cup \left(\bigcup_{e \in E} e\right) \subseteq \bR^2.\]
The point set of a plane graph \(G\) is a closed set in \(\bR^2\), and \(\bR^2 - G\) is open. Two points in an open set \(O\) are equivalent if they are equal or they can be linked by an arc in \(O\). This is an equivalence relation. \\

The equivalence classes of \(\bR^2 - G\) are open connected regions, call the \textbf{faces} of \(G\). Since \(G\) is bounded (that is, it lies within some sufficiently large disc \(D \subseteq R^2\)), exactly \textbf{one} face of \(G\) is unbounded: it is the face that contains \(\bR^2 - D\). We call the unbounded faces the \textbf{outer face} of \(G\). All other faces of \(G\) are called inner faces. \\

Let \(F(G)\) be the set of faces of \(G\). The \textbf{boundary} of a face \(f\) is called the \textbf{frontier} of \(f\). It is the set of all points \(y \in \bR^2\) such that every neighbourhood of \(y\) meets both \(f\) and \(\bR^2 - f\).

\begin{lemma}
    Let \(G\) be a plane graph with subgraph \(H \subseteq G\) and face \(f \in F(G)\).
    \begin{enumerate}[label=(\roman*)]
        \item There is a face \(f' \in F(H)\) which contains \(f\) (that is, \(f \subseteq f'\)).
        \item If the frontier of \(f\) lies in \(H\) then \(f' = f\).
    \end{enumerate}
    \proof{
        \begin{enumerate}[label=(\roman*)]
            \item Points in \(f\) are also equivalent in \(\bR^2 - H\), so they belong to an equivalence class \(f'\) of \(\bR^2 - H\). That is, \(f \subseteq f' \tand f' \in F(H)\).
            \item We prove the contrapositive. Suppose that \(f\) is a proper subset of \(f' (f \subsetneq f')\). Choose points \(a \in f\) and \(b \in f' - f\). Both \(a \tand b\) belong to \(f\) in \(\bR^2 - H\), so there is an arc between them in \(\bR^2 - H\).

                  But \(a \tand b\) are not equivalent in \(\bR^2 - G\) as \(a \in f \tand b \notin f\). So the arc must meet a point \(x\) on the frontier \(X\) of \(f\), and \(x \notin H\) as \(x \in f' \subseteq \bR^2 - H\). Therefore \(X \notin H\).
        \end{enumerate}
    }
\end{lemma}

\begin{lemma}
    Let \(G\) be a plane graph and let \(e\) be an edge of \(G\).
    \begin{enumerate}[label=(\roman*)]
        \item If \(X\) is the frontier of a face of \(G\) then either \(e \subseteq X\) or \(X \cap \mathring{e} = \emptyset\).
        \item If \(e\) lies on a cycle \(C \subseteq G\) then \(e\) lies on the frontier of exactly two faces of \(G\), and these are contained in the distinct faces of \(C\).
        \item If \(e\) does not lie on a cycle then \(e\) lies on the frontier of exactly one face of \(G\).
    \end{enumerate}
\end{lemma}

\begin{corollary}
    The frontier of a face of a plane graph \(G\) is always the point set of a subgraph of \(G\).
\end{corollary}

The subgraph of \(G\) whose point set is the frontier of a face \(f\) is said to bound \(f\) and is called the \textbf{boundary} of \(f\). Denote this subgraph by \(G[f]\). A face is said to be \textbf{incident} with the vertices and edges of its boundary. By Lemma 6.1.1 (ii), every face of \(G\) is also a face of it's boundary.

\begin{proposition}
    A plane forest has exactly one face.
\end{proposition}

\begin{lemma}
    If a plane graph has two distinct faces with the same boundary then the graph is a cycle.
\end{lemma}

\begin{proposition}
    In a 2-connected plane graph, every face is bounded by a cycle.
\end{proposition}

\section{Greedy Algorithms}

\subsection{Foundations for The Greedy Method}
\paragraph{Method}
Search for an admissible solution of maximal reward (/minimal cost)
\begin{itemize}
    \item Introduce problem \textit{elements}
    \item Establish which combinations of elements are \textit{admissible}
    \item Define a \textit{quality} measure on problem's elements
    \item Build a solution step by step by adding elements of highest quality
\end{itemize}

\paragraph{Optimality proof method (exchange argument)}
\begin{itemize}
    \item Pick any solution \(S\)
    \item morph it by swapping elements with higher quality ones
    \item show that any swap leads to a solution with higher reward
    \item stop when we arrive at the greedy solution \(G\)
    \item Conclude that the reward \(G\) is larger than \(S\)
\end{itemize}

\subsection{Activity Selection Problem}
\paragraph{Setting} 
A list of activites \(a_i, (1 \leq i \leq n)\) with starting times
\(s_i\) and finishing times \(f_i\). No two activities can take place simultaneously.

\paragraph{Task} 
Find a \textit{maximum size} subset of compatible activities.

\paragraph{Solution} 
Among the activities which do not conflict with the previously chosen activities always chose
the one with the earliest end time.

\paragraph{Proof} 
Claim: any solution \(S\) has \(\leq\) number of activities than the greedy solution \(G\).
\begin{enumerate}
    \item Find the first place where the chosen activity violates the greedy choice.
    \item Show that replacing that activity with the greedy choice produces a non-conflicting 
    selection \(S^\prime\) with the same number of activities. 
    \item Continue until all activities match those in the greedy solution \(G\).
\end{enumerate}

\paragraph{Complexity} 
We sort activities using their finishing times in increasing order in \(O(n\log n)\) time.
Then loop through all activities linearly for a total time of \(O(n\log n)\).

\paragraph{Setting} A list of activities \(a_i, (1 \leq i \leq n)\) with starting times \(s_i\)
and finishing times \(f_i = s_i + d\). Thus, all activities are of  the same duration. No two
activities can take place simultaneously.

\paragraph{Task} Find a subset of compatible activities of \textit{maximal total duration}.

\paragraph{Solution} Since all activities are of the same duration, this is equivalent to finding
a selection with the largest number of non-conflicting activities, i.e., the previous problem.

A greedy strategy no longer works - we need a more sophisticated technique.

\subsection{Dijkstra's Shortest Path Algorithm}

\paragraph{Updating our Heap Data Structure}
%summeraise this bruh 
We will use heaps represented by arrays; the left child of \(A[i]\) is stored in \(A[2i]\)
and the right child in \(A[2i + 1]\). We will store in heaps vertices of graphs with key 
computed in various ways; if a graph has \(n\) vertices we will label them with positive 
integers 1 to \(n\). Thus every element of \(A\) is of the form \((i, k(i))\) where \(k(i)\)
is the key of element \(i\).

Besides the array \(A\) which represents the heap, we will use another array \(P\) of the 
same length which stores the position of elements in the heap; thus \(A[P[i]]=(i,k(i))\).
Changing the key of an element \(i\) is now an \(O(\lg n)\) operation: we look up its 
position \(P[i]\) in \(A\), change the key of the element in \(A[P[i]]\) and then perform
the Heappify operation to make sure the Heap property is being preserved. 

\paragraph{Setting} Let \(G=(V,E)\) be a directed graph with non-negative weight \(w(e) \geq 0\)
assigned to each edge \(e \in E\). We are also given a vertex \(v \in V\). For simplicity, we
assume that any \(u \in V\) can be reached from \(v\).

\paragraph{Task} Find for every \(u \in V\) the shortest path from \(v\) to \(u\).

\paragraph{Algorithm}
Starting from a set of vertices \(S = \{v\}\) which contains a single source vertex. At each
stage of construction we add the element \(u \in V \setminus S\) which has the shortest path
from \(v\) to \(u\) with all intermediate vertices already in \(S\).

\paragraph{Correctness}
Assume that there exists a shorter path from \(v\) to \(u\) in \(G\). By our choice of \(u\)
such a path cannot be entirely in \(S\). Let \(z\) be the first vertex outside \(S\) on 
such a shortest path. But then the path from \(v\) to such \(z\) would be shorter than the path
from \(v\) to \(u\), contradicting our choice of \(u\).

\paragraph{Efficient Implementation}
\begin{enumerate}
    \item All vertices expect \(v\) placed in heap with additional position array, weights
    \(w(u, v)\) if \((v, u)\in E\) or \(\infty\) as the key.
    \item Key of each element \(u\) will be updated with length \(lh_{S,v} (u)\) of the 
    shortest path from \(v\) to \(u\) which has all intermediate vertices on such a 
    path in \(S\).
    \item Pop the element \(u\) from the priority queue  with smallest key and add to \(S\).
    \item For all elements \(z \in V \setminus S\) for which \((u,z) \in E\), if 
    \(lh_{S,v} (u) + w(u, z) < lh_{s,v} (z)\) update key of \(z\) to \(lh_{S,v} (u) + w(u, z)\).
\end{enumerate} 

\paragraph{Complexity} For a graph with \(n\) vertices and \(m\) edges, each edge is 
inspected only once, and popping an element with the smallest key and updating a vertex 
key takes \(O(\lg n)\) many steps each. So in total, the algorithm runs in \(O(m\lg n)\) time.

\subsection{Huffman Code}
\paragraph{Encoding Texts}
Given a set of symbols you want to encode these symbols using binary strings, so that 
sequences of such symbols can be decoded in an unambiguous way.

\textbf{Fixed-width Encodings} Reverse bit strings of equal and sufficient length, given
the number of distinct symbols to be encoded. This is the main idea behind the ASCII code.

\paragraph{Towards Variable-Width Encoding}
The previous method is not economical: all symbols have codes of equal length. 
One would prefer an encoding in which frequent symbols have short codes
while infrequent ones can have longer codes.

\paragraph{Prefix Code}
The previous method was unable to partition a bitstream uniquely into segments.
To do this we use a prefix code. A prefix code is a map from symbols to bit sequences
such that no code of a symbol is a prefix of a code for another symbol.

\paragraph{The Huffman Code} 
Given the frequencies of each symbol, design an optimal prefix code, i.e. a prefix code 
such that the expect length of an encoded text is as small as possible. 

\subsection{Union-Find}

\paragraph{Three Operations}
\begin{itemize}
    \item \texttt{MakeUnionFind(S)} - Given a set \(S\) returns a structure in which all elements
    are placed into distinct singleton sets. Runs in \(O(n)\) time where \(n = |S|\).
    \item \texttt{Find(v)} - Given a vertex \(v\), returns the set to which
    \(v\) belong. Runs in \(O(1)\) time.
    \item \texttt{Union(A, B)} - Given two sets \(A, B\), changes the data
    structure by replacing sets \(A\) and \(B\) with the set \(A \cup B\). Initial
    sequence of \(k\) consecutive \texttt{Union} operations runs in time \(O(k\lg k)\).
        \begin{itemize}
            \item Run time of single \texttt{Union} not given. This approach is amortized analysis.
            \item \textit{inital} sequence of \(k\) \texttt{Union} operations means we start 
            with all sets being singletons and then apply \(k\) \texttt{Union} operations.
            \item \textit{consecutive} sequence of \texttt{Union} means a sequence of 
            \texttt{Union} operations possibly interspered with some \texttt{FIND} operations
            but not other \texttt{Union} operations not belonging to the considered sequence
            of \(k\) \texttt{Union} operations.
        \end{itemize}
\end{itemize}

\paragraph{Implementation}
The simplest implementation of the UF data structure consists of:
\begin{enumerate}
    \item an array \(A\) such that \(A[i] = j\) means that \(i\) belongs to set labeled by \(j\);
    \item an array \(B\) such that \(B[i]\) contains the number of elements in the set \(i\) and
    pointers to the first and last elements of the list of elements in the set \(i\).
\end{enumerate}
\texttt{Union(\(i, j\))} if defined as follows: if the set labeled by \(i\) has \(\geq\)
elements than the set labeled by \(j\) then labels in array \(A\) of all elements in the set
labeled by \(j\) is changed to \(i\) and array \(B\) is updated accordingly. Else do the 
opposite. 

\paragraph{Complexity}
Any sequence of \(k\) initial consecutive \texttt{Union} operations can touch at most \(2k\)
elements of \(S\). Every \texttt{Union} operation at least doubles the size of the set
and could change fewer than \(\lg 2k\) many times. Thus any sequence of \(k\) initial 
consecutive \texttt{Union} operations will have in total fewer than \(2k \lg 2k\) many
label changes. Thus, every sequence of \(k\) inital consecutive \texttt{Union} operations
has time complexity of \(O(k \lg k)\).

\subsection{Minimum Spanning Trees}
Let \(G = (V, E)\) be a connected undirected graph. 

\paragraph{Spanning Tree} A \textit{spanning tree} is a subgraph \(T=(V, E_T)\) of \(G\)
such that \(T\) does not contain any cycle and is connected.

\paragraph{Minimum Spanning Tree} If \(G\) is a (edge-) weighted graph, then a
\textit{minimum spanning tree} is a spanning tree of minimum weight. 

\paragraph{Kruskal's Algorithm}
Sort all edges \(E\) in non-decreasing order by weight. Then, starting from the lowest weight to
highest, if adding an edge will not result in a cycle, then add it to the graph. Otherwise,
discard that edge. The process terminates when the list of all edges has been exhausted.

\paragraph{Correctness} 
(Spanning Tree) 
Let \(T\) be the output of the algorithm, we know that \(T\) does not contian any cycle.
Assume there are two or more connected components \(C_1\) and \(C_2\). \(G\) is connected,
so there are some edges connecting \(C_1\) to \(C_2\) in \(G\). The first of such edges would 
have been added to \(T\) because it would not create any cycle in \(T\). So \(T\) is a spanning
tree. \\[0.2cm]
(Minimality)
We consider the case where all weights are distinct. Let \(T\) be the output of KA.
Consider a spanning tree \(T'\) distinct from \(T\). Let \(e = \{u, v\}\) be the smallest-weight
edge in \(T\) that is not in \(T'\). \(T'\) is spanning so there exists a path \(P\) from
\(u\) to \(v\). \(T\) has no cycles, so there exists an edge \(f \in P\) that is not in \(T\).
Let \(T''=(V, \{e\} \cup E_{T'} \setminus \{f\})\); it is a spanning tree. \(w(e) < w(f)\) because
otherwise KA would have added \(f\) to \(T\) instead of \(e\). Furthermore, \(T''\) weighs
less than \(T'\), so \(T'\) is no an MST. \(G\) has an MST and any \(T' \neq T\) is not an
MST, so \(T\) is an MST.

\paragraph{Implementation}
The \texttt{Union-Find} data structure lets us efficiently implement Kurskal's algorithm
on graph \(G = (V, E)\) with \(n\) vertices and \(m\) edges. We first sort \(m\) edges
which takes \(O(m \lg m)\). Since \(m \leq n^2\) this step also takes \(O(m \lg n^2) = O(m\log n)\).
For the algorithm we making connected components and merge them into a single connected 
component which is the same as \texttt{Union-Find}.

For each edge \(e = (v, u)\) we use two \texttt{Find} operations \texttt{Find(\(u\))} and
\texttt{Find(\(v\))} to determine if they belong in the same component. If they are
we add edge \(e = (u, v)\) to the spanning tree and perform \texttt{Union(\(i,j\))} to 
place \(u\) and \(v\) into the same connected component.

In total we perform \(2m\) \texttt{Find} operations, each costing \(O(1)\), in total coasting 
\(O(m)\). We also perform \(n-1\) \texttt{Union} operations which cost \(O(n \lg n)\).
In total, together with the initial sorting the time complexity is \(O(m \log n)\).

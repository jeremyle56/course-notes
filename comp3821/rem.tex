
\section{Mathematical Reminders}

\subsection{Geometric Series}
\paragraph{Claim:}  
\[ 
    \text{If} \hspace{0.2cm} r \neq 1 \hspace{0.2cm} \text{then} \hspace{0.2cm}
    \sum_{k=0}^m r^k = \frac{r^{m+1}-1}{r-1}
\]
\paragraph{Proof.}
\begin{align*}
    (r-1)\sum_{k=0}^m r^k &= (r-1)(r^m + r^{m-1} + \cdots + r + 1) \\
    &= (r^{m+1} + r^m + \cdots + r^2 + r) - (r^m + r^{m-1} + \cdots + r + 1) \\
    &= r^{m+1} - 1
\end{align*}

\subsection{Basic logarithm identity}
\paragraph{Claim:}  
\[ 
    \text{If} \hspace{0.2cm} a,b,c > 0 \hspace{0.2cm} \text{then} \hspace{0.2cm}
    a^{\log_b c} = c^{\log_b a}
\]
\paragraph{Proof.}
\begin{align*}
    \log_b c \cdot \log_b a & = \log_b a \cdot \log_b c \hspace{1.7cm} 
    \text{(because \(\times\) is commutative)} \\
    \log_b (a^{\log_b c}) & = \log_b(c^{log_b a}) \hspace{2.05cm} 
    (\text{because} y\log_b x = log_b x^y)\\
    a^{\log_b c} & = c^{\log_b a} \hspace{3cm}
    (\log_b \text{is injective:} \log_b y = log_b x \implies y = x)
\end{align*}

\subsection{Asymptotic notations}
\paragraph{Big \(O\) Notation}
We say \(f(n) = O(g(n))\) if there exists a positive constants \(c, N\)
such that
\[0 \leq f(n) \leq c g(n) \quad \forall n \geq N.\]

We may refer to \(g(n)\) to be the asymptotic upper bound for \(f(n)\).

\paragraph{Big Omega Notation}
We say \(f(n) = \Omega(g(n))\) if there exists positive constants \(c, N\)
such that 
\[0 \leq c g(n) \leq f(n) \quad \forall n \geq N.\]

Then, \(g(n)\) is said to be an asymptotic lower bound for \(f(n)\).
It is useful to say that a problem is at least \(\Omega(g(n))\).

\paragraph{Big Theta Notation} 
We say \(f(n) = \Theta(g(n))\) if and only if 
\[f(n) = O(g(n)) \text{ and } f(n)=\Omega(g(n)).\]
That is, both \(f\) and \(g\) have the same asymptotic growth.
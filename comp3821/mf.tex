
\section{Maximum Flow}

\subsection{Flow Networks}
\paragraph{Flow Network} A \textit{flow network} 
\( G = (V, E) \) is a directed graph where each edge \( e = (u, v) \in  E \) has a 
positive integer capacity \( c(u, v) > 0 \).

There are two distinguished vertices. A source \( s \) and sink  \( t \).
There are no outgoing edges for a sink and likewise, no incoming edges
for a source. \\

\paragraph{Flow} 
A \textit{flow} in \( G \) is a function  \( f: E \to  \mathbb R^+ \), 
\(  f(u, v) \geq 0 \). which satisfies
\begin{enumerate}
  \item \textbf{Capacity Constraints:} for all edges \(e(u,v) \in E\) we require 
  \(f(u,v) \leq c(u, v)\).
  \item \textbf{Flow Conservation:} For all \( v \in V \setminus \{(s, t)\} \),
    we require 
    \[
        \sum_{(u, v) \in E} f(u, v) = \sum_{(v, w) \in  E} f(v, w)
    .\]
    That is, the incoming flow must be equal to the outgoing flow.
\end{enumerate}

\paragraph{Value of flow} 
The \textit{value of the flow} is defined as 
\[|f| = \sum_{v:(s,v) \in E} f(s,v) = \sum_{v:(v,t)\in E} f(v,t).\]

\paragraph{Residual Flow Network} 
The \textit{residual flow network} for a flow network with some flow in it: the network 
with the leftover capacities.

\paragraph{Augmenting Path} 
Residual flow networks can be used to increase the total flow through the network
by adding an \textit{augmenting path}.

The capacity of an augmenting path is the capacity of its ''bottleneck"
edge, i.e., the capacity of the smallest capacity edge on that path.

We should then send that amount of flow along the augmenting path, recalculating
the flow and the residual capacities for each edge used.

\subsection{Ford-Fulkerson Algorithm}
\textbf{Ford-Fulkerson algorithm for finding maximal flow in a flow network:}
\begin{itemize}
    \item Keep adding flow through new augmenting paths for as long as it is possible.
    \item When there are no more augmenting paths, you have achieved the largest 
    possible flow in the network.
\end{itemize}

\paragraph{Cut}
A \textit{cut} in a flow network is any partition of the vertices of the underlying 
graph into two subsets \(S\) and \(T\) such that:
\begin{enumerate}
    \item \(S \cup T = V\)
    \item \(S \cap T = \emptyset\)
    \item \(s \in S \text{ and } t \in T\).
\end{enumerate}

\paragraph{Capacity of a Cut}
The \textit{capacity} \(c(S,T)\) \textit{of a cut} \((S, T)\) is the sum of capacities
of all edges leaving \(S\) and entering \(T\), i.e.
\[c(S,T) = \sum_{(u,v) \in E} \{c(u,v):u\in S \text{ \& } v \in T\}\]
Note that the capacities of edges going in the opposite direction, i.e., from \(T\)
to \(S\) do not count.

\paragraph{Flow of Cut}
The \textit{flow through a cut} \(f(S,T)\) is the total flow through edges from
\(S to T\) minus the total flow through edges from \(T\) to \(S\):
\[
    f(S,T) = \sum_{u,v} \in E \{f(u,v): u \in S \text{ \& } v \in T\}
    - \sum_{(u,v) \in E} \{f(u,v): u \in T \text{ \& } v \in S\}
\]
Clearly, \(f(S,T) \leq c(S,T)\) because for every edge \((u, v) \in E\) we assumed
\(f(u,v) \leq c(u,v)\) and \(f(u,v) \geq 0\).

\paragraph{Max Flow Min Cut Theorem}
The maximal amount of flow in a flow network is equal to the capacity of the
cut of minimal capacity.

\paragraph{Edmonds-Karp Algorithm}
The Edmonds-Karp algorithm improves the Ford-Fulkerson algorithm in a simple way:
always choose the shortest path from source \(s\) to the sink \(t\), where
the ``shortest path" means the fewest number of edges, regardless of their capacities
(i.e., each edge has the same unit weight). This algorithm runs in time \(O(|V||E|^2)\).

The fastest max flow algorithm to date, an extension of the \texttt{PREFLOW-PUSH}
algorithm runs in time \(|V|^3\).

\subsection{Solving Different Problems with Maximum Flow}
\subsubsection{Networks with Multiple Sources and Sinks}
Flow networks with multiple sources and sinks are reducible to networks with
a single source and single sink by adding a ``super-sink" and ``super-source" and
connecting them to all sources and sinks, respectively, by edges of infinite capacities.

\subsubsection{Maximum Matching In Bipartite Graphs}
We will consider bipartite graphs; i.e., graphs whose vertices can be split into two
subsets, \(L\) and \(R\) such that every edge \(e \in E\) has one end in the set \(L\)
and the other in the set \(R\).

\paragraph{Matching} 
A \textit{matching} in a graph \(G\) is a subset \(M\) of all edges \(E\) such that each 
vertex of the graph belongs to at most one of the edges in the matching \(M\).

\paragraph{Maximum Matching} 
A \textit{maximum matching} in a bipartite graph \(G\) is a matching containing the 
largest possible number of edges.

We turn a Maximum Matching problem into a Max Flow problem by adding a super
source and a super sink, and by giving all edges a capacity of 1.

Note how the residual flow networks allow rerouting the flow in order to increase
the total throughput.

\subsubsection{Max Flow with Vertex Capacities}
Sometimes not only the edges but also the vertices \(v_i\) of the flow graph might
have capacities \(C(v_i)\), which limit the total throughput of the flow coming to
the vert (and, consequently, also leaving the vertex):
\[\sum_{e(u,v) \in E} f(u,v) = \sum_{e(v,w) \in E} f(u,w) \leq C(v).\]
Such a case is reduced to the case where only edges have capacities by splitting
each vertex \(v\) with limited capacity \(C(v)\) into two vertices \(v_{in}\)
and \(v_{out}\) so that all edges coming into \(v\) go into \(v_{in}\), all edges
leaving \(v\) now leave \(v_{out}\) and by connecting the new vertices \(v_{in}\) and
\(v_{out}\) with an edge \(e^* = (v_{in}, v_{put})\) with capacity equal to the capacity
of the original vertex \(v\).

\subsection{Applications of Max Flow Algorithm}
\subsubsection{Allocation: Movie Rental}
 \paragraph{Problem}
 Assume you have a movie rental agency. At the moment you have \(k\) movies in stock,
 with \(m_i\) copies of the movie \(i\). Each of \(n\) customers can rent out at most 5
 movies at a time. The customers have sent you their preferences which are a list
 of movies they would like to see. Your goal is to dispatch the largest possible
 number of movies.

 \subsubsection{Multiple Sources and Sinks: Cargo Allocation}
 \paragraph{Problem} The storage space of a ship is in the form of a rectangular grid of 
 cells with \(n\)  rows and \(m\) columns. Some of the cells are taken by support pillars 
 and cannot be used for storage, so they have 0 capacity. You are given the capacity of
 every cell; cell in row \(r_i\) and column \(c_j\) has capacity \(C(i,j)\). To ensure
 the stability of the ship, the total weight in each row \(r_i\) must not exceed 
 \(C(r_i)\) and the total weight in each column \(c_j\) must not exceed \(C(c_j)\).
 Find how to allocate the cargo weight to each cell to maximise to total load without
 exceeding the limits per column, limits per row and limits per available cell.

 \subsubsection{Vertex Capacities: Disjoint Paths}
 \paragraph{Problem}
 You are given a connected, directed graph \(G\) with \(N\) vertices. Out of these \(N\)
 vertices \(k\) are painted red, \(m\) are painted blue, and the remaining \(N-k-m >0\)
 of the vertices are black. Red vertices have only outgoing edges and blue vertices have 
 only incoming edges. Your task is to determine the largest possible number of disjoint
 (i.e., non-intersecting) paths in this graph, each of which starts at a red vertex
 and finishes at a blue vertex.
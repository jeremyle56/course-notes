
\section{Reductions}

\subsection{Decision Problems}
A \textit{decision problem} is a problem with a YES/NO answer.

\paragraph{Certificates and Counter-Example}
For a given problem P and a given instance \(x\),
\begin{itemize}
    \item a certificate for \(x\) is data that lets us verify easily that \(P(x)=\)YES.
    \item a counter-example for \(x\) is data that lets us verify easily that \(P(x)=\)NO.
\end{itemize}

\paragraph{Polynomial Time Algorithms}
A decision problem \(A\) is in polynomial time if there exists a polynomial time algorithm that solves it. An algorithm runs in polynomial time for every input if it terminates in polynomially many steps in the length of the input (i.e. \(T(n) = O(n^k)\) where \(k\) is a natural number and \(n\) is the size of the input). We denote this by \(A\in \textbf{P}\).

\paragraph{Input}
The length of an input is the number of symbols needed to describe the input precisely.

\paragraph{Reductions}
Decision problem \(U\) is reducible to dec. prob. \(V\) if there is a function \(f\) such that
\begin{enumerate}
    \item \(f\) maps instances of \(U\) into instances of \(V\);
    \item For every instance \(x\) of \(U\), \(U(x)\) is true iff \(V(f(x))\) is true.
\end{enumerate}
If \(f\) is commutable in polynomial time then \(U\) is polynomially reducible to \(V\).

\paragraph{Polynomial Reduction from SAT to 3SAT}
Every instance of SAT (Boolean SATisfiability Problem) is polynomially reducible to an instance of 3SAT. (See Video/Slide)

\subsection{Linear Programming}

\paragraph{Variables} 
\[x_j \text{ for } 1 \leq j \leq n\]
\paragraph{Objective}
\[\text{ maximise or minimise } \sum_{j=1}^n c_j x_j\]
\paragraph{Constraints}
\[\sum_{j=1}^n (a_{ij}x_j)\mathcal{R}_ib_i, \qquad \text{for } 1\leq i \leq m \text{ with } \mathcal{R}_i \in \{\leq, =, \geq \}\]
A feasible solution is a variable assignment satisfying all constraints. An optimal solution is a feasible solution satisfying the objective.

\paragraph{Canonical Form}
\begin{itemize}
    \item Objective: maximise \(\sum_{j=1}^n c_jx_j\)
    \item Constraints: \(\sum_{j=1}^n a_{ij}x_j \leq b_i,\) for \(1 \leq i \leq m\) and \(x_j \geq 0\) for \(1 \leq j \leq n\).
\end{itemize}

\paragraph{Matrix Form}
To specify a linear programming problem we can simply provide a triplet \((A,\textbf{b},\textbf{c})\) where \(A\) is a matrix and  \(\textbf{b},\textbf{c}\) are column vectors (see slide).

\paragraph{Standard Form}
\begin{itemize}
    \item maximise \(z = \textbf{c}^T\textbf{x}\)
    \item subject to the constraints \(A\textbf{x}+I\textbf{s}=\textbf{b}\) and \(\textbf{x} \geq 0\) and \(\textbf{s} \geq 0\).
\end{itemize}
\(\textbf{s}\) are the slack and surplus variables that are used to transform constraints using inequalities into equality constraints.

\paragraph{Transformations}
Any LP can be transformed into a canonical form or into standard form if needed. In general a Linear Program does not necessarily produce the non-negativity constraints for all variables. However, in the standard form such constraints are required for all of the variables. This is not a problem because each occurrence of an unconstrained variable \(x_j\) can be replaced by the expression \(x_j'-x_j*\) where \(x_j', x_j*\) are new variables satisfying the constraints \(x_j', x_j* \geq 0\). Similarly constraints of the form \(|A\textbf{x}| \leq \textbf{b}\) can be replaced to two linear constraints: \(A\textbf{x} \leq \textbf{b} , -A\textbf{x} \leq \textbf{b}\).

\paragraph{Algorithms}
\begin{itemize}
    \item Simplex (1947): Exponential runtime in the worst case, very efficient in practice 
    \item Ellipsoid Method (1979): Polynomial Algorithm \(O(n^6L) (n\) variables, input of size \(L)\)
    \item Interior Point algorithms: Worst-case \(O(n^{3.5}L^2\lg L \lg \lg L)\), farily efficient in practice 
\end{itemize}

\paragraph{Variants}
\begin{itemize}
    \item Integer Linear Programs (ILP) (NP-complete)
    \item Mixed Integer Linear Programs (continuous and integer variables)
    \item 0-1 Linear Programming (variables are \(\in \{0, 1\}\))
\end{itemize}

\subsubsection{Decide Diet}
\paragraph{Setting}
You are given a list of food sources \(f_1,f_2,\dots f_n\) for each source \(f_i\) you are given:
\begin{itemize}
    \item its price per gram \(p_i\)
    \item the number of calories \(c_i\) per gram and
    \item for each of 13 vitamins \(V_1, V_2, \dots, V_{13}\) you are given the content \(v(i,j)\) of milligrams of vitamin \(V_j\) in one gram of food source \(f_i\).
\end{itemize}
For each vitamin \(V_j\), you are given the recommended daily intake of \(w_j\) milligrams.

\paragraph{Task} 
Find a combination of quantities of food sources such that:
\begin{itemize}
    \item the total number of calories in all of the chosen food is equal to a recommended daily value of 2000 calories
    \item the total intake of each vitamin \(V_j\) is at least the daily intake of \(w_j\) milligrams for all \(1 \leq j \leq 13\)
    \item the price of all food per day is as low as possible.
\end{itemize}

\paragraph{Solution}
To obtain the corresponding constraints let us assume that we take \(x_i\) grams of each food source \(f_i\) for \(1 \leq i \leq n\). Then:
\begin{itemize}
    \item the total number of calories must satisfy \(\sum_{i=1}^n x_ic_i = 2000;\)
    \item for each vitamin \(V_j\) the total amount in all food must satisfty
    \[\sum_{i=1}^n x_iv(i,j) \geq w_j \quad (1\leq j \leq 13);\]
    \item an implicy assumption is that all quantities must be non-negative \(x_i \geq 0, 1 \leq i \leq n.\)
\end{itemize}
Our goal is to minimise the objective function which is the total cost \(y = \sum_{i=1}^n x_ip_i\). Note that here all the equalities and inequalities, as well as the objective function are linear. 

\subsubsection{Infrastructure Politics - Integer Programming}
\paragraph{Setting} 
You are the (Shadow?) Treasurer and you want to make certain promises to the electorate that will ensure that your party will win in the forthcoming elections. You promise that you will build 
\begin{itemize}
    \item a certain number of bridges, each 3 billion a piece. Each bridge you promise brings you 5\% of city votes, 7\% of suburban votes and 9\% of rural votes.
    \item a certain number of rural airports, each 2 billion a piece. Each rural airport you promise brings you no city votes, 2\% of suburban votes and 15\% of rural votes.
    \item a certain number of olympic swimming pools each a billion a piece. Each olympic swimming pool promised brings you 12\% of city votes, 3\% of suburban votes and no rural votes.
\end{itemize}

\paragraph{Task} 
In order to win, you have to get at least 51\% of each of the city, suburban and rural votes. Win the election by cleverly making a promise that appears to blow as small hole in the budget as possible. 

\paragraph{Solution}
Let the number of bridges, airports and swimming pools to be \(x_b, x_a, x_p\) respectively.
The problem amounts to minimising the objective \(y = 3x_b + 2x_a + x_p\), while making sure that the following constraints are satisfied. 
\begin{align*}
    0.05 x_b + 0.12x_p & \geq 0.51 \tag{securing majority of city votes} \\
    0.07 x_b + 0.02x_a + 0.03x_p & \geq 0.51 \tag{securing majority of suburban votes} \\ 
    0.09 x_b + 0.15x_a & \geq 0.51 \tag{securing majority of rural votes} \\
    x_b, x_a, x_p & \geq 0. 
\end{align*}
This is an example of Integer Linear Programming which is much harder than the ''plain'' Linear Programming and is in fact NP hard!

\subsection{NP Completeness}
A decision problem \(A(x)\) is in non-deterministic polynomial time, denotes by \(A\in\textbf{NP}\), if:
\begin{enumerate}
    \item there exists a problem \(B(x,y)\) such that for every input \(x, A(x)\) is true just in case there exists \(y\) such that \(B(x,y)\) is true; and
    \item such that the truth of \(B(x,y)\) can be verified by an algorithm running in polynomial time in the length of \(x\) only.
\end{enumerate}
We call \(y\) a certificate of \(x\).

\paragraph{Examples}
COMPOSITE, Vertex Cover, SAT

\paragraph{NP-hardness}
A problem is NP-hard if any problem in NP is reducible to it. I.e., a problem \(P\) is NP-hard if for any other problem \(P'\) that is in the class NP, there exists a polynomial reduction \(f_{P'}\) from \(P'\) to \(P\). A problem is NP-complete if it is both NP-hard and in the class NP. 
\paragraph{Proving NP completeness} 
Sometimes the distinction between a problem in P and an NP complete problem can be subtle!

\begin{tabularx}{0.9\textwidth} { 
  | >{\raggedright\arraybackslash}X 
  | >{\raggedright\arraybackslash}X | }
 \hline
 \textbf{in P} & \textbf{NP complete} \\
 \hline
Given a graph \(G\) and two vertices \(s\) and \(t\), is there a path from \(s\) to \(t\) of length \textbf{at most} \(K\)?  & Given a graph \(G\) and two vertices \(s\) and \(t\), is there a simple path from \(s\) to \(t\) of length \textbf{at least} \(K\)?  \\
\hline
Given a propositional formula in CNF form such that every clause has at most \textbf{two} propositional variables, does the formula have a satisfying assignment? & Given a propositional formula in CNF form such that every clause has at most \textbf{three} propositional variables, does the formula have a satisfying assignment? \\
\hline 
Given a graph \(G\), does \(G\) have a tour where every \textbf{edge} is traversed exactly once? (An Euler tour.) & Given a graph \(G\), does \(G\) have a tour where every \textbf{vertex} is visited exactly once? (A Hamiltonian cycle.) \\
\hline
\end{tabularx}

\paragraph{Theorem} 
Let \(U\) be an NP-hard problem and let \(V\) be another decision problem. If \(U\) is polynomially reducible to \(V\) then \(V\) is also NP-hard.

\section{Field Extensions}

\begin{definition}[Field Extensions]
    If \(F\) is a subfield of \(E\). We say \(E\) is an extension of \(F\), or we say that \(E / F\) is a field extension.
\end{definition}

\begin{definition}[Generators of Field Extensions]
    Let \(E / F\) be a field extension, and let \(\alpha_1, \dots, \alpha_n \in E\). Denote \(F(\alpha_1, \dots, \alpha_n)\) the subfield of \(E\) generated by \(F, \alpha_1, \dots, \alpha_n\). This is called the subfield generated by \(\alpha_1, \dots, \alpha_n\) over \(F\). If \(E\) is of the form \(E = F(\alpha_1, \dots, \alpha_n)\), we say that \(E / F\) is a finitely generated extension.
\end{definition}

\exmp{
    \(\bQ(i) \subseteq \bC\), \(\bQ(i) = \{ a + ib: a, b \in \bQ \} = \bQ[i]\). Also, \(\bQ(\pi) \subseteq \bR\), \(\bQ(\pi) = \left\{\frac{f(\pi)}{g(\pi)} : fg \in \bQ[x], g \neq 0 \right\} \neq \bQ[x]\).
}

Let \(E / F\) be a field extension and \(\alpha \in E^\times\). Recall the evaluation homomorphism, \(\epsilon: F[x] \to E; p \mapsto p(\alpha)\) and \(\Im \epsilon = F[\alpha] \subseteq E\).

\begin{thrm-defn}[Transcendental and Algebraic]
    There are two possibilities:
    \begin{enumerate}
        \item \(\ker \epsilon = 0\). (\(\epsilon\) is injective). i.e. \(\alpha\) is not a root of any nonzero polynomial in \(F[x]\). We say that \(\alpha\) is transcendental over \(F\). Hence, \(F[\alpha] \cong F[x]\).
        \item \(\ker \epsilon \neq 0 = \langle p \rangle\) where \(p\) is monic of minimal degree. Then \(F[\alpha] \cong F[x] / \langle p \rangle\). We say that \(\alpha\) is algebraic over \(F\) and \(p(x)\) is called the minimal polynomial of \(\alpha\) over \(F\). We say that \(E / F\) is algebraic if every \(\alpha \in E\) is algebraic over \(F\).
    \end{enumerate}
\end{thrm-defn}

\exmp{
    \begin{enumerate}
        \item \(\sqrt{2} = 1.414 \dots \in \bR\). Minimal polynomial of \(\sqrt{2}\):
            \begin{itemize}
                \item over \(\bQ: x^2 - 2\)
                \item over \(\bR: x - \sqrt{2}\)
            \end{itemize}
        \item In \(\bR(x) / \bR\), the element \(x\)is transcendental over \(\bR\). \(\epsilon: \bR[x] \to \bR(t); x \mapsto t\).
        \item \(\bR / R\) is algebraic. Let \(z = a + ib \in \bC\). \((z - a)^2 + b^2 = 0\) then \(p(x) = (x - a)^2b^2 = x^2 - 2ax + (a^2 + b^2) \in R[x]\), \(p(z) = 0\).
    \end{enumerate}
}

\begin{proposition}
    If \(\alpha \in E\) is algebraic over \(F\), then its minimal polynomial in \(F[x]\) is irreducible. 
    % \proof{
    %     Let \(p\) be a minimal polynomial. Suppose \(p(x) = f(x)g(x)\) for \(f, g \in F[x]\). Substitute \(x = \alpha: 0 = f(\alpha)g(\alpha)\). If \(E\) is a field then either \(f(\alpha) = 0\) or \(g(\alpha) = 0\). If \(f(\alpha) = 0\) then \(\deg f \geq \deg p\) but then \(f = p\)
    % }
\end{proposition}

\begin{proposition}
    Let \(F(\alpha)\) be a simple extension.
    \begin{enumerate}
        \item If \(\alpha\) is transcendental over \(F\), then \(F(\alpha) \cong F(x)\) (field of rational functions in 1 variable)
        \item If \(\alpha\) is algebraic over \(F\), then \(F(\alpha) = F[\alpha] \cong F[x] / \langle p \rangle\) where \(p\) is the minimal polynomial.
    \end{enumerate}
    \proof{
        \begin{enumerate}
            \item Know \(F[\alpha] \cong F[x]\), take fraction fields gives \(F(\alpha) \cong K(F[x]) \cong F(x)\).
            \item Know \(F[\alpha] \cong F[x] / \langle p \rangle\). \(\langle p \rangle\) is maximal because \(p\) is irreducible hence \(F[x] / \langle p \rangle\) is a field. Therefore since \(F[\alpha]\) is already a field, so \(F(\alpha) = F[\alpha]\).
        \end{enumerate}
    }
    \exmp{
        \begin{itemize}
            \item \(\bQ(i) = \bQ[i] \cong \bQ[x] / \langle x^2 + 1 \rangle\)
            \item Let \(f(x) = x^3 + x^2 - 1 \in \bQ[x]\) which is irreducible. Let \(\alpha\) be a root of \(f\). Consider \(\bQ[\alpha] = \{ r + s\alpha + t\alpha^2: r, s, t \in \bQ\}\). E.g. try \(\beta = \alpha^2 + 1 \in \bQ[\alpha]\). Apply Euclidean algorithm to \(f(x) \tand g(x) = x^2 + 1\) which gives \(\frac{1}{5}(x - 2)f(x) + \frac{1}{5}(-x^2 + x + 3)g(x) = 1\) in \(\bQ[x]\). Substituting \(x = \alpha\): \(0 + \frac{1}{5}(-\alpha^2 + \alpha + 3) \beta = 1\). So \(\beta^{-1} = \frac{1}{5}(-\alpha^2 + \alpha + 3) \in \bQ[\alpha]\). This kind of calculation shows that \(\bQ(\alpha) = \bQ[\alpha]\). i.e. \(\bQ[\alpha]\) is a field.
        \end{itemize}
    }
\end{proposition}

\begin{definition}[Degree]
    Let \(E / F\) be a field extension. Then \(E\) is a vector space over \(F\). The degree of \(E / F\) is \([E : F] = \dim_F E\). We say \(E / F\) is a finite extension if \([E : F] < \infty\).
    \exmp{
        \([\bC : \bR] = 2, [\bR : \bQ] = \text{ uncountable } \infty\).
    }
\end{definition}

\begin{proposition}
    Any finite extension is algebraic.
    \proof{
        Let \(E / F\) be finite, say \(\dim n \geq 1\). Let \(\alpha \in E\). Then \(1, \alpha, \alpha^2, \dots, \alpha^n\) must be linearly dependent over \(F\). i.e. there exists \(c_0, \dots, c_n \in F\) not all 0 such that \(c_0 + c_1 \alpha + \dots + c_n \alpha^n = 0\). i.e. \(p(\alpha) = 0\) where \(p(x) = c_0 + c_1 x + \dots + c_n x^n \in F[x]\). So \(\alpha\) is algebric over \(F\).
    }
\end{proposition}

\begin{theorem}[The Tower Law]
    Let \(K / E\) and \(E / F\) be finite. Then \(K / F\) is finite and \([K : F] = [K : E][E : F]\).
\end{theorem}

\begin{proposition}
    Suppose \(\alpha \in E\) is algebraic over \(F\). Then \([F(\alpha) : F] = \deg p\) where \(p\) is a minimal polynomial of \(\alpha\) over \(F\).
    \exmp{
        \(\bQ \subseteq \bQ(\sqrt{2}) \subseteq \bQ(2^{1/4})\). What is \([\bQ(2^{1/4}) : \bQ]\)?
        \begin{itemize}
            \item \([\bQ(\sqrt{2}) : \bQ] = 2\) because minimal polynomial of \(\sqrt{2} / \bQ\) is \(x^2 - 2\) has degree 2.
            \item \([\bQ(2^{1/4}) : \bQ(\sqrt{2})] = 2\) because minimal polynomial of \(2^{1/4}\) over \(\bQ(\sqrt{2})\) is \(x^2 - \sqrt{2}\).
        \end{itemize}
        Then by the tower law, \([\bQ(2^{1/4}) : \bQ] = [\bQ(2^{1/4}) : \bQ(\sqrt{2})][\bQ(\sqrt{2}) : \bQ] = 2 \cdot 2 = 4\).
    }
\end{proposition}

\begin{theorem}[Eisenstein's Criterion]
    Let \(R\) be a UFD, \(K = K(R)\). Let \(f = f_0 + f_1 x + \dots + f_n x^n \in R[x]\). Suppose there exists a prime \(p \in R\) such that \(p \mid f_0, \dots, p \mid f_{n - 1}\) but \(p \nmid f_n\) and \(p^2 \nmid f_0\). Then \(f\) is irreducible in \(K[x]\).
\end{theorem}

\begin{theorem}[Splitting Fields]
    Let \(F\) be a field, \(f \in F[x], f \neq 0\). Then there exists a field extension \(E / F\) such that \(f(x)\) is a product of linear factors in \(E[x]\), i.e. \(f(x) = c(x - \alpha_1)\cdots(x - \alpha_n)\) for \(\alpha_1, \dots, \alpha_n \in E\). The subfield \(F(\alpha_1, \dots, \alpha_n)\) generated by \(F\) and the \(\alpha\)'s is called a splitting field for \(f(x)\) over \(F\).
    \proof{
        Induction on \(n = \deg f\). For \(n = 1\), just take \(E = F\). Suppose \(n > 1\), let \(p \in F[x]\) be an irreducible factor of \(f\). Let \(K = F[x] / \langle p \rangle\). Then \(K\) is a field (since \(p\) is irreducible), \(K\) contains a root of \(p\) namely \(\alpha = x + \langle p \rangle \in K\). Also \(F\) is a subfield of \(K\). In \(K[x]\) we have \(f(x) = (x - \alpha)g(x)\) for \(g \in K[x], \deg g < \deg f\). By induction, there is an extension \(E\) of \(K\) such that \(g\) factors into linear factors in \(E[x]\). So does \(f\).
    }
    \exmp{
        Splitting field of \(x^3 - 2\) over \(\bQ\).
        \begin{quote}
            We already know in \(\bC\): \(x^3 - 2 = (x - 2^{1/3})(x - 2^{1/3}\omega)(x - 2^{1/3}\omega^2)\) where \(\omega = e^{2\pi i / 3}\) so splitting field is \(\bQ(2^{1/3}, \omega)\).
        \end{quote}
        \(x^3 - 2\) is irreducible in \(\bQ[x]\) by Eisenstein's Criterion. Let \(K = \bQ[x] / \langle x^3 - 2 \rangle\) and \(\alpha = x + \langle x^3 - 2 \rangle \in K\). So \(\alpha^3 = (x + \langle x^3 - 2 \rangle)^3 = x^3 + \langle x^3 - 2 \rangle = x^3 - 2 + 2 + \langle x^3 - 2 \rangle = 2 + \langle x^3 - 2 \rangle = 2\). Then \(x^3 - 2 = (x - \alpha)(x^2 + \alpha x + \alpha^2)\) in \(K[x]\).
        \begin{quote}
            {\bf Q:} is \(x^2 + \alpha x + \alpha^2\) irreducible in \(K[x]\).

            {\bf Proof.} Suppose not. Say \(\beta\) is a root in \(K\). i.e. \(\beta^2 +\alpha\beta + \alpha^2 = 0\). Let \(\omega = \beta / \alpha\). Then \(\omega^2 + \omega + 1 = 0\), but \(x^2 + x + 1\) is irreducible over \(\bQ\). Thus \([\bQ(\omega) : \bQ] = 2\) but \(\omega \in K\) and \([K : \bQ] = 3 (= \deg (x^3 - 2))\) but this is a contradiction by the Tower Law, \([K : \bQ] = [K : \bQ(\omega)][\bQ(\omega): \bQ]\).
        \end{quote}
        Now define \(E = K[x] / \langle x^2 + \alpha x + \alpha^2 \rangle\), then \(E\) is a field. Let \(\beta = x + \langle x^2 + \alpha x + \alpha^2 \rangle\). so \(\beta \in E\) is a root of \(x^2 + \alpha x + \alpha^2\) get \(x^ - 3 = (x - \alpha)(x - \beta)(x - \alpha^2 / \beta) = (x - \alpha)(x - \omega)(x - \omega^2 \alpha)\) with \(\omega = \beta / \alpha\).
    }
\end{theorem}

\begin{prop-defn}[Algebraically Closed]
    A field \(F\) is algebraically closed if one of the following equivalent conditions hold:
    \begin{enumerate}
        \item Any non-constant \(p \in F[x]\) has a root in \(F\).
        \item There are no non-trivial algebraic extensions of \(F\).
    \end{enumerate}
\end{prop-defn}

\begin{theorem}
    Let \(F\) be a field. There exists a ``smallest'' extension \(\tilde{F} / F\) which is algebraically closed, called the algebraic closure of \(F\). It is unique up to isomorphism.
\end{theorem}
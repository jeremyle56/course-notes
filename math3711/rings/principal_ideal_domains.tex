\section{Principal Ideal Domains}

\begin{definition}[Principal Ideal Domain]
    Let \(R\) be a commutative ring. An ideal \(I\) is principal if \(I = \langle r \rangle, r \in R\) (generated by a single element). A principal ideal domain (PID) is a domain where every ideal is principal.
    \exmp{
        \(\bZ\) is a PID, every ideal of is of the form \(n\bZ\).
    }
\end{definition}

\begin{proposition}
    Let \(R\) be a PID. Let \(p \in R, p \neq 0\), then \(p\) is irreducible if and only if \(\langle p \rangle\) is maximal.
    \proof{
        \((\impliedby)\) Assume \(p\) is not irreducible, so \(p = rs\). Neither \(r, s\) are units. Then \(\langle p \rangle = \langle rs \rangle \subsetneq \langle r \rangle\) so \(\langle p \rangle\) is not maximal. (\textit{Alternatively}: \(\langle p \rangle\) maximal \(\implies \langle p \rangle\) prime \(\implies p\) prime \(\implies p\) irreducible.)\\

        \((\implies)\) Suppose \(\langle p \rangle \subseteq I\). Since \(R\) is a PID, \(I = \langle q \rangle\) for some \(q\) hence \(q \mid p\). Since \(p\) irreducible, either \(q = up (u \in R^*) \implies I = \langle q \rangle = \langle p \rangle\) or \(q\) is a unit so \(I = \langle q \rangle = R\).
    }
\end{proposition}

\begin{corollary}
    In a PID, irreducibles are prime.
    \proof{
        \(p\) ideal \(\implies \langle p \rangle\) maximal \(\implies R / \langle p \rangle\) is a field \(\implies R / \langle p \rangle\) is a domain \(\implies \langle p \rangle\) prime \(\implies p\) is prime.
    }
\end{corollary}

Note, in a PID factorisations are unique if they exist.

\begin{lemma}
    Let \(S\) be a ring. Let \(I_0, I_1, I_2, \dots\) are ideals of \(S\) such that \(I_0 \subseteq I_1 \subseteq I_2 \subseteq \cdots\). Then \(\bigcup_{i \geq 0} I_i\) is an ideal of \(S\).
    \proof{
        Suppose \(x, y \in \cup_{i \geq 0} I_i\) then \(x \in I_n\) and \(y \in I_m\), so \(x, y \in I_k\) where \(k = \max(n, m)\) therefore \(x + y \in T_k \subseteq \cup_{i \geq 0} T_i\). Then prove other ideal properties.
    }
\end{lemma}

\begin{theorem}
    Any PID is a UFD.
    \proof{
        We need to prove that any \(r_0 \in R\), not has a factorisation into ideals. Suppose \(r_0 \in R\), not a unit is not a product of irreducibles. In particular \(r\) itself is not irreducible, so \(r = r_1q_1\) where \(r_1, q_1\) not units. At least one of \(r_1, q_1\) is not a product of irreducibles. Repeat this argument for \(r_1 = r_2q_2\) where without loss of generality, \(r_2\) is not a product of irreducibles. Then we have \(r_0, r_1, r_2\) so \(r_1 \mid r_0, r_2 \mid r_1\) etc.. Then \(\langle r_0 \rangle \subseteq \langle r_1 \rangle \subseteq \langle r_2 \rangle \subseteq \dots\). \\

        Let \(I = \cup_{i \geq 0} \langle r_i \rangle\). By the previous Lemma, \(I\) is an ideal. Since \(R\) is a PID, \(I = \langle s \rangle\), \(s \in R\). So \(s \in \langle r_n \rangle\) for some \(n\), \(I \subseteq \langle r_n \rangle \subseteq \langle r_{n + 1} \rangle \subseteq \dots \subseteq I\). So in fact, \(I = \langle r_n \rangle  = \langle r_{n + 1} \rangle = \dots\) but this contradicts \(\langle r_n \rangle \subsetneq \langle r_{n + 1} \rangle\) because \(r_n = r_{n + 1}q_{n + 1}\) where \(q_{n + 1}\) is not a unit.
    }
\end{theorem}

\begin{definition}[Greatest Common Divisor]
    Let \(R\) be a PID (works for UFD). Let \(r,s \in R, r,s \neq 0\). Then a greatest common divisor (\(\gcd\)) of \(r\) and \(s\) is an element \(d \in R\) such that \(d \mid r, d \mid s\) and if \(c \in R\) is any element such that \(c \mid r, c \mid s,\) then \(c \mid d\). Write \(d = \gcd(r, s)\). \(d\) is defined only up to units.
\end{definition}

Any 2 \(\gcd\)'s divide each other so are associates.

\begin{proposition}
    In a PID, \(r, s \in R - \{0\}\) then \(r, s\) have a gcd \(d\) such that \(\langle d \rangle = \langle r, s \rangle\).
    \proof{
        Given \(r, s\). Consider \(\langle r, s \rangle = \{ ar + bs: a, b \in R \}\). Since \(R\) is PID, \(\langle r, s \rangle = \langle d \rangle\) for some \(d \in R\). \(d \mid r\) is clear since \(r \in \langle d \rangle\). Similarly \(d \mid s\). Now suppose \(c \mid r\) and \(c \mid s\). Then \(r, s \in \langle c \rangle \implies \langle r, s \rangle \subseteq \langle c \rangle \implies \langle d \rangle \subseteq \langle c \rangle \implies c \mid d\).
    }
\end{proposition}
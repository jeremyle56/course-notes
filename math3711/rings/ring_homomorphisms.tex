\section{Ring Homomorphisms}

\begin{prop-defn}[Homomorphism]
Let \(R, S\) be rings. A ring homomorphism is a group homomorphism \(\phi: R \to S\) such that:
\begin{enumerate}
    \item \(\phi(1_R) = 1_S\)
    \item \(\phi(rr') = \phi(r)\phi(r') \forall r, r' \in R\).
\end{enumerate}
\end{prop-defn}

\begin{definition}[Isomorphism]
    A ring isomorphism is a bijective ring homomorphism \(\phi: R \to S\). In this case \(\phi^{-1}\) is also a ring homomorphism. We write \(R \cong S\) as rings.
\end{definition}

\begin{proposition}
    Let \(\phi: R \to S\) be a ring homomorphism.
    \begin{enumerate}
        \item If \(R'\) is a subring of \(R\) then \(\phi(R')\) is a subring of \(S\).
        \item If \(S'\) is a subring of \(S\) then \(\phi^{-1}(S')\) is a subring of \(R\).
        \item If \(I \trianglelefteq S\) then \(\phi^{-1}(I) \trianglelefteq R\)
    \end{enumerate}
\end{proposition}

\begin{corollary}
    In particular, \(\Im \phi = \phi(R)\) is a subring of S and \(\ker \phi = \phi^{-1}(0) \trianglelefteq R\).
\end{corollary}

\begin{theorem}
    Let \(R \) = ring, \(I = \) ideal with \(\pi: R \to R / I\) be a quotient map. Suppose \(\phi: R \to S\) is a ring homomorphism such that \(I \subseteq \ker \phi\). Recall group situation gives a map \(\psi: R / I \to S\) then \(\psi\) is also a ring homomorphism. Special case for \(I = \ker \phi\): \(R / \ker\phi \cong \Im \phi\) (as rings).
\end{theorem}

\begin{proposition}
    Let \(J \trianglelefteq R\) and let \(\pi: R \to R / J\) be quotient map. Then there is a 1-1 correspondence:
    \[\{I \trianglelefteq R \text{ such that } J \subseteq I \} \leftrightarrow	\{\text{ideals } \bar{I} \trianglelefteq R / J\}\]
\end{proposition}

\begin{definition}
    An ideal \(I \trianglelefteq R\), with \(I \neq R\), is called maximal if it is not contained in any strictly larger ideal \(J \neq R\).
\end{definition}

\exmp{
    \(10 \bZ \trianglelefteq \bZ\) is not maximal as \(10\bZ \subsetneqq 2\bZ \trianglelefteq \bZ\). However \(2\bZ \trianglelefteq \bZ\) is maximal.
}


\begin{proposition}
    Let \(R \neq 0\) be a commutative ring.
    \begin{enumerate}
        \item \(R\) is a field \(\iff\) every proper ideal is maximal
        \item if \(I \trianglelefteq R\), with \(I \neq R\), \(I\) is maximal \(\iff R / I\) is a field
    \end{enumerate}
    \proof{
        Assume \(R\) is a field. Let \(I \trianglelefteq R\), adn assume \(I \neq 0\). Then can choose \(x \in I, x \neq 0\). Then \(x\) is invertible, let \(y = x^{-1}\) then \(1 = yx \in I\) therefore \(I = R\). \\

        Converse: assume only ideals of \(R\) are \(0\) and \(R\). Take any \(x \in R, x \neq 0\). Consider \(I = \langle x \rangle\), cannot be 0, since \(x \in I\) then \(I = R\) so \(xy = 1\) for some \(y\). This proves \(x\) is invertible so \(R\) is a field.
    }
\end{proposition}

\begin{theorem}[Second Isomorphism Theorem]
    \(R\) is a ring. \(I \trianglelefteq R, J \trianglelefteq\) with \(J \subseteq I\). Then \(\frac{R / J}{I / J} \cong R / I\).
    \proof{
        Consider \(R \to R / J \to \frac{R/J}{I/J}\), show kernel is \(I\). Then follows from First Isomorphism Theorem.
    }
\end{theorem}

\begin{theorem}[Third Isomorphism Theorem]
    Let \(S \subseteq R\) be a subring and \(I \trianglelefteq R\). Then \(S + I\) is a subring of \(R\) and \(S \cap I \trianglelefteq S\).
    \[\frac{S}{S \cap I} \cong \frac{S + I}{I}.\]
\end{theorem}

\exmp{
    \(S = \bC[x]\) subring of \(R = \bC[x, y].\) Let \(I = \langle y \rangle \trianglelefteq \bC[x, y]\).

    \begin{itemize}
        \item \(S \cap I = \bC[x] \cap \langle y \rangle = 0\).
        \item \(S + I = \bC[x, y] = R\)
    \end{itemize}
    Then by the Third Isomorphism Theorem,
    \[\frac{S}{S \cap I} = \frac{\bC[x]}{0} = \bC[x] \quad \text{and} \quad \frac{S + I}{I} = \frac{\bC[x, y]}{\langle y \rangle},\]
    \[\bC[x, y] / \langle y \rangle \cong \bC[x].\]
}
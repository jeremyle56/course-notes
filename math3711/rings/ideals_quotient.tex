\section{Ideals and Quotient Rings}

Let \(R = \) ring.

\begin{definition}[Ideals]
    A subgroup \(I\) of the underlying abelian group \(R\) is called an ideal of \(R\) if
    \[\forall r \in R, x \in I, \text{ we have } rx \in I \tand xr \in I.\]
    Then we write \(I \trianglelefteq R\).
\end{definition}

\exmp{
    \(n\bZ \trianglelefteq \bZ\) is an ideal of \(\bZ\). It is a subgroup as if \(m \in n\bZ\) then \(rm \in n\bZ\) for any integer \(r\).
}

\begin{lemma}
    If \(\{I_i\}_{i \in A}\) ideals in \(R\) then \(\bigcap_{i \in A} I_i\)is an ideal of \(R\).
\end{lemma}

\begin{corollary}
    Let \(R = \) ring, \(S \subseteq R\) any subset. Let \(J = \) set of all ideals \(I \trianglelefteq R\) such that \(S \subseteq I\). Define \(\langle S \rangle = \bigcap_{I \in J} J\) as the ideal generated by \(S\). (i.e. smallest ideal containing \(S\)).
\end{corollary}

\begin{proposition}
    \begin{enumerate}
        \item If \(I, J \trianglelefteq R\) then ideal generated by \(I \cup J\) is \(I + J = \{i + j: i \in I, j \in J\}\).
        \item Assume \(R\) is commutative and \(x \in R\). Then \(\langle x \rangle = Rx = \{rx : r \in R\} \subseteq R\).
        \item \(R\) commutative, \(x_1, \dots, x_n \in R\). Then \(\langle x_1, \dots, x_n \rangle = Rx_1 + \dots Rx_n = \{r_1x_! + \dots r_nx_n : r_1, \dots, r_n \in R\}\). Set of \(R\)-linear combinations of \(x_1, \dots, x_n\).
    \end{enumerate}
\end{proposition}

\begin{prop-defn}[Quotient Ringa]
Let \(I \trianglelefteq R\). The abelian group \(R / I\) has a well-defined multiplication map \(\mu: R / I \times R / I \to R / I; (r + I, s + I) \mapsto rs + I\) which makes \(R / I\)into a ring, called the quotient ring of \(R\) by \(I\).
\proof{
    Check multiplication is well defined, given \(x, y \in I\), we need \(rs + I = (r + x)(s + y) + I.\) RHS = \(rs + xs + ry + xy + I = rs + I\) as \(xs, ry, xy \in I\). Note that the ring axioms for \(R / I\) follow from ring axioms for \(R\).
}
\end{prop-defn}

\exmp{
    Again \(\bZ / n\bZ\) is essentially modulo \(n\) arithmetic, i.e. \((i + n\bZ)(j + n\bZ) = ij + n\bZ\). Thus \(\bZ / n\bZ\) represents not only the addition but also the multiplication in modulo \(n\).
}
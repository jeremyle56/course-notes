\section{Polynomial Rings}

\begin{definition}[Polynomials]
    Let \(R\) be a ring. A polynomial in \(x\) with coefficients in \(R\) is a formal expression of the form
    \begin{align*}
        p &= \sum_{i \geq 0} r_i x^i \quad \text{ where } r_i \in R \tand r_i = 0 \text{ for all sufficiently large } i. \\
        &= r_0 x^0 + r_1 x^1 + \dots + r_n x^n.
    \end{align*}
    Let \(R[x]\) denote the set of all such polynomials.
\end{definition}

\begin{prop-defn}[Polynomial Ring]
    \(R[x]\) is a ring. called the (univariate) polynomial ring with coefficients in \(R\), when equipped with:
    \begin{itemize}
        \item Addition: \(\sum_{i \geq 0} r_i x^i + \sum_{i \geq 0} r_i' x^i = \sum_{i \geq 0} (r_i + r_i')x^i.\)
        \item Multiplication: \(\left(\sum_{i \geq 0} r_i x^i\right) + \left(\sum_{i \geq 0} r_i' x^i\right) = \sum_{i \geq 0} \left(\sum_{j + k = i} r_j r_k'\right) x^i.\)
        \item Zero: \(r_i = 0\) for all \(i\).
        \item One: \(r_0 = 1\) and \(r_i = 0\) for all \(i \geq 1\).
    \end{itemize}
\end{prop-defn}

\begin{proposition}
    Let \(\phi: R \to S\) be a ring homomorphism
    \begin{enumerate}
        \item \(R\) is a subring of \(R[x]\) under \(r \mapsto r + 0x + 0x^2 + \dots\)
        \item \(\phi\) induces \(\phi[x]: R[x] \to S[x]\) where \(\phi\left(\sum_{i \geq} r_i x^i \right) = \sum_{i \geq 0} \phi(r_i)x^i\) and this is a ring homomorphism.
    \end{enumerate}
\end{proposition}

\begin{definition}[Evaluation Homomorphism]
    Let \(S \subset R\) be a subring. Let \(r \in R\) such that \(rs = sr\) for all \(s \in S\). Define evaluation map:
    \[\epsilon_r: S[x] \to R; \quad p = \sum_{i \geq 0} s_i x^i \mapsto \sum_{i \geq 0} s_ir^i = p(r). \]
\end{definition}

\begin{proposition}
    \(\epsilon_r\) is a ring homomorphism from \(S[x] \to R\).
\end{proposition}

\begin{corollary}
    Assume \(R\) is commutative. Consider the map \(c: S[x] \to \Fun(R, R); p \mapsto (r \mapsto p(r))\). Then\(c\) is a ring homomorphism.
\end{corollary}

\exmp{
    \(p(x) := x^2 + x \in (\bZ / 2\bZ)[x]\). Trying values
    \[p(0) = 0^2 + 0 = 0 \quad \quad p(1) = 1^2 + 1 = 0 \]
    \(p(\alpha) = 0\) for all \(\alpha\) in domain \((\bZ / 2\bZ)\). We have \(p \neq 0\) in \((\bZ / 2\bZ)[x]\) but \(c(p) = 0\). That is, \(p\) defines a zero function.
}

\paragraph{Polynomials in Several Variables}
A possible definition is that \[R[x_1, x_2, \dots, x_n] = (\dots ((R[x_1])[x_2])[x_3] \dots [x_n]) = R[x_1][x_2] \cdots[x_n].\]

Another definition is that \(R[x_1, \dots, x_n] = \left\{ \sum_{i \in \bN^n} r_i x^i : \text{ only finitely many non-zero } r_i\text{'s}. \right\}\). Defined similarly to \(i = (i_1, \dots, i_n) : x^i = x_1^{i_1} x_2^{i_2} \dots x_n^{i_n}\). This definition then requires you to define suitable ring operations.

\begin{prop-defn}
    Let \(S\) be a subring of commutative ring \(R\) and \(r_1, \dots, r_n \in R\). Then \(S[r_1, \dots, r_n]\) is the subring of \(R\) generated by \(S \cup \{r_1, \dots, r_n\}\). Equivalently it is the image of \(S[x_1, \dots, x_n]\) under the evaluation map \(x_i \mapsto r_i\) for all \(i\).
\end{prop-defn}

\exmp{
    \(R = \bC, S = \bZ\). Then \(\bZ[i]\) is the subring generated by \(\bZ\) and \(i\). That is,
    \[\bZ[i] = \Im(\epsilon_i : \bZ[x] \mapsto \bC) = \left\{ \sum_{j \geq 0} a_j i^j : a_j \in \bZ \right\} = \{a + ib: a, b \in \bZ\}\]
}


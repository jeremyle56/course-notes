\section{The Mathematical Language of Symmetry}

\begin{definition}[Isometry]
    A function \(f: \bR^n \to \bR^n\) is an isometry if \(\norm{f(x) - f(y)} = \norm{x - y} \forall x, y \in \bR^n\). i.e. preserves distances.
\end{definition}

\begin{definition}[Symmetry]
    Let \(F \subseteq \bR^n\), a symmetry of \(F\) is a (surjective) isometry \(T : \bR^n \to \bR^n\) such that \(T(F) = F\).
\end{definition}

\begin{properties}
    Let \(S, T\) be symmetries of \(F \subseteq \bR^n\). Then \(S \cdot T : \bR^n \to \bR^n\) is also a symmetry of \(F\).
\end{properties}

\proof{
    Given \(x, y \in \bR^n\).
    \begin{align*}
        \norm{ST x - ST y} & = \norm{Tx - Ty} \tag{\(S\) is an isometry} \\
                           & = \norm{x - y}. \tag{\(T\) is an isometry}
    \end{align*}
    Therefore \(ST\) is an isometry. Clearly \(ST\) is surjective as both \(S\) and \(T\) are surjective. Also,
    \begin{align*}
        ST(F) & = S(F) \tag{\(T(F) = F\)} \\
              & = F. \tag{\(S(F) = F\)}
    \end{align*}
    So \(ST\) is a symmetry of \(F\).
    \qed
}

\begin{properties}
    If \(G =\) set of symmetries of \(F \subseteq \bR^n\), then \(G\) satisfies:
    \begin{enumerate}
        \item Composition is associative, \(ST(R) = S(TR) \forall S,T,R \in G\).
        \item \(\id_{\bR^n} \in G \quad (\id_{\bR^n}(x) = x \quad \forall x \in \bR^n)\). Also, \(\id_G T = T\) and \(T\id_G = T \forall T \in G\).
        \item If \(T \in G\), then \(T\) is bijective and \(T^{-1} \in G\).
              \proof{
                  If \(Tx = Ty\), then \(\norm{Tx - Ty} = 0\). So \(\norm{x - y} = 0, x = y\), therefore \(T\) is injective. By definition \(T\) is surjective, hence, \(T\) is bijective and therefore \(T^{-1}\) is surjective. \\

                  To prove \(T^{-1}\) is an isometry.
                  \begin{align*}
                      \norm{T^{-1}x - T^{-1}y} & = \norm{TT^{-1}x - TT^{-1}y} \\
                                               & = \norm{\id x - \id y}       \\
                                               & = \norm{x - y}.
                  \end{align*}

                  To prove symmetry, \(T^{-1}F = F\):
                  \[T^{-1}F = T^{-1}(T(F)) = F.\]

                  Thus \(T^{-1} \in G\).
                  \qed
              }
    \end{enumerate}
\end{properties}

\begin{definition}[Group]
    A group is a set \(G\) equipped with a ``multiplication map'' \(\mu: G \times G \to G\) such that
    \begin{enumerate}[label=\arabic*)]
        \item Associativity: \((gh)k = g(hk) \forall g,h,j \in G\).
        \item Existence of identity: There exists \(1 \in G\) such that \(1g = g\) and \(g1 = g \forall g \in G\).
        \item Existence of inverses: \(\oldforall g \in G, \exists h \in G\) such that \(gh = 1\) and \(hg = 1\). Denoted by \(g^{-1}\).
    \end{enumerate}
\end{definition}

\begin{properties}
    Basic facts about groups.
    \begin{itemize}
        \item ``\textbf{Generalised Associativity}''. When multiplying three or more elements, the bracketing does not matter. E.g. \((a(b(cd)))e = (ab)(c(de))\).
              \proof{Mathematical Induction as for matrix multiplication.}
        \item \textbf{Cancellation Law}. If \(gh = gk\) then \(h = k \forall g, h, k \in G\).
              \proof{
                  \(gh = gk \implies g^{-1}(gh) = g^{-1}(gk) \implies (g^{-1}g)h = (g^{-1}g)k \implies 1h = 1k \implies h = k.\)
              }
    \end{itemize}
\end{properties}

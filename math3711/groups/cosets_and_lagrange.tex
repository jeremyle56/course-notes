\section{Cosets and Lagrange's Theorem}

Let \(H \leq G\) be a subgroup. This will apply to all statements in this section unless mentioned otherwise.

\begin{definition}[Coset]
    A left coset of \(H\) in \(G\) is a set of the form \(gH = \{gh : h \in H\} \subseteq G\) for some \(g \in G\). The set of left cosets is denoted byt \(G / H\).
    \begin{example}
        Let \(H = A_n \leq S_n = G\) for \(n \geq 2\). Let \(\tau\) be any transposition. We claim that \(\tau A_n = \{\text{odd permutations}\}\).
        \begin{enumerate}
            \item [\(\subseteq\)]: \(\tau A_n = \{\tau\sigma : \sigma \text{ even}\}\), they are all odd.
            \item [\(\supseteq\)]: Suppose \(\sigma\) is odd, then \(\sigma = \tau \cdot (\tau^{-1}\sigma) \in \tau A_n\).
        \end{enumerate}
    \end{example}
\end{definition}

\begin{theorem}
    Define a relation on \(G: g \equiv g'\) if and only if \(g \in g'H\). Then \(\equiv\) is an equivalence relation, the equivalence classes are the left cosets. Therefore \(G = \dot{\bigcup}_{i \in I} g_iH\) (disjoint union).
    \proof{
        \begin{enumerate}
            \item Reflexive. i.e. \(g \in gH\) for all \(g \in G\). True because \(1 \in H\).
            \item Symmetry. Suppose \(g \in g'H\), need to prove \(g' \in gH\). Since \(g \in g'H\) we have \(g = g'H\) for some \(h \in H\). \(g' = gh^{-1}\) so \(g' \in gH\) (as \(h^{-1} \in H\)).
            \item Transitivity. Suppose \(g \in g'H\) and \(g' \in g''H\). Then \(g = g'h\) and \(g' = g''h'\) for \(h, h' \in H\). Therefore \(g = (g''h)h = g''(h'h) \in g''H\) from associativity and \(h'h \in H\).
        \end{enumerate}
        Thus \(\equiv\) is an equivalence relation and \(G\) is a disjoint union of equivalence classes.
    }
\end{theorem}

Note \(1H = H\) is always a coset of \(G\) and the coset containing \(g \in G\) is \(gH\).

\exmp{
    \(H = A_n \leq S_n = G\) cosets are exactly \(S_n \tand \tau S_n\) where \(S_n = A_n \dot{\bigcup} \tau A_n\).
}

\begin{definition}[Index]
    The index of \(H\) in \(G\) is the number of left cosets, i.e. \(|G/H|\). Denoted by \([G:H]\).
\end{definition}

\begin{lemma} \label{coset-cardinality-lemma}
    Let \(g \in G\). Then \(H\) and \(gH\) have the same cardinality.
    \proof{
        Bijection, \(H \to gH, h \mapsto gh\). Surjective and injective (multiply on lefy by \(g^{-1}\)).
    }
\end{lemma}

\begin{theorem}[Lagrange's Theorem]
    Assume \(G\) finite. Then \(|G| = |H|[G:H]\) i.e. \(|G/H| = |G| / |H|\).
    \proof{
    Using Lemma \ref{coset-cardinality-lemma}, we have:
    \[G = \bigcup_{i=1}^{[G:H]} g_iH \quad (\text{disjoint union}) \implies |G| = \sum_{i = 1}^{[G:H]} |g_i H| = \sum_{i=1}^{[G:H]}|H| = [G:H]|H|.\]
    }
\end{theorem}

\exmp{
    \(A_n \leq S_n\). \([S_n : A_n] = 2 \implies |S_n| = 2 |A_n| \implies n! = 2 * n!/2\).
}

\bigskip
All above statements hold for right cosets which have form \(Hg = \{hg : h \in H\}\) denoted \(H \backslash G\). The number of left cosets are equal the number of right cosets.

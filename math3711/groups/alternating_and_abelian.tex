\section{Alternating and Abelian Groups}

\begin{definition}[Symmetric Functions]
    Let \(f(x_1, \dots, x_n)\) be a function of \(n\) variables. Let \(\sigma \in S_n\). We define function \((\sigma f)(x_1, \dots, x_n) = f(x_{\sigma(1)}, \dots, x_{\sigma(n)})\). We say that \(f\) is symmetric if \(\sigma f = f \forall \sigma \in S_n\).
    \exmp{
        Suppose \(f(x_1, x_2, x_3) = x_1^3 x_2^2 x_3\) and \(\sigma = (12)\) then \(\sigma f(x_1, x_2, x_3) = x_2^3, x_1^2 x_3\). Not symmetric because \(x_1^3 x_2^2 x_3 \neq x_2^3 x_1^2 x_3\). But \(f(x_1, x_2) = x_1^2 x_2^2\) is symmetric in two variables.
    }
\end{definition}

\begin{definition}[Difference Product]
    The difference product in (\(n\) variables) is
    \[\Delta(x_1, \dots, x_n) = \Pi_{i < j} (x_i - x_j).\]
\end{definition}

\begin{lemma}
    Let \(f(x_1, \dots, x_n)\) be a function in \(n\) variables. Let \(\sigma, \tau \in S_n\), then \((\sigma\tau) \cdot f = \sigma \cdot (\tau f)\).
    \proof{
        \begin{align*}
            (\sigma \cdot (\tau f))(x_1, \dots, x_n) & = (\tau f)(x_{\sigma(1)}, \dots, x_{\sigma(n)}) \tag{by definition}    \\
                                                     & = f(y_{\tau(1)}, \dots, y_{\tau(n)}) \tag{where \(y_i = x_\sigma(i)\)} \\
                                                     & = f(x_{\sigma(\tau(1))}, \dots, x_{\sigma(\tau(n))})                   \\
                                                     & = f(x_{(\sigma\tau)(1)}, \dots, x_{(\sigma\tau)(n)})                   \\
                                                     & = ((\sigma\tau) \cdot f)(x_1, \dots, x_n).
        \end{align*}
        Note, the second and third step follows because \(x_{\sigma(1)}\) is not necessarily \(x_1\), so \(\tau\) is applied to \(x_1\) first, then \(\sigma\) can be applied.
    }
\end{lemma}

\begin{prop-defn}
For \(\sigma \in S_n\) write \(\sigma = \tau_1\tau_2 \dots \tau_m\) where \(\tau_i\) are transpositions. Then
\[\sigma \cdot \Delta = \begin{cases}
        \Delta  & \text{if } m \text{ even (call \(\sigma\) an even permutation)} \\
        -\Delta & \text{if } m \text{ odd (call \(\sigma\) an odd permutation)}
    \end{cases}\]
\proof{
    Sufficent to prove for a single transposition (i.e. \(m = 1\)) because by the above Lemma,
    \[\sigma\Delta = \tau_1(\tau_2\dots(\tau_{m-1}(\tau_m\Delta))\dots) = \tau_1((-1)^{m-1}\Delta) = (-1)^m \Delta.\]

    Let's assume \(\sigma = (ij), i < j\). There are 3 cases:
    \begin{enumerate}
        \item \(x_i - x_j \implies x_j - x_i\) (factor of -1).
        \item \(x_r - x_s\) where \(i,j,r,s\) all distinct \(\implies x_r - x_s\) (factor of +1).
        \item \(x_r - x_s\) where one of \(r, s\) is equal to \(i\) or \(j\). There are several subcases:
              \begin{enumerate}
                  \item \(r < i < j\): \(x_r - x_i \implies x_r - x_j\) but also \(x_r - x_j \implies x_r - x_i\), no change (factor of +1).
                  \item \(i < r < j\): \((x_i - x_r)(x_r - x_j) \implies (x_j - x_r)(x_r - x_i)\) (factor of +1).
                  \item \(i < j < r\): similar to (a) (factor of +1).
              \end{enumerate}
    \end{enumerate}
    So only change in i). Multiplying the three cases together yields \(\sigma \cdot \Delta = -\Delta\).
}
\end{prop-defn}

\begin{corol-defn}[Alternating Group]
The alternating group (on \(n\) symbols) is
\[A_n = \{\sigma \in S_n: \sigma \text{ is even}\}.\]
This is a subgroup of \(S_n\). Also \(A_n\) is generated by \(\{\tau_1 \tau_2: \tau_1, \tau_2 \text{ are transposition}\}\).
\exmp{
    \(A_3 = \{1, (123), (132)\}, S_3 \setminus A_3 = \{(12), (13), (23)\}. |A_n| = n!/2\) except for \(n = 1, A_1 = S_1 = \{1\}\).
}
\end{corol-defn}

\begin{definition}[Abelian Group]
    A group \(G\) is abelian if any two elements commute.
\end{definition}

In abelian groups, often switch to additive notation:
\begin{enumerate}
    \item product \(gh \implies g + h\)
    \item identity \(1 \implies 0\)
    \item power \(g^n \implies ng\)
    \item inverse \(g^-1 \implies -g\)
\end{enumerate}
This notation follows from \(\bZ\) endowed with addition which forms an abelian group.

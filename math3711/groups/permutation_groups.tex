\section{Permutation Groups}

\begin{definition}[Permutations]
    Let \(S\) be a set. Let \(\Perm(S)\) be the set of permutations of \(S\). This is the set of bijections of form \(\sigma: S \to S\).
\end{definition}

\begin{proposition}
    \(\Perm(S)\) is a group when endowed with composition of functions.
    \proof{
        Composition of bijections is a bijection. The identity is \(\id_S\) and group inverse is the inverse function.
    }
\end{proposition}

\begin{definition}[Symmetric Group]
    Let \(S = \{1, \dots, n\}\). The symmetric group \(S_n\) is \(\Perm(S)\).
\end{definition}

Two notations are used. With the two line notation, represent \(\sigma \in S_n\) by
\[\begin{pmatrix}
        1         & 2         & 3         & \cdots & n         \\
        \sigma(1) & \sigma(2) & \sigma(3) & \cdots & \sigma(n)
    \end{pmatrix}\]
(\(\sigma(i)\)'s are all distinct, hence \(\sigma\) is one to one and bijective). Note this shows \(|S_n| = n!\). \\

With the cyclic notation, let \(s_1, s_2, \dots, s_k \in S\) be distinct. We define a new permutation \(\sigma \in \Perm(S)\) by \(\sigma(s_i) = s_{i + 1}\) for \(i = 1, 2, \dots, k - 1, \sigma(s_k) = \sigma(s_1) \tand \sigma(s) = s\) for \(s \notin \{s_1, s_2, \dots, s_k\}\). Denoted \((s_1 s_2 \dots s_k)\) and called a k-cycle.

\exmp{
    For \(n = 4\),
    \[
        \sigma =
        \begin{pmatrix}
            1 & 2 & 3 & 4 \\
            2 & 3 & 1 & 4
        \end{pmatrix} \in S_4 \quad \text{ means } \quad
        \begin{matrix}
            \sigma(1) = 2, & \sigma(2) = 3  \\
            \sigma(3) = 1, & \sigma(4) = 4.
        \end{matrix}
    \]
    In cyclic notation this is \((123)(4)\) or \((123)\) where the cycle is \(1 \to 2 \to 3 \to 1\).
}

\bigskip
Note that a 1-cycle is the identity and the order of a k-cycle is \(k\). So \(\sigma^k = 1\) and \(\sigma^{-1} = \sigma^{k - 1}\).

\begin{definition}[Disjoint Cycles]
    Cycles \(s_1 \dots s_k\) and \(t_1 \dots t_k\) are disjoint if \(\{s_1, \dots, s_k\} \cup \{t_1, \dots, t_k\} = \emptyset\).
\end{definition}

\begin{definition}[Commuativity]
    In any group, two elements \(g, h\) commute if \(gh = hg\).
\end{definition}

\begin{proposition}
    Disjoint cycles commute.
\end{proposition}

\begin{proposition}
    Any permutation \(\sigma\) of a finite set \(S\) is a product of disjoint cycles.
    \exmp{
        \(\sigma = \begin{pmatrix}
            1 & 2 & 3 & 4 & 5 & 6 \\
            2 & 4 & 6 & 1 & 5 & 3
        \end{pmatrix} \in S_6 \) does \(1 \to 2 \to 4 \to 1\), \(3 \to 6 \to 3\) and \(5 \to 5\).

        Thus \(\sigma = (124)(36)\) since \((5)\) is the identity.
    }
\end{proposition}

\begin{proposition}
    Let \(\sigma\) be a permutation of a finite set \(S\). Then \(S\) is a disjoint union of subsets, say \(S_1, \dots, S_r\), such that \(\sigma\) permutes the elements of each \(S_i\) cyclically.
\end{proposition}

\begin{definition}[Transposition]
    A transposition is a \(2-\)cycle i.e. \((ab)\).
\end{definition}

\begin{proposition}
    \begin{enumerate}
        \item The k-cycle \((s_1 s_2\dots s_k) = (s_1 s_k)(s_1 s_{k - 1}) \dots (s_1 s_3)(s_1 s_2)\)
              \exmp{
                  \((3625) = (35)(32)(36) = (36)(62)(25)\)
              }
              \proof{
                  The RHS produces the mapping below which is equivalent to the LHS.
                  \begin{align*}
                      s_1       & \to s_2         \\
                      s_2       & \to s_1 \to s_3 \\
                      s_3       & \to s_1 \to s_4 \\
                                & \vdots          \\
                      s_{k - 1} & \to s_1 \to s_k \\
                      s_k       & \to s_1.
                  \end{align*}
              }
        \item Any permutations in \(S_n\) is a product of transpositions.
              \proof{
                  We can write any \(\sigma \in S_n\) as product of (disjoint) cycles. By part i), each cycle is a product of transpositions. So we can write \(\sigma\) as product of transpositions.
              }
    \end{enumerate}
\end{proposition}
\section{Group Homomorphisms}

\begin{definition}[Homomorphism]
    Given groups \(G, H\). A function \(\phi: H \to G\) is a homomorphism of groups if \(\phi(hh') = \phi(h)\phi(h') \forall h, h' \in H\).
\end{definition}


\begin{prop-defn}[Isomorphisms and Automorphisms]
Let \(\phi: H \to G\) be a group homomorphism. The following are equivalent:
\begin{itemize}
    \item There exists a group homomorphism, \(\psi: G \to H\) such that \(\psi\phi = \id_H \tand = \phi\psi = \id_G\)
    \item \(\phi\) is bijective.
\end{itemize}
We call \(\phi\) is a group isomorphism. If \(H = G\), \(\phi\) is an automorphism.
\end{prop-defn}

\begin{proposition}
    If \(\phi: H \to G, \psi: K \to H\) are group homomorphism then \(\phi \cdot \psi: K \to G\) is a homomorphism.
    \proof{
        \((\phi \cdot \psi)(kk') = \phi(\psi(kk')) = \phi(\psi(k)\psi(k')) = \phi(\psi(k))\phi(\psi(k'))\)
    }
\end{proposition}

\begin{proposition}
    Let \(\phi: H \to G\) be a group homomorphism.
    \begin{enumerate}
        \item \(\phi(1_H) = 1_G\).
        \item \(\phi(h^{-1}) = \phi(h)^{-1} \forall h \in H\).
        \item if \(H' \leq H\) then \(\phi(H') \leq G\).
    \end{enumerate}
\end{proposition}

\begin{prop-defn}
Let \(G\) be a group with \(g \in G\). Conjugation by \(g\) is the map \(C_g: G \to G; h \mapsto ghg^{-1}\). Then \(C_g\) is an automorphism with inverse \(C_{g^{-1}}\).
\proof{
\(C_g\) is a homomorphism: \(C_g(h_1h_2) = C_g(h_1)C_g(h_2)\). Check: \(C_g(h_1h_2) = gh_1h_2g^{-1} = gh_1g^{-1}gh_2g^{-1} = C_g(h_1)C_g(h_2)\). Now check \(C_{g^{-1}}\) is an inverse. \(C_{g^{-1}}(C_g(h)) = C_{g^{-1}}(ghg^{-1}) = g^{-1}ghg^{-1}g = h\). Similarly \(C_g(C_{g^{-1}})(h) = h\), therefore \((C_g)^{-1} = C_{g^{-1}}\).
}
\end{prop-defn}

\begin{corol-defn}
For \(H \leq G\), a conjugate of \(H\) (in \(G\)) is a subgroup of \(G\) of the form \(gHg^{-1} := c_g(H)\).
\end{corol-defn}

\begin{definition}[Epimorphism and Monomorphism]
    Let \(\phi: H \to G\) be a group homomorphism. \(\phi\) is an epimorphism if \(\phi\) is surjective. \(\phi\) is a monomorphism if \(\phi\) is injective.
\end{definition}

\exmp{
    Linear map \(T: V \to W\) where \(V\) and \(W\) are vector spaces. Suppose \(T\) is a projection onto some subspace. What does \(T^{-1}(w) = \{v \in V: T(v) = w\}\) looks like, for a given \(w \in W\)? \\

    If \(w \in L\), \(T^{-1}(w) = \emptyset\)

    If \(w \in L, T^{-1}(w)\) = plane containing \(w\), orthogonal to \(L\) = \(w + K\) where \(K = \) kernel of \(T = T^{-1}(0)\).
}

\begin{definition}
    Let \(\phi: H \to G\) be a group homomorphism. The kernel of \(\phi\) is
    \[\ker \phi = \phi^{-1}(1_G) = \{h \in H: \phi(h) = 1_G\}\]
\end{definition}

\begin{proposition}
    Let \(\phi: H \to G\) be a group homomorphism.
    \begin{enumerate}
        \item If \(G' \leq G\) then \(\phi^{-1}(G') \leq H\).
        \item If \(G' \trianglelefteq G\) then \(\phi^{-1}(G') \trianglelefteq H\).
              \proof{
                  (Normality) Given \(h \in \phi^{-1}(G')\) and\(g \in H\). We need to prove \(ghg^{-1} \in \phi^{-1}(G') \implies \phi(ghg^{-1}) \in G \implies \phi(g)\phi(h)\phi(g)^{-1} \in G\) true because \(\phi(h) \in G'\) and \(G' \trianglelefteq G\).
              }
        \item \(K = \ker \phi \trianglelefteq H\).
              \proof{
                  Follows from (ii) because \(K = \phi^{-1}(\{1\})\) and \(\{1\} \trianglelefteq G\).
              }
        \item The non-empty fibres of \(\phi\), i.e. \(\phi^{-1}(g)\) for all \(g \in G\), are exactly the cosets of \(H\).
              \proof{
              Suppose \(g \in G\), consider \(\phi^{-1}(g)\). Assume \(\phi^{-1}(g) \neq \phi\). Let \(h \in \phi^{-1}(g)\). \\

              {\bf Claim.} \(\phi^{-1}(g) = hK\). \\
              {\bf Proof.} \(hK \subseteq \phi^{-1}(g)\) because \(\phi(hK) = \phi(h)\phi(j) = g \cdot 1 = g\).

                  {\bf Converse:} \(\phi^{-1}(g) \subseteq hK\). Let \(h' \in \phi^{-1}(g)\). Then \(\phi(h') = g\), also \(\phi(h) = g\). Therefore \(\phi(h'h^{-1}) = \phi(gg^{-1}) = \phi(1) = 1\). So \(h'h^{-1} \in K, h' \in Kh = hK\), thus \(\phi^{-1}(g) = hK\).
              }
        \item \(\phi\) is one to one if and only if \(K = \{1\}\).
              \proof{
                  \((\implies)\) trivial. \((\impliedby)\) Assume \(K = \{1\}\). By part (iv) fibres \(\phi^{-1}(g)\) are cosets of \(\{1\}\) hence contain single element.
              }
    \end{enumerate}
\end{proposition}

\begin{prop-defn}
Let \(N \trianglelefteq G\). The quotient monomorphism (of \(G\) by \(N\)) is the map \(\pi: G \to G/N; g \mapsto gN\). Its an epimorphism with kernel \(N\).
\end{prop-defn}
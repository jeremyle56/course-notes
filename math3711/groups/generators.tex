\section{Generators and Dihedral Groups}

\begin{lemma} \label{intersection-subgroup}
    Let \(\{H_i\}_{i \in I}\) be a (non-empty) collection of subgroups of \(G\). Then \(\bigcap_{i \in I} H_i \leq G\).
    \proof{
        \begin{enumerate}[label=\arabic*)]
            \item Why is \(1 \in \bigcap_{i \in I} H_i\)? Because \(1 \in H_i \forall i\).
            \item Closed under multiplication? If \(g, h \in \bigcap_{i \in I}H_i\), then \(g,h \in H_i \forall i \implies gh \in H_i \forall i \implies gh \in \and_{i \in I} H_i\).
            \item Closed under taking inverse? If \(g \in \bigcap_{i \in I} H_i\) then \(g \in H_i \forall i\) as \(H_i\) are subgroups, every element has an inverse. So an inverse exists for all elements in \(H_i \forall i\).
        \end{enumerate}
    }
\end{lemma}

\begin{prop-defn}
Let \(G\) be a group and \(S \subseteq G\). Let \(\mathcal{J}\) be the set of subgroups \(J \leq G\) containing \(S\).
\begin{enumerate}
    \item \text{[Definition]} The subgroup generated by \(S\), \(\langle S \rangle\) is \(\bigcap J \in \mathcal{J} \leq J \leq G\). i.e. it's the intersection of all subgroups of \(G\) containing \(S\).
          \proof{
              Lemma \ref{intersection-subgroup} implies \(\langle S \rangle\) is a subgroup of \(G\).
          }
    \item \text{[Proposition]} \(\langle S \rangle\) is the set of elements of the form \(g = s_1 s_2 \dots s_n\) where \(n \geq 0\) and \(s_i \in S \cup S^{-1}\). Define \(g = 1\) when \(n = 0\).
          \proof{
          Let \(H = \{ s_1 \dots s_n : s_i \in S \cup S^{-1}\}\). First, \(H \subseteq \langle S \rangle\). Need to prove that \(s_i \cdots s_n \in\) every \(J\).
          Each \(s_i \in J\) because \(s_i = s\) or \(s^{-1}\) for some \(s \in S \leq J\) and \(J\) closed under inversion. Therefore, \(s_1 \dots s_n \in J\) by closure under multiplication. Hence \(s_1 \dots s_n \in \bigcap_{J \in \mathcal{J}} J = \langle S \rangle\). \\

          Second, \(\langle S \rangle \subseteq H\). Need to prove \(H\) is a subgroup containing \(S\). Closure under multiplication: \((s_1 \dots s_n)(t_1 \dots t_m) = s_1 \dots s_nt_1 \dots t_m\) also closure under inversion: \( (s_1 \dots s_n)^{-1} = s_1^{-1} \dots s_n^{-1} \in H\) since \(s_i^{-1} \in S \forall i\). Identity: \(s,s^{-1} \in S \neq \emptyset \implies ss^{-1} = 1 \in H\).
          }
\end{enumerate}
\end{prop-defn}

\begin{definition}[Finitely Generated]
    A group \(G\) is finitely generated \(f.g.\) if \(G = \langle S \rangle\) for a finite subset \(S \subseteq G\). \(G\) is cyclie if we can take \(|S| = 1\).
\end{definition}

\exmp{
    Take \(G \in \operatorname{GL}_2(\bR)\) with \(\sigma = \begin{pmatrix}
        \cos(\frac{2\pi}{n}) & -\sin(\frac{2\pi}{n}) \\
        \sin(\frac{2\pi}{n}) & -\cos(\frac{2\pi}{n})
    \end{pmatrix}\) and \(\tau = \begin{pmatrix}
        1 & 0  \\
        0 & -1
    \end{pmatrix}\). Find the subgroup generated by \(\{ \sigma, \tau \}\). \\

    Notice both \(\sigma, \tau\) are symmetries of any \(n\)-gon. Any element of \(\langle \sigma, \tau \rangle\) has form
    \[\sigma^{i_1}\tau^{j_1}\sigma^{i_2}\tau^{j_2} \dots \sigma^{i_r}\tau^{j_r} \quad \text{ for } i_1, \dots, i_r, j_1, \dots, j_r \in \bZ.\]
    We have relations: \(\sigma^n = 1, \tau^2 = 1 \tand \tau\sigma\tau^{-1} = \sigma^{-1}\). We use these relations to push all \(\sigma\)'s to the left and all \(\tau\)'s to the right to achieve the form \(\sigma^i\tau^j\) where \(0 \leq i < n \tand j = 0, 1\).
}

\begin{prop-defn}
\(\langle \sigma, \tau \rangle\) = dihedral group of \(2n\), denoted \(D_n\) (sometimes \(D_{2n}\)).
\[D_n = \{1, \sigma, \dots, \sigma^{n-1}, \tau, \sigma\tau, \sigma^2\tau, \dots, \sigma^{n - 1}\tau \} \tand |D_n| = 2n.\]
\proof{
    Need to show \(2n\) elements are all distinct. \(\det(\sigma^i) = 1\) (because \(\det(\sigma) = 1\)), \(\det(\tau) = -1\) and \(\det(\sigma^i\tau) = -1\). We conclude, \(\{ 1, \sigma, \dots, \sigma^{n-1} \} \cap \{\tau, \sigma\tau, \dots, \sigma^{n-1}\tau\} = \emptyset\) because \(\sigma^k = \begin{pmatrix}
        \cos(\frac{2k\pi}{n}) & -\sin(\frac{2k\pi}{n}) \\
        \sin(\frac{2k\pi}{n}) & \cos(\frac{2k\pi}{n})
    \end{pmatrix}\) are distinct. If \(\sigma^i \tau = \sigma^j \tau\) then \(\sigma^i = \sigma^j\) then \(i = j\).
}
\end{prop-defn}
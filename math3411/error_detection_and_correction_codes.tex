\section{Error Detection and Correction Codes}

We say that \(\fx\) \textbf{corrupted} to \(\fy\) is denoted by \(\fx \rightsquigarrow \fy\).
\subsection{ISBN-10 Error Capability}
ISBN-10 numbers are capable of detecting the two types of errors:
\begin{enumerate}
    \item getting a digit wrong,
    \item interchanging two (unequal) digits.
\end{enumerate}

\subsection{Types of Codes}
\begin{itemize}
    \item \textbf{variable length code}: codewords have different lengths
    \item \textbf{block code}: codewords have the same lengths
    \item \textbf{\(t\)-error correcting code}: code can always correct up to \(t\) errors
    \item \textbf{systematic code}: code with \textbf{information digits} and \textbf{check digits} distinct
\end{itemize}

\subsection{Binary Repetition Codes}
A \textbf{binary \(r\)-repetition code} encodes \(0 \to \overbrace{0 \cdots 0}^r\) and \(1 \to \overbrace{1 \cdots 1}^r\). \\

The binary \((2t + 1)\)-repetition code is \(t\)-error correcting.

The binary \(2t\)-repetition code is \((t- 1)\)-error correcting and \(t\)-error detecting.

\subsection{Information Rate and Redundancy}
The \textbf{information rate} \(R\) is given by,
\begin{itemize}
    \item For a code \(C\) of radix \(r\) and length \(n\), \(R = \dfrac{\log_r|C|}{n}\)
    \item For a \textbf{systematic code}, \(R = \dfrac{\text{\# information digits}}{\text{length of code}}\)
\end{itemize}
We then define \textbf{redundancy} = \(\frac{1}{R}\).

\subsection{Binary Hamming Error-Correcting Codes}
A Binary Hamming \((n, k)\) code is a code of length \(n\) with \(k\) information bits, such that it is a single error correcting and has a parity check matrix, \(H\), of size \(n - k\) by \(n\).

\subsection{Hamming Distance, Weights}
The \textbf{weight} of an \(n\)-bit word \(\fx\) is defined to be
\[w(\fx) = \# \{i : 1 \leq i \leq n, x_i \neq 0\}.\]

Given two \(n\)-bit words, the \textbf{Hamming distance} between them is
\[d(\fx, \fy) = \# \{i : 1 \leq i \leq n, x_i \neq y_i \}.\]

Given some code with set of codewords \(C\), we define \textbf{(minimum) weight} of \(C\) to
\[w = w(C) = \min\{w(\fx) : \fx \in C, \fx \neq \fzero\}.\]

Similarly, the \textbf{(minimum) distance of \(C\)} is defined by
\[d = d(C) = \min\{d(\fx, \fy) : \fx, \fy \in C, \fx \neq \fy \}.\]

If \(\fx \rightsquigarrow \fy\), then \(d(\fx, \fy)\) is the number of errors in \(\fy\).

\subsection{Decoding Strategies}
\paragraph{Minimum Distance Decoding Strategy}
Given a received word \(y\), decode to \textit{closest} codeword \(x\).

\paragraph{Standard Strategy} If received word \(y\) is distance at most \(t\) from a codeword \(x\), then decode \(y\) to \(x\); otherwise flag an error.

\paragraph{Pure Error Detection}
If received word \(y\) is not a codeword \(x\), then flag an error.

% If \(e + f = d - 1\) and \(f \geq e\), then there is a strategy which is \(e\)-error correcting and \(f\)-error detecting.

\subsection{Sphere Packing}
The \textbf{sphere} of radius \(r\) around \(\mbf{c}\):
\[S_r(\mbf{c}) = \{ \fx \in \bZ_2^n : d(\fx, \mbf{c}) \leq r \}.\]
The volume of this sphere is its size \(|S_r(\mbf{c})|\).

\paragraph{Sphere-Packing Condition Theorem}
A \(t\)-error correcting binary code \(C\) of length \(n\) has minimum distance \(d = 2y + 1\) or \(2t + 2\), and
\[|C|\sum_{i=0}^t \binom{n}{i} \leq 2^n.\]

If we have equality in the bound, then we say that the code is perfect.  This means that codewords are evenly spread around in \(\bZ_2^n\) space.

\subsection{Binary Linear Codes}
A linear code \(C\) is a vector space over some field \(\bF\). Equivalently it is the null-space of
\[C = \{\fx \in \bF^n : H\fx^T = \fzero \}\]
of an \(m \times n\) parity check matrix \(H\) with \(m = \rank(H)\).
\begin{itemize}
    \item \(\dim C = k = n - m\) by the Rank-Nullity Theorem.
    \item If \(C\) is binary, then \(|C| = 2^k\).
    \item \(C\) is systematic.
    \item If \(H\) is \textbf{reduced echelon form}, then we can choose the non-leading columns of \(H\) to be \textbf{information bits} and the leading columns of \(H\) to be \textbf{check bits}.
\end{itemize}

\paragraph{Minimum Distance for Linear Codes}
If \(C\) is a linear code with parity check matrix \(H\), then
\begin{itemize}
    \item\(w(C) = d(C)\),
    \item \(d(C) = \min\{ r: H \text{ has } r \text{ linearly dependent columns } \}\).
\end{itemize}

For a linear code \(C\), the \textbf{row space} (or \textbf{row span}) of a \(k \times n\) \textbf{generator matrix} \(G\) over \(\bF\) generates \(C\), in the sense that \(C\) is a set of linear combinations of \(G\).

\subsection{Standard Form Matrices}
For a linear code \(C\) of dimension \(k\) and length \(n = k + m\),
\begin{itemize}
    \item \(H = (I_m \mid B)\) is a parity check matrix for \(C\) \textit{if and only if}
    \item \(G = (-B^T \mid I_k)\) is a generator matrix \(C\).
\end{itemize}

Linear codes \(C\)and \(C'\) are \textbf{equivalent} if \(C'\) is obtained by permuting the codeword entries of \(C\) by a fixed permutation:
\[C' = CP = \{ \fx P : \fx \in C \} \text{ for some permutation matrix } P\]
Note that \(G' = GP\) and \(H' = HP\).

\subsection{Extending Linear Codes}
The \textbf{extension} of a linear code \(C\):
\[\hat{C} = \{ x_0x_1 \cdots x_n : x_1 \cdots x_n \in C, x_0 = -(x_1 + \cdots + x_n) \}.\]

The \textbf{extension} \(\hat{C}\) is a linear code with minimum distance \(d(C)\) or \(d(C) + 1\).

\subsection{Radix \(r\) Hamming Codes}
\begin{itemize}
    \item Let \(r\) be a prime number and \(m \geq 1\) some integer.
    \item Write the numbers \(1, \dots, r^m - 1\) in base \(r\), as length \(m\) column vectors.
    \item Of each set of \(r - 1\) parallel columns, delete all whose \textit{first nonzero entry} is not 1.
    \item This gives the \textbf{radix \(r\) Hamming code} of length \(n = \frac{r^m - 1}{r - 1}\).
\end{itemize}

\newpage
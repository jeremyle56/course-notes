\section{Error Detection and Correction Codes}

We say that \(\fx\) \textbf{corrupted} to \(\fy\) is denoted by \(\fx \rightsquigarrow \fy\).
\subsection{ISBN-10 Error Capability}
ISBN-10 numbers are capable of detecting the two types of errors:
\begin{enumerate}
    \item getting a digit wrong,
    \item interchanging two (unequal) digits.
\end{enumerate}

\subsection{Types of Codes}
\begin{itemize}
    \item \textbf{variable length code}: codewords have different lengths
    \item \textbf{block code}: codewords have the same lengths
    \item \textbf{\(t\)-error correcting code}: code can always correct up to \(t\) errors
    \item \textbf{systematic code}: code with \textbf{information digits} and \textbf{check digits} distinct
\end{itemize}

\subsection{Binary Repetition Codes}
A \textbf{binary \(r\)-repetition code} encodes \(0 \to \overbrace{0 \cdots 0}^r\) and \(1 \to \overbrace{1 \cdots 1}^r\). \\

The binary \((2t + 1)\)-repetition code is \(t\)-error correcting.

The binary \(2t\)-repetition code is \((t- 1)\)-error correcting and \(t\)-error detecting.

\subsection{Information Rate and Redundancy}
The \textbf{information rate} \(R\) is given by,
\begin{itemize}
    \item For a code \(C\) of radix \(r\) and length \(n\), \(R = \dfrac{\log_r|C|}{n}\)
    \item For a \textbf{systematic code}, \(R = \dfrac{\text{\# information digits}}{\text{length of code}}\)
\end{itemize}
We then define \textbf{redundancy} = \(\frac{1}{R}\).

\subsection{Binary Hamming Error-Correcting Codes}
A Binary Hamming \((n, k)\) code is a code of length \(n\) with \(k\) information bits, such that it is a single error correcting and has a parity check matrix, \(H\), of size \(n - k\) by \(n\).

\subsection{Hamming Distance, Weights}
The \textbf{weight} of an \(n\)-bit word \(\fx\) is defined to be
\[w(\fx) = \# \{i : 1 \leq i \leq n, x_i \neq 0\}.\]

Given two \(n\)-bit words, the \textbf{Hamming distance} between them is
\[d(\fx, \fy) = \# \{i : 1 \leq i \leq n, x_i \neq y_i \}.\]

Given some code with set of codewords \(C\), we define \textbf{(minimum) weight} of \(C\) to
\[w = w(C) = \min\{w(\fx) : \fx \in C, \fx \neq \fzero\}.\]

Similarly, the \textbf{(minimum) distance of \(C\)} is defined by
\[d = d(C) = \min\{d(\fx, \fy) : \fx, \fy \in C, \fx \neq \fy \}.\]

If \(\fx \rightsquigarrow \fy\), then \(d(\fx, \fy)\) is the number of errors in \(\fy\).
\section{Number Theory and Algebra}
\subsection{Revision of Discrete Mathematics}
\begin{multicols}{2}
    \begin{itemize}
        \item Division Algorithm
        \item (Extended) Euclidean Algorithm
        \item Bezout's Identity
        \item Inverses
        \item Chinese Remainder Theorem
    \end{itemize}
\end{multicols}

\subsection{Number Theory Results}
Given \(m \in \bZ^+\), the set of invertible elements in \(\bZ_m\) is denoted by
\[\bU_m = \{ a \in \bZ_m : \gcd(a, m) = 1\}\]
and its elements are the \textbf{units} in \(\bZ_m\).

Euler's \textbf{phi-function} is defined by
\[\phi(m) = |\bU_m|.\]

\paragraph{Formula for \phi(m)}
\begin{enumerate}
    \item If \(\gcd(m, n) = 1\), then \(\phi(mn) = \phi(m)\phi(n)\).
    \item For a prime \(p\) and \(\alpha \in \bZ^+\), we have \(\phi(p^\alpha) = p^\alpha - p^{\alpha - 1}\).
    \item Hence, if \(m = p_1^{\alpha_1}p_2^{\alpha_2} \cdots p_r^{\alpha_r}\) is the prime factorisation of \(m\), then
          \[\phi(m) = (p_1^{\alpha_1} - p_1^{\alpha_1 - 1})(p_2^{\alpha_2} - p_2^{\alpha_2 - 1}) \cdots (p_r^{\alpha_r} - p_r^{\alpha_r - 1}).\]
\end{enumerate}

\paragraph{Primitive Element Theorem}
Given a prime \(p\), there exists \(g \in \bU_p\) such that
\[\bU_p = \{g^0 = 1, g, g^2, \dots, g^{p - 2} \quad \text{ and } \quad g^{p - 1} = 1.\]

\paragraph{Primitve Powers}
If \(g\) is primitive in \(\bZ_p\), then \(g^k\) is primitive if and only if \(\gcd(k, p - 1) = 1\) and hence there are \(\phi(p - 1)\) primitive elements in \(\bZ_p\).

\paragraph{Euler's Theorem}
If \(\gcd(a, m) = 1\), then \(a^{\phi(m)} \equiv 1 \pmod m\).

\paragraph{Corollary} If \(\gcd(a, m) = 1\), then \(\mathrm{ord}_m(a) \mid \phi(m)\).

\paragraph{Fermat's Little Theorem}
For prime \(p\) and any \(a \in \bZ\), \(a^p \equiv a \pmod p\)

\subsection{Finite Fields}
\paragraph{Finite Field Theorem}
If \(p\) is prime, \(m(x)\) a monic irreducible in \(\bZ_p[x]\) of degree \(n\), and \(\alpha\) denotes a root of \(m(x) = 0\), then
\begin{enumerate}
    \item \(\bF = \bZ_p[x] / \langle m(x) \rangle\) is a field,
    \item \(\bF\) is a vector space of dimension \(n\) over \(\bZ_p\),
    \item \(\bF\) has \(p^n\) elements,
    \item \(\{\alpha^{n - 1}, \alpha^{n - 2}, \dots, \alpha, 1 \}\) is a basis for \(\bF\),
    \item \(\bF = \bZ_p (\alpha)\) i.e. the smallest field containing \(\bZ_p\) and \(\alpha\),
    \item there exists a primitive element \(\gamma\) of order \(p^n - 1\) for which \(\bF = \{0, 1, \gamma, \gamma^2, \dots, \gamma^{p^{n - 2}}\}\),
\end{enumerate}
if a field \(\bF\) has a finite number of elements, then \(|\bF| = p^n\) where \(p\) is prime, and \(\bF\)is isomorphic to \(\bZ_p[x] / \langle ,(x) \rangle\). Hence ALL fields with \(p^n\) elements are isomorphic to one another.

\subsection{Primality Testing}
\paragraph{Lucas' Test}
If there exists an \(a\) with \(\gcd(a, n) = 1\) and \(ord_n(a) = n - 1\), then \(n\) is prime.

\paragraph{Miller-Rabin Test}
Let \(p\) be prime. Then if \(a^2 = 1 \pmod p, a \equiv \pm 1 \pmod p\).
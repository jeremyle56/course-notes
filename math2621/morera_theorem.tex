\section{Morera's Theorem and Analytic Continuation}

\subsection{Morera's Theorem}
\paragraph{Morera's Theorem}
Suppose that \(\Omega\) is a domain, that hte function \(f: \Omega \to \bC\) is continuous, and that
\[\int_A f(z) dz = \int_B f(z) dz,\]
whenever the simple contours \(A\) and \(B\) have the same initial point and the same final point. Then \(f\) is holomorphic in \(\Omega\).

\paragraph{Holomorphic Extension}
Suppose that \(\Lambda\) is a (possibly infinite) line segment in an open set \(\Omega\) and \(\Omega \setminus \Lambda\) is open. If function \(f: \Omega \to \bC\) is continuous in \(\Omega\) and is holomorphic in \(\Omega \setminus \Lambda\), then \(f\) is holomorphic in \(\Omega\).

\subsection{Analytic Continuation}
\paragraph{\(f\) is 0 for Ball in Ball}
Suppose that \(B(z_1, r_1) \subset B(w, r)\), that \(f \in H(B(w, R))\), and that \(f(z) = 0\) for all \(z \in B(z_1, r_1)\). Then \(f(z) = 0\) for all \(z \in B(w, R)\).

\paragraph{Theorem}
Suppose that \(\Upsilon\) is a nonempty open subset of a domain \(\Omega\) in \(\bC\), and that \(f \in H(\Omega)\). If \(f(z) = 0\) for all \(z\) in \(\Upsilon\), then \(f(z) = 0\) for all \(z\) in \(\Omega\).

\paragraph{Corollary}
Suppose that \(\Upsilon\) is a nonempty open subset of a domain \(\Omega\) in \(\bC\), and that \(f, g \in H(\Omega)\). If \(f(z) = g(z)\) for all \(z\) in \(\Upsilon\), then \(f(z) = g(z)\) for all \(z\) in \(\Omega\).
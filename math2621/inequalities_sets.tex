\section{Inequalities and Sets of Complex Numbers}
\subsection{Equalities and Inequalities}
\paragraph{Modulus Squared of a Sum}
For all complex numbers \(w\) and \(z\),
\[|w + z| ^2 = |w|^2 + 2\Re(w\conjz) + |z|^2.\]

\paragraph{Triangle Inequality}
For all complex numbers \(w\) and \(z\),
\[|w + z| \leq |w| + |z| \qquad \forall w,z, \in \bC.\]

\paragraph{Circle Inequality}
For all complex numbers \(w\) and \(z\),
\[\left||w| - |z|\right| \leq |w - z|.\]

\paragraph{Modulus of \(e^z\)}
If \(z \in \bC\), then
\[|e^z| = e^{\Re(z)}.\]

\paragraph{Modolus of \(e^z - 1\) inequality}
For all real numbers \(\theta\),
\[|e^{i\theta} - 1 | \leq |\theta|.\]

\subsection{Properties of Sets}
\paragraph{Open Ball}
The open ball with centre \(z_0\) and radius \(\epsilon\), written \(\ballz\), is the set \(\{z \in \bC: |z - z_0| < \epsilon\}\).

\paragraph{Punctured Open Ball}
The punctured open ball with centre \(z_0\) and radius \(\epsilon\), written \(B^\circ(z_0, \epsilon)\), is the set \(\{z \in \bC : 0 < |z - z_0 | < \epsilon\}\).

\paragraph{Interior, Exterior and Boundary Points}
Suppose that \(S \subseteq \bC\). For any point \(z_0\) in \(\bC\), there are three mutually exclusive and exhaustive possibilities:
\begin{enumerate}[label=(\arabic*)]
    \item When the positive real number \(\epsilon\) is sufficiently small, \(\ballz\) is a subset of \(S\), that is, \(\ballz \cap S = \ballz\). In this case, \(z_0\) is an interior point of \(S\).
    \item When the positive real number \(epsilon\) is sufficiently small, \(\ballz\) does not meet \(S\), that is, \(\ballz\ cap S = \emptyset\). In this case, \(z_0\) is an exterior point of \(S\).
    \item No matter how small the positive real number \(\epsilon\) is, neither of the above holds, that is, \(\emptyset \subset \ballz \cap S \subset \ballz\). In this case, \(z_0\) is a boundary point of \(S\).
\end{enumerate}

\paragraph{Open, Closed, Closure, Bounded, Compact, Region Sets}
Suppose that \(S \subseteq \bC\).
\begin{enumerate}[label=(\arabic*)]
    \item The set \(S\) is open if all its points are interior points.
    \item The set \(S\) is closed if it contains all of its boundary points, or equivalently, if its complement \(\bC \setminus S\) is open.
    \item The closure of the set \(S\) is the set consisting of the points of \(S\) together with the boundary points of \(S\).
    \item The set is bounded if \(S \subseteq \ball(0, R)\) for some \(R \in \bR^+\)
    \item The set \(S\) is compact if it is both closed and bounded.
    \item The set \(S\) is a region if it is an open set together with none, some, or all of its boundary points.
\end{enumerate}

\subsection{Arcs}
\paragraph{Polygonal Arc}
A polygonal arc is a finite sequence of finite directed line segments, where the end point of one line segment is the initial point of the next one.

\paragraph{Simple Closed Polygonal Arc}
A simple closed polygonal arc is a polygonal arc that does not cross itself, but the final point of the last segment is the initial point of the first segment.

\paragraph{Interior and Exterior Arc}
The complement of a simple closed polygonal arc is made up of two pieces: one, the interior of the arc, is bounded, and the other, exterior is not.

\paragraph{Polygonally Path-connectedness}
Let \(X \subseteq \bC\) be a subset of the complex plane.
\begin{enumerate}[label=(\arabic*)]
    \item The set \(X\) is polygonally path-connected if any two points of \(X\) can be joined by a polygonal arc lying inside \(X\).
    \item The set \(X\) is simply polygonally connected if it is polygonally path-connected and if the interior of every simple closed polygonal arc in \(X\) lies in \(X\), that is, if "\(X\) has no holes".
    \item The set \(X\) is a domain if it is open and polygonally path-connected.
\end{enumerate}
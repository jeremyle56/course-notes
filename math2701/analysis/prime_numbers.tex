\subsection{Prime Numbers}

\subsubsection{Infinitude of Primes}
\begin{theorem}[Fundamental Theorem of Arithmeitc]
    Every natural number \(n\) can be written uniquely, up to re-ordering of the factors, as a product of primes.
\end{theorem}

\begin{theorem}[Euclid]
    There are \(\infty\)-ly many primes. As \(n\) tends to infinity, we have
    \[\sum_{{p \leq n}_{p \in \bP}} \frac{1}{p} \to \infty.\]
\end{theorem}

\subsubsection{Elementary Estimates for the Growth of \(\pi(x)\)}

\begin{theorem}[Gauss]
    For \(x \geq 2\), we have \(\pi(x) \geq \log\log x\).
\end{theorem}

\subsubsection{Statement of the Prime Number Theorem}
\begin{theorem}
    There exists a constant \(c > 0\), effectively computable such that for \(x \geq 2\)
    \[\pi(x) = \mathrm{Li}(x) + O\left[x \exp(-c \sqrt{\log x})\right],\]
    where the implied constant is absolute.
\end{theorem}
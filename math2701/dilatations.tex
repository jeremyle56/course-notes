\section{Dilatations}

\begin{definition}
    A collineation \(\delta\) on \(\bR^n\) is called a \textbf{dilatation} if, for every line \(\ell\) in \(\bR^n, \ell \parallel \delta(\ell)\).
\end{definition}

\begin{proposition}
    The set \(\scrD(\bR^n)\) of all dilatations in \(\bR^n\) forms a subgroup of \(\scrC(\bR^n)\).
\end{proposition}

\begin{lemma}
    A dilatation that fixes two points is the identity map. Hence, for dilatations \(\delta_1, \delta_2\) and distinct point \(A, B, \) if \(\delta_1(A) = \delta_2(A)\) and \(\delta_1(B) = \delta_2(B), \) then \(\delta_1 = \delta_2\).
\end{lemma}

\begin{lemma}
    \begin{statements}{}
        \item If \(A, B, C\) are collinear, distinct, with \(\frac{CB}{CA} = r \neq 0\), then \(\delta_{C, r}(A) = B\).
        \item For collinear points \(A, B, P, P'\), if \(\frac{AP}{PB} = \frac{AP'}{P'B}\), then \(P = P'\).
        \item Let \(\tau\) be a dilatation and let \(\tau(P) = P'\) for every point \(P\). If there exist points \(A, B\) such that \(\overrightarrow{AB}\) and \(\overrightarrow{A'B'}\) have the same (resp., opposite) direction, then, for any points \(C, D, \overrightarrow{CD}\) and \(\overrightarrow{C'D'}\) have the same (resp., opposite) direction.
    \end{statements}
\end{lemma}

\begin{corollary}
    If points \(A, B, C\) are sent to \(A', B', C'\) under a dilatation, then
    \[\frac{AB}{A'B'} = \frac{BC}{B'C'} = \frac{CA}{C'A'}.\]
\end{corollary}

\begin{theorem}
    A dilatation is either a translation or a dilation. Hence, every dilatation is a similarity.
\end{theorem}
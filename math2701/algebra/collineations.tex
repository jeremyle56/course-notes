\subsection{Collineations}

\begin{theorem}
    A transformation is a collineation in \(\bR^n\) if and only if the images of collinear points are themselves collinear.
\end{theorem}

\begin{lemma}
    If \(\alpha\) is a collineation in \(\bR^n\), and \(l, m\) are parallel lines, then \(\alpha(l)\) and \(\alpha(m)\) are parallel.
\end{lemma}

\begin{theorem}
    A collineation takes the midpoint of points \(A, B\) to the midpoints of points \(\alpha(A), \alpha(B)\).
\end{theorem}

\begin{corollary}
    For a collineation \(\alpha\), if \(n + 1\) points \(P_0, P_1, \dots, P_n\) divide the segment \(\overline{P_0P_n}\) into \(n\) congruent segments \(\overline{P_{i-1}P_i}\), and \(P_i' = \alpha(P_i)\), then the \(n + 1\) points \(P_0', \dots, P_n'\) divide the segment \(\overline{P_0'P_n'}\) into \(n\) congruent segments \(\overline{P_{i-1}'P_i'}\).

    In particular, if a point \(P\) is between \(A\) and \(B\), and \(\frac{AP}{PB} = r\) is \textbf{rational}, then \(P' = \alpha(P)\) is between \(\alpha(A)\) and \(\alpha(B)\) and \(\frac{A'P'}{P'B'} = r\).
\end{corollary}
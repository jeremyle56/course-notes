\subsection{Classification of Plane Similarities}

\begin{definition}
    We say that figure \(f_1 \subseteq \bR^n\) and figure \(f_2 \subseteq \bR^n\) are \textbf{similar} if there is a similarity \(\alpha\) such that \(\alpha(f_1) = f_2\).
\end{definition}

\begin{theorem}
    If \(\triangle ABC \sim \triangle A'B'C'\) in \(\bR^2\), then there exists a \textbf{unique} plane similarity \(\alpha\) such that
    \[\alpha(A) = A', \alpha(B) = B', \alpha(C) = C'.\]
\end{theorem}

\begin{theorem}[Equations of Similarities]
    If \(\alpha\) is a similarity in \(\bR^n\), then there exist \(Q \in O_n(\bR), \fb \in \bR^n\) and \(r \in \bR_{> 0}\) such that
    \[\alpha(\fx) = rQ\fx + \fb, \quad \text{ for all } \fx \in \bR^n.\]
\end{theorem}

\begin{lemma}
    A similarity without a fixed point is an isometry.
\end{lemma}

\begin{definition}
    \begin{statements}{}
        \item A \textbf{stretch reflection} in \(\bR^2\) is a non-identity stretch about some point \(C\) followed by a reflection about a line through \(C\).
        \item A \textbf{stretch rotation} in \(\bR^2\) is a non-identity stretch about some point \(C\) followed by a non-identity rotation about \(C\).
    \end{statements}
\end{definition}

\begin{theorem}
    A non-identity plane similarity is exactly one of the following:
    \begin{center}
        Isometry, \quad Stretch of ratio \(r \neq 1\), \quad Stretch reflections, \quad Stretch rotation.
    \end{center}
\end{theorem}

\begin{theorem}
    In the equation of similarities, the algebraic classification is as follows:
    \begin{enumerate}
        \item \(\alpha\) is an isometry if \(r = 1\);
        \item \(\alpha\) is a stretch (of ratio \(r \neq 1\)) if \(r \neq 1\) and \(Q = I\);
        \item \(\alpha\) is a stretch reflection if \(r \neq 1, Q \neq I\) and \(\det(Q) = -1\);
        \item \(\alpha\) is a stretch rotation if \(r \neq 1, Q \neq I\) and \(\det(Q) = 1\);
    \end{enumerate}
\end{theorem}

\pagebreak
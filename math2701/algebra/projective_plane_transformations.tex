\subsection{Projective Plane Transformations}

\begin{theorem}
    Let \(f = \{ P,Q,R,S\}\) and \(P', Q', R', S'\) be two sets of four points, no three of which are collinear in \(\bR P^2\). Then there is a unique \(\pi \in \scrP\) such that \(\pi(P) = P', \pi(Q) = Q', \pi(S) = S'\).
\end{theorem}

\begin{proposition}
    For two figures \(f, g \subseteq \bR^n\), if they are affine equivalent, then the images \(f, g\) in \(\bR P^n\) are projective equivalent.
\end{proposition}

\begin{theorem}
    All non-degenerate conic sections are projective equivalent.
\end{theorem}

\begin{definition}
    A bijective map \(\tau: \bR P^2 \to \bR P^2\) is called a \textbf{(projective) collineation} if \(\tau\) takes collinear points to collinear points. (Equivalently, \(\tau\) sends any projective line to a projective line.)
\end{definition}

\begin{lemma}
    If \(\tau\) is a collineation of \(\bR P^2\) and \(\tau\) fixes points
    \[P_1 = [1, 0, 0], P_2 = [0, 1, 0], P_3 = [0, 0, 1], Q = [1,1,1],\]
    then \(\tau\) is the \textbf{identity} map.
\end{lemma}

\begin{theorem}
    A bijective map \(\tau\) on \(\bR P^2\) is a projective collineation if and only if \(\tau\) is a projective transformation.
\end{theorem}

\pagebreak
\section{Subgroups and the Group of Isometries}

\begin{lemma}
    If \(S\) is a subset of a group \((G, *)\), then \(\langle S \rangle = \bigcap_{S \subseteq H \leq G} H\). In other words, \(\langle S \rangle\) is the \textbf{smallest} subgroup of \(G\) that contains all the elements of \(S\).
\end{lemma}

\begin{definition}
    We call \(\langle S \rangle\) the \textbf{subgroup of \(G\) generated by \(S\)}. A group generated by one element is called a \textbf{cyclic group}.
\end{definition}

\paragraph{Notation:}
\begin{itemize}
    \item space: \(\bR^n\);
    \item points: \(A, B, C, P, Q, R, \dots\) with position vectors \(\mathbf{a, b, c, p , q, r} \dots\);
    \item transformations: \(\tau, \pi, \sigma, \delta, \dots\);
    \item lines: \(l, m, n, \dots\); line equations in \(\bR^n: \mathbf{x} = \mathbf{a} + \lambda \mathbf{v}\) for all \(\lambda \in \bR\);
    \item planes in \(\bR^n = \mathbf{x} = \mathbf{a} + \lambda \mathbf{u} + \mu \mathbf{v}\) for all \(\lambda, \mu \in \bR\);
    \item \textbf{Hyperplanes} through \(\mathbf{a} \in \bR^n\) with normal \(\mathbf{n} \in \bR^n = \mathbf{0}\):
          \[\mathbb{H}_{\mathbf{n, a}} = \{ \mathbf{x} \in \bR^n \mid (\mathbf{x} - \mathbf{a}) \cdot \mathbf{n} = 0 \} = \langle \mathbf{n} \rangle^\perp + \mathbf{a}. \]
    \item For points \(P, Q\) in \(\bR^n\), we may also define the \textbf{perpendicular bisector} of the line segment \(\bar{PQ}\) to be the hyperplane \(\mbb{H}\) that passes through the midpoint of \(\bar{PQ}\) and perpendicular to \(\bar{PQ}\). So \(\mbb{H}\) has the equation \(\mbf{(x - m) \centerdot (p - q)} = 0\) where \(\mbf{m = \frac{1}{2} (p + q)}\).
    \item It is clear that, for all \(X \in \mbb{H},\)
          \[d(X, P) = \sqrt{\norm{\mbf{x - m}}^2 + \norm{\mbf{p - m}}^2} = \sqrt{\norm{\mbf{x - m}}^2 + \norm{\mbf{q - m}}^2} = d(X, Q).\]
\end{itemize}

\paragraph{The Euclidean space \(\bR^n\)}
\begin{itemize}
    \item Length of a vector: \(\norm{\mbf{a} = \sqrt{\mbf{a \centerdot a}}}\);
    \item Distance between two points \(P, Q: d(P, Q) := \norm{\mbf{p - q}}\);
    \item Projection of \(\mbf{a}\) on \(\mbf{b}\): \(\operatorname{proj}_{\mbf{b}}(\mbf{a}) = \frac{\mbf{a} \cdot \mbf{b}}{\mbf{b} \cdot \mbf{b}} \mbf{b}\);
    \item Angle between \(\mbf{a}\) and \(\mbf{b}\): \(\cos(\theta) = \frac{\mbf{a} \cdot \mbf{b}}{\norm{\mbf{a}}\norm{\mbf{b}}}\);
    \item Orthogonality: \(\mbf{a} \perp \mbf{b} \iff \mbf{a} \centerdot \mbf{b} = 0\);
\end{itemize}

\begin{definition}
    An \textit{isometry} on \(\bR^n\) is a map \(\tau : \bR^n \to \bR^n\) which preserves distance between points: \(d(P, Q) = d(\tau(P), \tau(Q)), \forall P, Q \in \bR^n\).
\end{definition}

\begin{lemma}
    The set of isometries which fix the zero vector is equal to the set of (linear) maps that represent multiplication by an orthogonal matrix.
\end{lemma}

\begin{theorem}
    An isometry can be decomposed into a translation multiplied by a linear transformation, which can be represented by an orthogonal matrix. In other words, for every \(\tau \in \mscr{I}(\bR^n)\), there exist an orthogonal \(n \times n\) matrix \(Q\) and a vector \(\mathbf{b} \in \bR^n\) such that \(\tau = T_{Q, \mbf{b}} = T_{I, \mbf{b}} \circ T_{Q, \mbf{0}}\). In particular, an isometry is a \textbf{transformation}.
\end{theorem}

\begin{theorem}
    The group of Isometries
    \begin{statements}{}
        \item The set \(\mscr{I}(\bR^n)\) of all isometries forms a subgroup of the group \(\mscr{B}(\bR^n)\) of all transformations.
        \item The group \(\mscr{I} = \mscr{I}(\bR^n)\) contains two subgroups: the group \(\mscr{T}\) of translations and the group \(\mscr{O}\) of all orthogonal linear transformations. Moreover, we have \(\mscr{I} = \mscr{T}\mscr{O} := \{\tau \sigma \mid \tau \in \mscr{T}, \sigma \in \mscr{O} \}\).
    \end{statements}
\end{theorem}

\newpage
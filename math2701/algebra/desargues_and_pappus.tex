\subsection{Desargues' Theorem and Pappus' Theorem}

\begin{proposition}
    \begin{statements}{}
        \item Three distinct projective points \(P = \langle \fp \rangle, Q = \langle \fq \rangle,\) and \(R = \langle \mathbf{r} \rangle\) in \(\bR P^n\), are collinear if and only if the vectors \(\fp, \fq, \fr\) are linearly dependent. Moreover, we may choose \(\fp, \fq, \fr\) to satisfy \(\fp = \fq + \fr\).
        \item Four distinct projective points \(P = \langle \fp \rangle, Q  = \langle \fq \rangle, R = \langle \fr \rangle\) and \(S = \langle \mathbf{s} \rangle\) in \(\bR P^n\), no three of which are collinear, are coplanar if and only if the vectors \(\fp, \fq, \fr, \mathbf{s}\) are linearly dependent. Moreover, we may choose \(\fp, \fq, \fr, \mathbf{s}\) to satisfy \(\fp = \fq + \fr + \mathbf{s}\).
    \end{statements}
\end{proposition}


\begin{theorem}[Desargues' Theorem]
    Let \(A, B, C, A', B', C'\) be distinct points in \(\bR P^2\), such that the projective lines \(p\ell(A, A'), p\ell(B, B'), p\ell(C, C')\) are \textbf{distinct} and \textbf{concurrent}. Then the projective points of intersections \(C'' = p\ell(A, B) \cap p\ell(A', B'), A'' = p\ell(B, C) \cap p\ell(B', C'), B'' = p\ell(A, C) \cap p\ell(A', C')\) are \textbf{collinear}.
\end{theorem}

\paragraph{Dual Desargues' Theorem}
Let \(l, m, n, l', m', n'\) be distinct \textbf{lines} in \(\bR P^2\) such that their intersections \(l \cap l', m \cap m', n \cap n'\) are distinct projective points, and collinear. Then the projective lines joining \(l \cap m, l' \cap m'\), and \(m \cap n, m' \cap n\), and \(n \cap l, n' \cap l'\) are concurrent.

\begin{theorem}[Pappus' Theorem]
    Let \(A, B, C\) and \(A', B', C'\) be two pairs of collinear triples of distinct points in a projective plane. Then the three points \(A'' = p\ell(B, C') \cap p\ell(B', C), B'' = p\ell(C, A') \cap p\ell(C', A)\) and \(C'' = p\ell(A, B') \cap p\ell(A' ,B)\) are collinear.
\end{theorem}

\paragraph{Dual Pappu's Theorem}
Let \(l, m, n, l', m', n'\) be two pairs of concurrent projective lines in \(\bR P^2\). Then the projective lines \(p\ell(m \cap n', m' \cap n), p\ell(n' \cap l, n \cap l'), p\ell(l \cap m', l' \cap m)\) are concurrent.
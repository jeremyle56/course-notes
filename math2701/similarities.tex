\section{Similarities}

\begin{definition}
    A transformation \(\alpha: \bR^n \to \bR^n\) is called a \textbf{similarity of ratio} \(r > 0\) if \[d(\alpha(P), \alpha(Q)) = rd(P, Q), \text{ for all } P, Q \in \bR^n.\]
\end{definition}

\begin{proposition}
    \begin{statements}{}
        \item An isometry is a similarity of ratio 1.
        \item A similarity fixing two points is an isometry.
        \item A similarity fixing \(n + 1\) points in generic position is the identity.
        \item The set of all similarities in \(\bR^n\) forms a group, denote this set by \(\scrS\) or \(\scrS(\bR^n)\).
    \end{statements}
\end{proposition}

\begin{definition}
    A \textbf{stretch of ratio} \(r > 0\) about point \(C\) is a transformation \(\delta_{C, r}\) that fixes \(C\) and otherwise sends a point \(P\) to a point \(P'\), where \(P'\) is the unique point on the \textbf{ray} from \(C\) through \(P\) such that \(d(C, P') = r \cdot d(C, P)\).
\end{definition}

% \begin{proposition}
%     For any \(r, s \in \bR_{> 0}\), and any \textbf{fixed} point \(C\), we have
%     \begin{statements}{}
%         \item \(\delta_{C, r}\delta_{C, s} = \delta_{C, rs}\);
%         \item \(\delta_{C, r}^{-1} = \delta_{C, r^{-1}}\).
%     \end{statements}

%     Hence, \(\{ \delta_{C, r} \mid r \in \bR_{> 0} \}\) forms a group that is isomorphic to the group \((\bR_{> 0}, \cdot)\).
% \end{proposition}

\begin{theorem}{Decomposition of a similarity}
    If \(\alpha\) is a similarity of ratio \(r > 0\), and \(P\) is any \textbf{fixed} point, then \(\alpha = \tau\delta_{P, r} = \delta_{P, r}\tau'\), for some isometries \(tau, \tau'\). Moreover, we have
    \[\scrS = \bigsqcup_{r > 0} \scrI S_{P, r} = \bigsqcup_{r > 0} S_{P, r} \scrI \text{ (disjoint unions)},\]
    where \(\scrI S_{P, r} = \{\tau S_{P ,r } \mid \tau \in \scrI \}\) and \(\S_{P, r} \scrI = \{ S_{P, r} \tau \mid \tau \in \scrI\}\).
\end{theorem}

\begin{corollary}
    A similarity is a \textbf{collineation} that preserves betweenness, midpoints, angles, perpendicularity, etc.
\end{corollary}

\begin{definition}
    \begin{statements}{}
        \item A \textbf{point reflection} about \(C(\mbf{c})\) is the isometry \(\rho_C : \bR^n \to \bR^n\) defined by
        \[\rho_C(\fx) = -(\fx - \mbf{c}) + \mbf{c} = -\fx + 2\mbf{c}.\]
        \item A \textbf{dilation} about the point \(C\) is a stretch transformation \(\delta_{C ,r} (r > 0)\) about \(C\), or it is a stretch transformation followed by a point reflection both about \(C\) (i.e., \(\rho_C \delta_{C ,r}\)).
    \end{statements}
\end{definition}

\begin{lemma}
    \begin{statements}{}
        \item A point reflection is an isometry.
        \item The product of two point reflections is a translation.
        \item The product of a translation and a point reflection is a point reflection.
    \end{statements}
\end{lemma}

\begin{proposition}
    All point reflections generate a subgroup \(\mathscr{H} (of \scrI)\). Moreover, \(\mathscr{H}\) is a (disjoint) union of the set \(\mathscr{T}\) of all translations and the set of all point reflections: for a fixed \(C\),
    \[\mathscr{H} = \mathscr{T} \sqcup \rho_C \mathscr{T} = \mathscr{T} \sqcup = \mathscr{T} \rho_C = \mathscr{T} \sqcup \{\rho_P \mid P \in \bR^n\}.\]
\end{proposition}

% \begin{corollary}
%     The set \(\{\mathscr{T}, \rho_C \mathscr{T}\}\) with the subset multiplication becomes a group.
% \end{corollary}

\begin{proposition}
    The dilation \(\tau = \rho_C \delta_{C, r} (r > 0)\) has the following equation:
    \[\tau(\fx) = (-r)\fx + (1 + r)\mbf{c}.\]
\end{proposition}

\begin{lemma}
    Let \(R^\times = \{ r \in \bR \mid r \neq 0\}\). For any \(r, s \in \bR^\times\), and any point \(P(\mbf{p})\), we have
    \begin{statements}{}
        \item \(\delta_{P, -r} = \rho_O \delta_{P, r}\);
        \item \(\delta_{P, 1} = 1, \delta_{P, -1} = \rho_P\);
        \item \(\delta_{P, r} \delta_{P, s} = \delta_{P, rs}\);
        \item \(\delta_{P ,r}^{-1} = \delta_{P, r^{-1}}\).
    \end{statements}
\end{lemma}

\begin{proposition}
    The set \(\{ \delta_{C, r} \mid r \in \bR^\times(:= \bR - 0)\}\) forms a group that is isomorphic to the group \((\bR^\times, \cdot)\).
\end{proposition}
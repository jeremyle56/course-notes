\documentclass[12pt, letterpaper]{article}

\usepackage[margin=1in]{geometry}
\usepackage{amsmath}
\usepackage{amssymb}
\usepackage{bigints}
\usepackage{enumitem}
\usepackage{mathtools}
\usepackage{fancyhdr}
\usepackage[utf8]{inputenc}
\usepackage{dirtytalk}                      % \say command for quotes
\usepackage{ wasysym }
\usepackage{graphicx}
\usepackage{physics}
\usepackage{esint}
\usepackage{subfiles}


\usepackage{hyperref}
\hypersetup {                                % Formatting for hyperlinks
    colorlinks,
    citecolor=black,
    filecolor=black,
    linkcolor=black,
    urlcolor=black
}

\graphicspath{'./assets}

\pagestyle{fancy}
\setlength{\headheight}{15pt}
\renewcommand{\headrulewidth}{0pt}
\renewcommand{\footrulewidth}{0pt}	
\author{Jeremy Le \\ (Based of \href{https://github.com/imagine-hussain/math2111_notes/blob/main/math2111.pdf}{Hussain Nawaz's Notes})}\title{Higher Several Variable Calculus \\ Math2111 UNSW}

\date{2023T1}

\rhead{}
\lhead{}


\begin{document}
\maketitle
\tableofcontents
    \newpage
    \subfile{introduction.tex}
    \subfile{curves_and_surfaces.tex}
    \subfile{analysis.tex}
    \subfile{differentiability.tex}
    \subfile{integration.tex}
    \subfile{fourier_series.tex}
    \subfile{vector_fields.tex}
    \subfile{path_integrals.tex}
    \subfile{vector_line_integrals.tex}
    \subfile{surface_integrals.tex}
    \subfile{integral_theorems.tex}
\end{document}
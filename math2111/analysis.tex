
\section{Analysis}

\subsection{Formal Definition of a Limit}
\paragraph{1-variable Calculus}
Recall that \(\lim_{x\to a} f(x) = L\) requires that for all \(\epsilon > 0\),
there exists a \(\delta > 0 \) such that if \(|x-a| < \delta\)
then \[|f(x) - L| < \epsilon.\]

\subsection{Distance Functions (metrics)}
A function 
\(d: \mathbb{R}^n \times \mathbb{R}^n \to \mathbb{R}\)
which satisfies the following three properties is called a metric.
\begin{itemize}
    \item \textbf{Positive Definite}: for all \(x,y \in \mathbb{R}^n\), \(d(x,y) > 0\)
    and \(d(x, y) = 0\) iff \(x = y\).
    \item \textbf{Symmetric}: for all \(x,y \in \mathbb{R}^n\), \(d(x,y) = d(y, x)\).
    \item \textbf{Triangle Inequality} for all \(x,y,z \in \mathbb{R}^n\), 
    \(d(x,y) + d(y,z) \geq d(x, z)\).
\end{itemize}

\paragraph{Euclidean Distance} The Euclidean distance between \(x\) and \(y\) defined by 
\[d(x,y) = ||x - y|| = \sqrt{\sum_{i=1}^n (x_i - y_i)^2}\]
is a metric.

\paragraph{Equivalent Metrics}
Two metrics \(d\) and \(\delta\) are considered equal if 
there exists constants \(0 < c < C < \infty\) such that
\[ c\delta (x,y) \leq d(x,y) \leq C \delta(x, y).\]

\subsection{Limits of Sequences}
\paragraph{Ball} A ball around \(\textbf{a} \in \mathbb{R}^n\) of radius \(\epsilon > 0\)
is the set 
\[B(\textbf{a}, \epsilon) = 
\{ \textbf{x} \in \mathbb{R}^n : d(\textbf{a},\textbf{x}) < \epsilon \}.\]

\paragraph{Limit of Sequences} For a sequence \(\{\textbf{x}_i\}\) of points in \(\mathbb{R}^n\)
we say that \(\textbf{x}\) is the limit of the sequence if and only if 
\[\forall \epsilon > 0 \exists N \text{ such that } 
n \geq N \implies d(\textbf{x}, \textbf{x}_n) < \epsilon\]
or equivalently
\[\forall \epsilon > 0 \exists N \text{ such that } 
n \geq N \implies \textbf{x}_n \in B(\textbf{x}, \epsilon).\]
If \(\textbf{x}\) is the limit of the sequence \(\{\textbf{x}_i\}\) then for each positive
\(\epsilon\) there is a point in the sequence beyond which all points of the sequence are inside
\(B(\textbf{x}, \epsilon)\).

\paragraph{Convergence}
\begin{align*}
    \text{A} & \text{ sequence } \textbf{x}_k \text{ converges to a limit \(\textbf{x}\)}\\
    &\Leftrightarrow \text{the components of } \textbf{x}_k
    \text{ converge to the components of } \textbf{x} \\
    & \Leftrightarrow d(\textbf{x}_k, \textbf{x}) \rightarrow 0.
\end{align*}

\paragraph{Cauchy Sequences} A sequence \(\{\textbf{x}_k\}\) in \(\mathbb{R}^n\) is a Cauchy
sequence if 
\[\forall \epsilon > 0 \exists K \text{ such that } k, l > K \implies 
d(\textbf{x}_k, \textbf{x}_l) < \epsilon.\]

A sequence \(\{\textbf{x}_k\}\) converges in \(\mathbb{R}^n \text{ to a limit if and only if }
\{\textbf{x}_k\}\) is a Cauchy sequence.

\subsection{Open and Closed Sets}
\paragraph{Definitions}
Consider \(x_k\)
\begin{itemize}
    \item \(x_0\in\Omega\) is an \underline{interior point} of \(\Omega\) if there is a ball around \(x_0\) completely contained in \(\Omega\). That is, there exists a \(\epsilon > 0\) such that \(B(x_0, \epsilon) \subseteq \Omega\).
    \item  \(\Omega\) is \underline{open} if every point of \(\Omega\) is an interior point.
    \item \(\Omega\) is \underline{closed} if its complement is open.
    \item \(x_0\in\Omega\) is a \underline{boundary point} of \(\Omega\) if every ball around \(x_0\) contains points in \(\Omega\) and points not in \(\Omega\).
\end{itemize}

\paragraph{Closed Sets}
\(\Omega\subset\mathbb{R}^n\) is closed if and only if it contains all of its boundary points.

\paragraph{Union and Intersection}
\begin{itemize}
    \item A finite union/intersection of open sets is open.
    \item A finite union/intersection of closed sets is closed.
\end{itemize}

\paragraph{Limit Points and Sets}
\(\textbf{x}_0\) is a limit point (or accumulation point) of \(\Omega\) if there is a sequence \(\{\textbf{x}_i\}\) 
in \(\Omega\) with limit \(\textbf{x}_0\) and \(\textbf{x}_i \neq \textbf{x}\).

\begin{itemize}
    \item Every interior points of \(\Omega\) is a limit point of \(\Omega\).
    \item \(\textbf{x}_0\) is not necessarily in \(\Omega\).
    \item A set is closed \(\Leftrightarrow\) it contains all of its limit points.
\end{itemize}

\paragraph{Variations of a Set}
Consider the set \(\Omega \in \mathbb{R}^n\).
\begin{itemize}
    \item The \underline{interior} of \(\Omega\) is the set of all its interior points (denoted Int\((\Omega)\)).
    \item The \underline{boundary}  of \(\Omega\) is the set of all its boundary points (denoted \(\partial \Omega\)).
    \item The \underline{closure} of \(\Omega\) is \(\Omega \cup \partial \Omega\) (denoted by \(\bar{\Omega}\)).
\end{itemize}
The interior is the largest open subset and the closure is the smallest closed set containing \(\Omega\). \\

\subsection{Limits}
\paragraph{Limit of a Function at a Point} Let \(\textbf{b} \in \mathbb{R}^m, 
\Omega \subseteq \mathbb{R}^n, \textbf{a} \in \bar\Omega\) and let 
\(\textbf{f}: \Omega \rightarrow \mathbb{R}^m\) be a function. We say that 
\(\textbf{f}(\textbf{x})\) converges to \(\textbf{b}\) as \(\textbf{x}\to \textbf{a}\) if
\[\forall \epsilon > 0 \text{ } \exists \delta > 0 \text{ such that for } \textbf{x} \in \Omega:\]
\[0 < d(\textbf{x}, \textbf{x}_0) < \delta \implies 
d(\textbf{f}(\textbf{x}), \textbf{b}) < \epsilon.\]
or alternatively
\[\textbf{x} \in B(\textbf{a}, \delta) \cap \Omega \implies \textbf{f}(\textbf{x}) 
\in B(\textbf{b}, \epsilon).\]
If such \(\textbf{b}\) exists, then it is unique and we write 
\[\lim_{x\to a} \textbf{f}(\textbf{x}) = \textbf{b}.\]

\paragraph{Useful Limit Theorems}
Let \(\textbf{b} \in \mathbb{R}^m, \Omega \subseteq \mathbb{R}^n, \textbf{a} \in \bar\Omega\)
and let \(\textbf{f}:\Omega\rightarrow\mathbb{R}^m\) be a function. Then
\begin{align*}
    \lim_{\textbf{x}\to\textbf{a}} \textbf{f}(\textbf{x}) = \textbf{b} & \iff
    \lim_{\textbf{x}\to \textbf{a}}f_i(\textbf{x})=b_i \text{ for all } i=1,\dots,m \\
    \lim_{\textbf{x}\to \textbf{a}}\textbf{f}(\textbf{x}) = \textbf{b} & \iff
    \lim_{k\to\infty} \textbf{f}(\textbf{x}_k) = \textbf{b}
\end{align*}
for every sequence \(\{\textbf{x}_k\}_{k=1}^\infty \subseteq \Omega\) with 
\(\lim_{k \rightarrow\infty} \textbf{x}_k = \textbf{a}\). 

The first theorem is useful to show that a limit exists whilst the second is useful to
show the limit does not exist.

\paragraph{Algebra of limits}
Given that, \(\lim_{x\to x_0} f(x) = a\) and \(\lim_{x\to x_0} g(x) = b,\) then,
\begin{align*}
    \lim_{x\to x_0} (f + g)(x) & = a + b \\
    \lim_{x\to x_0} (fg)(x) & = ab \\
    \lim_{x\to x_0} (\frac{f}{g})(x) & = \frac{a}{b}, \text{ given } b \neq 0.
\end{align*}

\paragraph{Pinching Principle}
Let \(\Omega \subset \mathbb{R}^n\), let \(\textbf{a}\) be a limit point of \(\Omega\)
and let \(f,g ,h: \Omega \rightarrow \mathbb{R}\) be functions such that there exists
\(\epsilon > 0\) such that 
\[
    g(\textbf{x}) \leq f(\textbf{x}) \leq h(\textbf{x}) \qquad
    \forall \textbf{x} \in B(\textbf{a}, \epsilon) \cap \Omega.
\] 
Then
\[
    \lim_{x\to \textbf{a}} g(\textbf{x}) = \textbf{b} = \lim_{x\to \textbf{a}} h(\textbf{x}) 
    \implies \lim_{x\to \textbf{a}} f(\textbf{x}) = \textbf{b}.
\]

\subsection{Continuity}
Continuity is like an extension to limits. It first requires that the limit exists and that
the limit equals the actual value at that point.
\paragraph{Definition}
Let \(\textbf{a}\in\Omega \subseteq\mathbb{R}^n\) and let \(f:\Omega\rightarrow\mathbb{R}^m\) 
be a function. Then \(f\) is continuous at \(\textbf{a}\) if and only if 
\[\lim_{x\to a} f(\textbf{x}) = f(\textbf{a})\]
\(f\) is said to be continuous on \(\Omega\) if it is continuous at \(\textbf{a}\)
for every \(\textbf{a}\in\Omega\).


\paragraph{Epsilon-Delta Interpretation}
\[
    \text{For all } \epsilon > 0 \text{ there exists } \delta > 0 \text{ such that if }
    x\in B(\textbf{a}, \delta) \cap \Omega \implies f(x) \in B(f(\textbf{a}), \epsilon).
\]

\paragraph{Continuity by Components} 
All component functions \(f_i: \Omega \rightarrow \mathbb{R}\) are continuous at \(\textbf{a}\).

\paragraph{Continuity through Sequences}
For every sequence \(\{\textbf{x}_k\}_{k=1}^\infty\) with \(\textbf{x}_k \in \Omega\) for all
\(k\), if \(\{\textbf{x}_k\}_{k=1}^\infty\) has limit \(\textbf{a}\) then 
\(\{f(\textbf{x}_k)\}_{k=1}^\infty\) converges to \(f(a)\).

\paragraph{Elementary Functions}
If \(f: \Omega \subseteq \mathbb{R}^n \rightarrow \mathbb{R}\) is an elementary function,
then \(f\) is continuous on \(\Omega\).

\paragraph{Preimage}
Suppose that  \(\Omega \subseteq \mathbb{R}^n\) and  \(f: \Omega \rightarrow \mathbb{R}^m\) 
is a function. The preimage of a set \(U \subseteq \mathbb{R}^m\) is defined by 
\[ f^{-1}(U) = \{ x \in \mathbb{R}^n: f(x) \in U \}.\]

\paragraph{Continuity - Using Preimage} 
Suppose that \(f: \Omega \subset \mathbb{R}^n \rightarrow \mathbb{R}^m\). The following two
statements are equivalent.
\begin{itemize}
    \item \(f\) is continuous on \(\Omega\).
    \item \(f^{-1}(U)\) is open in \(\mathbb{R}^n\) for every open subset \(U\) of \(\mathbb{R}^m\).
\end{itemize}

\subsection{Path Connected Sets}
\paragraph{Definition} A set \(\Omega \subseteq \mathbb{R}^n\) is said to be path connected
if for any \(\textbf{x}, \textbf{y} \in \Omega\), there is a continuous function
\(\varphi\) such that \(\varphi(t) \in \Omega\) for all \(t \in [0, 1]\) and 
\(\varphi(0) = \textbf{x}\) and \(\varphi(1) = \textbf{y}\).

\paragraph{Theorem} Let \(\Omega \subseteq \mathbb{R}^n\) and 
\(\textbf{f}: \Omega \rightarrow \mathbb{R}^m\) be continuous. Then
\[
    B \subseteq \Omega \text{ and } B \text{ path connected } \implies \textbf{f}(B)
    \text{ path connected}.
\]

\subsection{Compact Sets}
\paragraph{Bounded} A set \(\Omega \subseteq \mathbb{R}^n\) is bounded if there is
an \(M \in \mathbb{R}\) such that \(d(\textbf{x}, \textbf{0}) \leq M\) for all 
\(\textbf{x} \in \Omega \Longleftrightarrow \Omega \subseteq B(\textbf{0}, M)\) .

\paragraph{Compact} A set \(\Omega \subseteq \mathbb{R}^n\) is compact if it is closed
and bounded.

\paragraph{Theorem} Let \(\Omega \subseteq \mathbb{R}^n\) and 
\(f: \Omega \rightarrow \mathbb{R}^m\) be continuous. Then
\[
    K \subseteq \Omega \text{ and } K \text{ compact } \implies f(K) \text{ compact.}
\]

\subsection{Bolzano-Weierstrass Theorem}
For \(\Omega \subseteq \mathbb{R}^n\), the following are equivalent.
\begin{enumerate}
    \item \(\Omega\) is compact.
    \item Every sequence in \(\Omega\) has a subsequence that converges to an element of 
    \(\Omega\). 
\end{enumerate}

\section{Introduction to Groups}
\subsection{Fields}

\defn{Field}{
    A field \(K\) is a commutative ring with identity in which every non-zero element has a multiplicative inverse.
}

\prop{No Divisors of Zero in Fields}{
    A field contains no divisors of zero.
}

\prop{All fields are Integral Domains}{
    A field is an integral domain.
}

\thrm{Inverse and GCD}{
    An element \(n \in \bZ_m^*\) has an inverse if and only if \(\gcd(m , n) = 1\).
}

\thrm{Rings and Fields}{
    The ring \(Z_m\) is a field if and only if \(m\) is prime.
}

\subsection{Units of a Ring}
\paragraph{Notation for Set of Units} In \(\bZ_m\), we denote the set of units by \(\bU_m\).

\thrm{Units in Commutative Rings with Identity}{
    Let \(R\) be a commutative ring with identity.
    \begin{enumerate}[label=\alph*)]
        \item 1 is a unit of \(R\)
        \item If \(a\) and \(b\) are units in \(R\), then so is their product \(ab\).
        \item If \(a\) is a unit in \(R\) then so is \(a^{-1}\)
    \end{enumerate}
}

\prop{Units are Closed in Commutative Rings with Identity}{
    In any commutative ring with identity, the set of all units is closed under multiplication and inverse.
}

\subsection{Groups}
\defn{Groups}{
    A group is a non-empty set \(G\) on which an operation \(\star\) is defined, such that the following properties hold:
    \begin{enumerate}
        \item Closure: if \(a, b \in G\) then \(a \star b \in G\).
        \item Associativity: if \(a, b, c \in G\) then \((a \star b) \star c = a \star (b \star c)\).
        \item Identity element: there is an element \(e\) of \(G\) such that for all \(a \in G\) we have \(a \star e = e \star a = a\)
        \item Inverses: for each \(a\) in \(G\) there is an element \(b\) of \(G\) such that \(a \star b = b \star a = e\). This element is usually denoted \(a^{-1}\).
    \end{enumerate}
}

\paragraph{Abelian Groups}
If the operation is commutative, i.e. \(a \star b = b \star a\), for all \(a, b \in G\), the group is called commutative, or Abelian.

\prop{Properties of Groups}{
    In any group \(G\) the following properties hold.
    \begin{enumerate}[label=\alph*]
        \item There is only one identity element in \(G\).
        \item Each \(x\) in \(G\) has only one inverse.
        \item If \(x,y \in G\) then \((xy)^{-1} = y^{-1}x^{-1}\).
        \item If \(x, y, z \in G\) and \(xy = xz\) then \(y = z\).
    \end{enumerate}
}

\subsection{Group Isomorphism}
\defn{Group Isomorphism}{
    Let \(G\) and \(H\) be groups with operations \(\star\) and \(\bullet\) rspectively. An isomorphism for \(G\) to \(H\) is a bijective function \(\psi: G \to H\) with the property that
    \[\psi(a \star b) = \psi(a) \bullet \psi(b) \quad \text{ for all elements } a, b \in G.\]
    The groups \(G\) and \(H\) are said to be isomorphic if there exists such a function.

    We write \(G \cong H\) to indicate that \(G\) and \(H\) are isomorphic.
}

\thrm{Identities and Inverses in Isomorphic Groups}{
    Suppose that \(G\) and \(H\) are groups with identities \(e_G\) and \(e_H\), respectively. Let \(\psi: G \to H\) be a group isomorphism. Then
    \begin{enumerate}
        \item \(\psi(e_G) = e_H\),
        \item \(\psi(a^{-1}) = (\psi(a))^{-1}\) for all \(a \in G\),
        \item \(\psi(a^n) = \psi(a)^n\) for all \(n \in \bZ\),
        \item If \(\psi: G \to H, \theta: H \to K\) homomorphic then \(\theta \circ \psi: G \to K\) is also homomorphic,
        \item If \(\psi: G \to H\) is a isomorphic then \(\psi^{-1}: H \to H\) is also isomorphic.
    \end{enumerate}
}

\subsection{Wilson's Theorem}
\thrm{Wilson's Theorem}{
    Let \(p \geq 2\). Then \(p\) is prime if and only if \((p - 1)! \equiv -1 \pmod p\).
}
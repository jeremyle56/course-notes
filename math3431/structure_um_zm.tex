\section{The Structure of \texorpdfstring{\(\bU_m\)}{Um} and \texorpdfstring{\(\bZ_m\)}{Zm}}
\subsection{Subgroups and Cyclic Groups}

\defn{Subgroup}{
    Let \(G\) be a group, and let \(H\) be a subset of \(G\) which is itself a group under the same operations as \(G\). Then we say that \(H\) is a subgroup of \(G\).
}

\prop{The Subgroup Lemma}{
    Let \(G\) be a group and \(H\) a non-empty subset of \(G\). Then \(H\) is a subgroup of \(G\) if and only if it is closed under the group operation and inverse.
}

\defn{Cyclic Groups}{
    A group \(G\) is said to be cyclic if there exists an element \(g \in G\) such that \(G = \langle g \rangle\), i.e. \(G\) is generated by a single element.
}

\defn{Order of a Group and Element}{
    The order of a finite group \(G\) is the number of elements in \(G, \abs*{G}\). \\
    The order of an element \(g\) in a group \(G\) is the smallest positive integer \(n\) (if any) such that \(g^n = e\). We write \(o(g)\) for the order of the element \(g\).
}

\prop{Distinct Powers of Elements}{
    If \(g \in G\) has order \(n\), then the elements \(e, g, g^2, \dots, g^{n-1}\) are all distinct.
}

\thrm{Isomorphic Cyclic Groups}{
    Two finite cyclic groups are isomorphic if and only if they have the same order.
}

\thrm{Prime Order is Isomorphic is Cycle Group}{
    Any group of prime order \(p\) is isomorphic to the cyclic group \(C_p\).
}

\prop{Groups of Prime Order are Abelian}{
    Any group of prime order is abelian.
}

\thrm{All Subsets Closed under Operation are Subgroups}{
    Let \(G\) be a group with operation \(\star\). If \(H\) is a non-empty finite subset of \(G\) that is closed under \(\star\), then \(H\) is a subgroup of \(G\).
}

\thrm{Lagrange's Theorem}{
    If \(G\) is a finite group and \(H\) is a subgroup of \(G\), then \(\abs*{H}\) is a factor of \(\abs*{G}\).
}

\defn{Left Coset}{
    Let \(G\) be a group and \(H\) a subgroup of \(G\). For any \(g \in G\) we define the left coset of \(H\) by \(g\) to be
    \[gH = \{ gh \mid h \in H \}.\]
    If we used additive notation we would write a coset of \(H\) as \(g + H\).
}

\prop{Order of Elements in a Group is a Divisor of the Group}{
    Suppose \(G\) is a group of finite order and \(g \in G\). Then \(g^\abs*{G} = e\).
}

\thrm{Fermat's Little Theorem}{
    If \(p\) is a prime and \(a\) is not a multiple of \(p\), then \(a^{p-1} \equiv 1 \pmod p\).
}

\prop{Corollary of Fermat's Little Theorem}{
    If \(p\) is prime and \(a\) is any integer, then \(a^p \equiv a \pmod p\).
}

\thrm{Euler's Theorem}{
    Let \(n\) be a positive integer, and let \(a\) be an integer relatively prime to \(n\). Then \(a^{\varphi(n)} \equiv 1 \pmod n\)
}

\subsection{Direct Product of Groups}
\defn{Cartesian Product}{
    Let \(A\) and \(B\) be two sets. The Cartesian production of the two sets is defined by
    \[A \times B = \{(a, b) : a \in A, b \in B\}.\]
}

\prop{Cartesian Proudct of Any Sets is a Group}{
    Let \(H\) and \(K\) groups with operation \(\star\) and \(\times\), respectively. The set \(H \times K\) with the operation \(\bullet\) defined by
    \[(h_1, k_1) \bullet (h_2, k_2) = (h_1 \star h_2, k_1 \star k_2)\]
    is a group.
}

\defn{Cartesian Product of Groups}{
    The group in the above Lemma is called the direct product of \(H\) and \(K\) and it is denoted by \(H \otimes K\).
}

\thrm{Condition of Isomorphism for Cartesian Products}{
    Let \(G\) be a finite abelian group. If \(H\) and \(K\) are subgroups of \(G\) such that \(|H||K| = |G|\) and \(H \cap K = \{e\}\), then the mapping
    \[\psi: H \otimes K \to G, \quad \text{ where } \quad \psi((h, k)) = h,k\]
    is an isomorphism.
}

\defn{Direct Sum}{
    The direct sum of two additive abelian subgroups \(H\) and \(K\) is
    \[H \oplus K = \{(h, k) \mid h \in H \text{ and } k \in K\},\]
    with operation defined by
    \[(h_1, k_1) + (h_2, k_2) = (h_1 + h_2, k_1 + k_2).\]
}

\thrm{Decomposition of \(\bZ_n\)}{
    Suppose positive integer \(n\) factorises as \(n = st\). \\
    If \(s\) and \(t\) are relatively prime, then \(\bZ_n \cong \bZ_s \oplus \bZ_t\). \\
    Conversly, if \(s\) and \(t\) are not relatively prime, then \(\bZ_n \ncong \bZ_s \oplus \bZ_t\).
}

\thrm{Direct Sum Cyclic if Pairwise Relatively Prime}{
    Let \(s_1, s_2, \dots, s_k\) be positive integers. Then the direct sum \(\bZ_{s_1} \oplus \bZ_{s_2} \oplus \cdots \oplus \bZ_{s_k}\) is cyclic if and only if \(s_1, s_2, \dots, s_k\) are pairwise relatively prime.
}

\thrm{The Chinese Remainder Theorem}{
    Suppose that the integers \(m_1, m_2, \dots, m_t\) are pairwise coprime, and let \(b_1, b_2, \dots, b_t\) be any intergers. Then the simulataneous congruences
    \[x \equiv b_1 \pmod{m_1}, \quad x \equiv b_2 \pmod{m_2}, \quad \dots \quad , x \equiv b_t \pmod{m_t}\]
    have a unique solution modulo \(m_1m_2\cdots m_t\).
}

\subsection{Decomposition of \(\bU_m\)}
\thrm{Decomposition of \(\bU_m\)}{
    If \(n = st\) is a positive integer, \(\bU_n \cong \bU_s \otimes \bU_t\) if and only if \(s\) and \(t\) are coprime.
}

\prop{Euler Function Multiplicative}{
    The Euler's function \(\varphi\) is multiplicative.
}

\thrm{Formula for \(\varphi(n)\)}{
Let \(n\) be a positive integer with canonical factorisation \(n = p^{\alpha_1}_1p^{\alpha_2}_2\cdots p^{\alpha_s}_s\). Then
\[\varphi(n) = \prod_{k=1}^s\left(p^{a_k}_k - p^{a_k - 1}_k\right) = \prod_{k=1}^s\left(p_k - 1\right)p^{a_k-1}_k = n \prod_{p\mid n} \left(1 - \frac{1}{p}\right).\]
}

\prop{Quotient Depends on Prime Factors}{
    The quotietnt \(\varphi(n) / n\) depends on the prime factors of n, but not on their mulltiplicity.
}

\prop{GCD involving Phi}{
    Suppose that \(\gcd(s, t) = g.\) Then
    \[\varphi(g)\varphi(st) = g\varphi(s)\varphi(t).\]
}

\thrm{Cyclic Sets of Units}{
    Let \(p\) be a prime, \(p \neq 2\). Then \(\bU_{p^\alpha}\) is cyclic,
    \[\bU_{p^\alpha} \cong C_{(p-1)p^{\alpha-1}}.\]
    For powers of 2 we have \(\bU_2 \cong C_1, \bU_4 \cong C_2\) and
    \[\bU_{2^\alpha} \cong C_2 \otimes C_{2^{\alpha - 2}} \quad \text{ for } \quad a \geq 3.\]
}

\thrm{Condition for Cyclic Units}{
    Let \(n \geq 2\). Then \(\bU_n\) is cyclic if and only if \(n = 2, n = 4, n = p^\alpha\) or \(n = 2p^\alpha\), where \(p\) is an odd prime and \(\alpha\) is a positive integer.
}

\thrm{Generator Implies Order and Relatively Primes}{
    Suppose that \(G\) is a cyclic group of order \(n\) and that \(g\) is a generator of \(G\).
    Then
    \begin{enumerate}[label=\alph*)]
        \item for any integer \(\alpha\), the order of \(g^\alpha\) is \(n / \gcd(\alpha, n)\);
        \item \(g^\alpha\) generates \(G\) if and only if \(\alpha\) is relatively prime to \(n\).
    \end{enumerate}
}

\subsection{Primitive Roots}
\defn{Primitive Root Modulo}{
    A generator of \(\bU_m\) is called a primitive root of modulo \(m\).
}

\thrm{Existence of Primitive Root}{
    A primitive root modulo \(m\) will exist if and only if \(m = 2, m = 4, m= p^\alpha\) or \(m = 2p^\alpha\), where \(p\) is an odd prime.
}

\thrm{Powers are Primitive Roots if Co-prime to Order}{
    Let \(g\) be a primitve root modulo \(m\). Then \(g^\alpha\) is a primitive root modulo \(m\) iff \(\alpha\) is relatively prime to \(\varphi(m)\), that is iff \(a \in U_{\varphi(m)}\).
}

\prop{Number of Primitive Roots}{
    If there are any primitve roots modulo \(m\) then there are \(\varphi(\varphi(m))\) of them.
}

\defn{Discrete Logarithm}{
    The exponent \(\alpha\) above is called the discrete logarithm of a modulo \(m\) to the base \(g\), or the index of \(a\) modulo \(m\), relative to \(g\). We write
    \[\alpha = \log_g a \quad \text{ or } \quad \alpha = \ind_g a.\]
}

\thrm{Addition and Multiplication of Discrete Logarithm}{
    Let \(g\) be a primitve root in \(\bU_m\).
    \begin{enumerate}[label=\alph*)]
        \item If \(a, b \in \bU_m\) then \(\ind_g(ab) = \ind_g a+ \ind_g b \pmod{\varphi(m)}\).
        \item If \(a \in \bU_m\) and \(k \in \bZ\) then \(\ind_g(a^k) = k \ind_g a \pmod{\varphi(m)}\).
    \end{enumerate}
}
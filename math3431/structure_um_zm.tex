\section{The Structure of \texorpdfstring{\(\bU_m\)}{Um} and \texorpdfstring{\(\bZ_m\)}{Zm}}
\subsection{Subgroups and Cyclic Groups}

\defn{Subgroup}{
    Let \(G\) be a group, and let \(H\) be a subset of \(G\) which is itself a group under the same operations as \(G\). Then we say that \(H\) is a subgroup of \(G\).
}

\prop{The Subgroup Lemma}{
    Let \(G\) be a group and \(H\) a non-empty subset of \(G\). Then \(H\) is a subgroup of \(G\) if and only if it is closed under the group operation and inverse.
}

\defn{Cyclic Groups}{
    A group \(G\) is said to be cyclic if there exists an element \(g \in G\) such that \(G = \langle g \rangle\), i.e. \(G\) is generated by a single element.
}

\defn{Order of a Group and Element}{
    The order of a finite group \(G\) is the number of elements in \(G, \abs*{G}\). \\
    The order of an element \(g\) in a group \(G\) is the smallest positive integer \(n\) (if any) such that \(g^n = e\). We write \(o(g)\) for the order of the element \(g\).
}

\prop{Distinct Powers of Elements}{
    If \(g \in G\) has order \(n\), then the elements \(e, g, g^2, \dots, g^{n-1}\) are all distinct.
}

\thrm{Isomorphic Cyclic Groups}{
    Two finite cyclic groups are isomorphic if and only if they have the same order.
}

\thrm{Prime Order is Isomorphic is Cycle Group}{
    Any group of prime order \(p\) is isomorphic to the cyclic group \(C_p\).
}

\prop{Groups of Prime Order are Abelian}{
    Any group of prime order is abelian.
}

\thrm{All Subsets Closed under Operation are Subgroups}{
    Let \(G\) be a group with operation \(\star\). If \(H\) is a non-empty finite subset of \(G\) that is closed under \(\star\), then \(H\) is a subgroup of \(G\).
}

\thrm{Lagrange's Theorem}{
    If \(G\) is a finite group and \(H\) is a subgroup of \(G\), then \(\abs*{H}\) is a factor of \(\abs*{G}\).
}

\defn{Left Coset}{
    Let \(G\) be a group and \(H\) a subgroup of \(G\). For any \(g \in G\) we define the left coset of \(H\) by \(g\) to be
    \[gH = \{ gh \mid h \in H \}.\]
    If we used additive notation we would write a coset of \(H\) as \(g + H\).
}

\prop{Order of Elements in a Group is a Divisor of the Group}{
    Suppose \(G\) is a group of finite order and \(g \in G\). Then \(g^\abs*{G} = e\).
}

\thrm{Fermat's Little Theorem}{
    If \(p\) is a prime and \(a\) is not a multiple of \(p\), then \(a^{p-1} \equiv 1 \pmod p\).
}

\prop{Corollary of Fermat's Little Theorem}{
    If \(p\) is prime and \(a\) is any integer, then \(a^p \equiv a \pmod p\).
}

\thrm{Euler's Theorem}{
    Let \(n\) be a positive integer, and let \(a\) be an integer relatively prime to \(n\). Then \(a^{\varphi(n)} \equiv 1 \pmod n\)
}